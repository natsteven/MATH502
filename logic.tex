%! Author = nathanaelsteven
%! Date = 4/1/25

\documentclass[11pt]{amsart}
\usepackage{amsmath}
\usepackage{amssymb}
\usepackage{amsthm}
\usepackage{mathrsfs}

\begin{document}

    Missing definitions of lexicon, atomic formula, terms, structure, etcetera.

    closure with quantifiers? universal clsoures
    \\ \\
    \newcommand{\FF}{\mathcal{F}}
    \newcommand{\LL}{\mathcal{L}}
    \newcommand{\TT}{\mathcal{T}}

    \underline{terms} built from $\FF \cup VAR$
    \underline{atomic formulas} $P^{\TT_0 ...}$ or $= \TT_1 \TT_2$
    \underline{formulas} use connectives, quantifiers, etc.


    - $V(\TT)$ is the set of variables which occur in $\TT$.
    - $\varphi$, let $V(\varphi)$ be the set of variables which have a free occurrence $\varphi$.
    \\
    For an $\LL$-structure $\mathfrak{A}=(A,I)$, a term $\TT$, and an assignment $\sigma$ for $\TT$ in $A$, define $val_{\mathfrak{A}(\TT)[\sigma]}$
    \\
    $\LL = \{+,\cdot\},\{<\}$ \\
    $\mathfrak{A}: A = \mathbb{R}$ \\
    $\TT: \cdot z + x y$ \\
    $V(\TT) = \{x,y,z\}$ \\
    $\sigma: [x y z][3 4 5]$ \\
    $val_\mathfrak{A}(x)[\sigma]=3$ \\

    For an $\LL$-structure $\mathfrak{A}$, an atomic formula $\varphi$, and an assignment $\sigma$ for $\varphi$ in $A$, define $val_{\mathfrak{A}(\varphi)[\sigma]}$ in $\{F,T\}=\{0,1\}$ by \\
    $val_\mathfrak{A}(P)[\sigma] = P_\mathfrak{A}$ if $P \in P_0$
    etcetera

    $\mathfrak{A}$ satisfies $\varphi$, $\mathfrak{A} \models \varphi$ when $V(\varphi) = \emptyset$?

    A set $\Sigma$ of $\LL$-sentences is semantically consistent or satisfiable $con_{\models}(\Sigma)$, if there is an $\LL$-structure $\mathfrak{A}$ such that $\mathfrak{A} \models \Sigma$

    Reductio ad Absurdum: Let $\Sigma$ be a set of $\LL$-sentences and $\Psi$ an $\LL$-sentence then:
    (1) $\Sigma \models \Psi \iff \Sigma \cup \{\neg \Psi\}$ is semantically inconsistent.

    A formula is logically valid if a structure $\mathfrak{A}$ satisfies it for $\sigma$ or all strucutre and assignment for psi in the structure
    if two formulas are both universal closures of a formula then they are logically equivalent (irrelevant what assignments are used)
    formula are equivalent with respect to a set of logical sentences if the universal closure is true in all models of the sentences.
    two terms are equivalent with respect ot logical sentences if for all structures that satisfy the sentences and all assignment for the two terms in the structure, the two terms have the same value.

    axiom of choise and zorn's lemma are equivalenet with respect to ZF.
    supppse sigma includes associativity and the two terms are the sides of the equivalence, then they are equivalent with respect to the set of sentences.
    \\
    let beta and tau be terms and x a variable.
    then beta of x to tau, or beta of x / tau, is the result of replcing all occurences of x in beta with tau.
    \\
    let psi be a formula, x a variable, tau a term, then psi x assigned to tau, is the resutl of replacing all free occurences of x by tau in psi.
    example problem is if exists y where x < y, and we substitute in y for x we get a potential problem.
    if we tried ot replace y itself nothing would change because it isn't free, it has a quantifier.
    \\
    a term tau is free for x in a formula phi if no free occurrence of x is inside the scope of a quantifier exists y or forall y, where y is a variable in tau.
    tau is y for above is not free for x because there is an occurence of x inside the scope of a quantifier applied to y.
    If tau is free for x in phi, then the following are logically valid: forall x, phi(x) implies phi(tau), phi(tau) implies exists x, phi(x).
    note that the above example that was not free for x fails.
    \\
    a reduct of the originsl strucutre is a restriction to the symbols in some other structure that is a substructure of the original and vice verse an expansion.
    \\
    isomporhpisms and shit.
    \\
    a set of axiom (L-sentences) sigma is complete with respect to the strucutre(L) if sigma is semantically consistent and for all L-sentences, phi, either sigma logically implies(models/satisfies) phi or not phi.
    \\
    ZFC is not complete.
    there are some natural complete theories.
    any `sufficiently complicated` (able to interpret arithmetic) and `nicely axiomatized` (finite list or sufficiently concrete) set of axioms will not be complete.
    \\
    let a be an l-structure.
    the theory of a, Th(a) is the set of all l-sentences true in a.
    Th(a) is complete.
    for any formula(l-sentence) phi exactly one : a satisfies phi or a satisfies not phi.
    \\

    \subsection{Tautologies}
    a basic formula is one which (when expressed in polish/prefix notation) does not begin with a propositional connective. \\
    ex: forall x P(x): forall x P x, or forall x (P(x) then q(x)) : forall x implies p x q x, are both basic\\
    ex (otoh): p(x) implies q(x) : implies p x q x is not basic.
    \\ A truth assignment for L is a function wich assigns a truth value to every basic formula.
    given such a function, we can extend it to all formulas by recursion.
    note that we don't require consistency so truth assignments can be semantically inconsistent (no structures or meaning to symbols).
    \\ a formula phi is a propositional tautology if it is true under every truth assignment.

    \subsection{April 8th}

    \textbf{Logical Axioms}
    \\ (1) Propositional tautologies
    \\ (2) $\varphi \rightarrow \forall x \varphi$if $x$ is not free in $\varphi$
    \\ (3) $\forall x (\varphi \rightarrow \psi) \rightarrow (\forall x \varphi \rightarrow \forall x \psi)$
    \\ (4) $\forall x \varphi \rightarrow \varphi(x / \tau)$ where $\tau$ is free for $x$ in $\varphi$
    \\ (5) $\varphi(x / \tau) \rightarrow \exists x \varphi$ where $\tau$ is free for $x$ in $\varphi$
    \\ (6) $\forall x \neg \varphi \leftrightarrow \neg \exists x \varphi$
    \\ (7) $x=x$
    \\ (8) $x=y \leftrightarrow y=x$
    \\ (9) $(x=y \wedge y = z) \rightarrow x=z$
    \\ (10)$ (x_1=y_1 \wedge \dots \wedge x_n =y_n) \rightarrow f(x_1 \dots x_n) = f(y_1 \dots y_n)$ where $f \in \mathscr(F_n)$ for some $n \geq 1$
    \\ (11) same but for predicates

    Note all of these are logically valid

    Def: For $\Sigma$ a set of $\mathcal{L}$-sentences and $\varphi$ an $\mathcal{L}$-senetence, a \underline{formal proof} of $\varphi$ from $\Sigma$ is a finite sequnce of $\mathcal{L}$-sentences $\varphi_1, \dots, \varphi_n$
    where for each $i = 0 \dots n$ $\varphi_i$ is either a lgoical axiom, a memeber of $\Sigma$ or there are $j,k < i$ so that $\varphi_i$ follows from $\varphi_j$ and $\varphi_k$ by modus ponens i.e. $\varphi_k$ is $\varphi_j \implies \varphi_i$ and $\varphi_n = \varphi$
    \\
    Def: We say ``$\Sigma$ proves $\varphi$``, written $\Sigma \vdash \varphi$ if there is a formal proof of $\varphi$ from $\Sigma$.
    \\
    \newcommand{\set}[1]{\{#1\}}
    Ex: $\Sigma = \set{\forall x (p(x) \wedge q(x))}$ \\
    $varphi : \forall x p(x)$ \\
    Show $\Sigma \vdash \varphi$ \\
    $\varphi_0: \forall x (p(x) \wedge q(x))$ \\
    $\varphi_1: \forall x ((p(x) \wedge q(x) \rightarrow p(x)) \rightarrow (\forall x (p(x) \wedge q(x)) \rightarrow \forall x p(x)))$ \\
    \\ this is one of the logical axioms \\
    $\varphi_2: \forall x ((p(x) \wedge q(x)) \rightarrow p(x))$ - universal closure of propositional tautology: $(p(x) \wedge q(x)) \rightarrow p(x)$
    \\ $\varphi_3: \forall x (p(x) \wedge q(x)) \rightarrow \forall x p(x)$ - modus ponens from $\varphi_1$ and $\varphi_2$ \\
    $\varphi_4: \forall x p(x)$ - modus ponens from $\varphi_0$ and $\varphi_3$ \\ q.e.d.

    \emph{\underline{Soundness Theorem}}: If $\Sigma \vdash_{\LL} \varphi$ then $\Sigma \models \varphi$.

    \begin{proof}
        Assume $\Sigma \vdash_{\LL} \varphi$ and let $\mathfrak{A}$ be an $\LL$-structure such that $\mathfrak{A} \models \Sigma$.
        Show that $\mathfrak{A} \models \varphi$.
        Let $\varphi_1, \dots, \varphi_n$ be a formal proof of $\varphi$ from $\Sigma$.
        By induction on $i$ show that $\mathfrak{A} \models \varphi_i$ for all $i = 0, \dots, n$.
        If $\varphi_i \in \Sigma$ then $\mathfrak{A} \models \varphi_i$ by assumption.
        If $\varphi_i$ is a logical axiom, then $\varphi_i$ is logically valid, so true in any $\LL$-structure.
        If $\varphi_i$ is obtained from $\varphi_j$ and $\varphi_k$ by modus ponens, then $\mathfrak{A} \models \varphi_j$ and $\mathfrak{A} \models \varphi_k$, so $\mathfrak{A} \models \varphi_i$.
        Finally, if $\varphi_n = \varphi$, then $\mathfrak{A} \models \varphi$.
    \end{proof}

    Note: Logical axioms are chosen from:
    (1) All are valid (needed for soundness)
    (2) When $\LL$ is ``nice`` e.g., finite, it is \underline{decidable} with eithere a sentence or a logical axiom
    (3) Have enough for \underline{completeness}.

    Note: Develop some ``strategies for proofs`` i.e. theorems which allow us to show certain sentences are provable without having to explicitly exhibit a proof.

    Lemma: \underline{Deduction Theorem}: \\
    $\Sigma \vdash_{\LL} \varphi \rightarrow \psi \text{iff} \Sigma \cup \set{\varphi} \vdash_\LL \psi$

    \begin{proof}
        $\implies$ \\
        Suppose $\Sigma \vdash_{\LL} \varphi \rightarrow \psi$.
        $\varphi_0 \dots \varphi_n$ be a proof of $\varphi \rightarrow \psi$ from $\Sigma$.
        Let $\varphi_{n+1} = \varphi$.
        $\varphi_{n+2} = \psi$.
        $\varphi_{n+1}$ is in $\Sigma \cup \set{\varphi}$
        $\varphi_{n+2}$ is derived from $\varphi_{n+1}$ and $\varphi_n = \varphi \rightarrow \psi$ by modus ponens.
        So $\varphi_0 \dots \varphi_{n+2}$ is a proof of $\psi$ from $\Sigma \cup \set{\varphi}$.
        So $\Sigma \cup \set{\varphi} \vdash_\LL \psi$.
        \\
        $\impliedby$ \\
        Suppose $\Sigma \cup \set{\varphi} \vdash_\LL \psi$.
        Let $\psi_0 \dots \psi_n$ be a proof of $\psi$ from $\Sigma \cup \set{\varphi}$.
        For each $i$ let $\varphi_i$ be $\varphi \implies \psi_i$.
        By induction on $i$ show $\Sigma \vdash_\LL \varphi_i$.
        Case 1: $\psi_i$ is a logical axiom or an element of $\Sigma$.
        Then: $\psi_i$, $\psi_i \implies (\varphi \implies \psi_i)$(tautology), then $\varphi \implies \psi_i$
        So $\Sigma \vdash_\LL \varphi \rightarrow \psi_i$.
        Case 2: $\psi_i$ is $\varphi$ then $\varphi \implies \psi_i$ is a tautology.
        So $\Sigma \vdash_\LL \varphi \implies \psi_i$.
        Case 3: There are $j$ and $k$ < $i$ so $\psi_i$ is derived from $\psi_j$ and $\psi_k = \psi_j \implies \psi_i$.
        Assume $\Sigma \vdash_\LL \varphi \rightarrow \psi_j$ and $\Sigma \vdash_\LL \varphi \rightarrow (\psi_j \rightarrow \psi_i)$.
        $(\varphi \rightarrow \psi_j) \rightarrow((\varphi(\psi_j \rightarrow \psi_i))\rightarrow(\varphi \rightarrow \psi_i))$ a propositional tautology.
        $(\varphi \rightarrow(\psi_j \rightarrow \psi_i)) \rightarrow (\varphi \rightarrow \psi_i)$ by modus ponens.
        $\varphi \rightarrow \psi_i$ by m.p. (took two previous proofs and added a tautology).
        This list is a proof of $\varphi \rightarrow \psi_i$ from $\Sigma$.
        So $\Sigma \vdash_\LL \varphi \rightarrow \psi$.
    \end{proof}

    Def: A set of $\Sigma$ of $\LL$-sentences is \underline{syntactically inconsistent}, $\neg con_{\vdash, \LL}(\Sigma)$
    if there is some $\LL$-sentences $\varphi$ so that $\Sigma \vdash_\LL \varphi$ and $\Sigma \vdash_\LL \neg \varphi$.

    Note: By Soundness Theorem $\neg con_{\vdash, \LL}(\Sigma)$ implies $\neg con_{\models}(\Sigma)$ since if $\Sigma \vdash_\LL \varphi$ and $\Sigma \vdash_\LL \neg \varphi$ then $Sigma \models \varphi$ and $\Sigma \models \neg \varphi$.
    So any $\LL$-strucutre $\mathfrak{A}$ with $\mathfrak{A} \models \Sigma$ must also have $\mathfrak{A} \models \varphi$ and $\mathfrak{A} \models \neg \varphi$ which can't happen so no $\LL$-structure cna satisfy $\Sigma$.

    So $Con_{\models})\Sigma$ implies $Con_{\vdash,\LL}(\Sigma)$\\
    Lemma TFAE:
    (1) $\Sigma$ is syntactically inconsistent in $\LL$.
    (2) $\Sigma \vdash_\LL \psi$ for every $\LL$-sentence $\psi$.

    Proof by Contradiction:
    (1) $\Sigma \vdash_\LL \varphi$ iff $\neg Con_{\vdash, \LL}(\Sigma \cup \set{\neg \varphi})$
    (2) $\Sigma \vdash_\LL \neg \varphi$ iff $ \neg Con_{\vdash, \LL}(\Sigma \cup \set{\varphi})$

    We say a formula $\psi$ follows \underline{tautologically} from $\varphi_1 \dots \varphi_n$ si the formula $(\varphi_1 \wedge \dots \wedge \varphi_n) \rightarrow \psi$ is a propositional tautology.
    \\ \underline{Tautological Reasoning}: If $\psi$ follows tautologically from $\varphi_1 \dots \varphi_n$ then $\set{\varphi_1 \dots \varphi_n} \vdash_\LL \psi$.
    \\ \underline{Transitivity of $\vdash$}: if


    \section{April 10}
    Recall a proof is s finite sequence of formulas from sgima such that they are either logical axioms, members of sigma or obtained from previous formulas by modus ponens.
    \\
    \underline{Quantifier Rules}
    Let sigma be a set of sentences in some lexicon L.
    psi will be some setnence, and phi(x) some formula with x as the only free variable.
    tau is a variable-free term.
    \\
    so phi(x/tau) is a sentence.
    L` is L with a new constant.
    Then:\\
    (UI) $\set{\forall x \varphi(x)} \vdash_{\LL} \varphi(\tau) also \vdash_{\LL`} \varphi(c)$ \\
    (UG) $\sigma \vdash_{\LL`} \varphi(c) \implies \sigma \vdash_{\LL} \forall x \varphi(x)$ \\
    (EI) $\sigma \cup \set{\varphi(c)} \vdash_{\LL`} \psi \implies \sigma \cup \set{\exists x \varphi(x)} \vdash_\LL \psi$\\
    (EG) $\varphi(\tau) \vdash_\LL \exists x \varphi(x)$

    Universal, Existential, Instantiation, Generalization
    if for all values a sentence is true then we can plug anything in for x.
    sigma is a set of sentences without a constant. if it holds for an arbitrary element then it holds for all elements.
    ...

    example showing formal proff of for all y phi(y) proves for all x phi(x)

    we can also show that something is provable without giving a proof.

    \underline{Completeness Theorem}: If $\Sigma$ is a set of $\LL$-sentences and $\varphi$ is an $\LL$-sentence, then $\Sigma \models \varphi$ iff $\Sigma \vdash_\LL \varphi$.

    \section{April 15}
    Equivalence classes, herbrand model, atomic sentence, universal closure \\

    maximally consistent set to help prove left to right of completeness, speficially problem of trichotomy model/satisfaction

    \section{April 17}
    induction proof of herbrandt model and maximal consistency stuff regarding connectives and quantifiers.
\end{document}