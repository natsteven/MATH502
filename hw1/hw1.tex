% Math 402/502 Homework



\documentclass[11pt]{amsart}


\pagestyle{empty}
\thispagestyle{empty}

\usepackage{graphicx}

\usepackage{amsmath}
\usepackage{amssymb}
\usepackage{latexsym}
\usepackage{amsopn}
\usepackage{amsthm}
\usepackage{setspace}

\def\bbN{{\mathbb N}}
\def\bbR{{\mathbb R}}
\def\bbQ{{\mathbb Q}}
\def\bbZ{{\mathbb Z}}
\def\bbF{{\mathbb F}}
\def\bbE{{\mathbb E}}
\def\bbP{{\mathbb P}}

\DeclareMathOperator{\dom}{dom}

\newcommand{\set}[1]{\left\{\,#1\,\right\}}
\newcommand{\NN}{\mathbb N}
\newcommand{\ZZ}{\mathbb Z}
\newcommand{\QQ}{\mathbb Q}
\newcommand{\RR}{\mathbb R}


\usepackage{bbold}


\newcommand{\hint}[1]{{\small \em \noindent [Hint: #1]}}


\begin{document}
\begin{spacing}{1.2}
\begin{center}
{\Large Math 402/502 Homework 1 -- due Friday, January 24}

\vspace{.5em}
\textbf{\emph{Nat Steven}}
\ \\
\end{center}

%\ \\
%\noindent \emph{\small Instructions: Show all your work to receive full credit. Please complete the problems in order, state what you are doing, write neatly, and staple. \\
%Homework may be submitted either on paper or through Canvas.}
%\ \\
%
 \begin{enumerate}

\item Let variables range over the natural numbers $\bbN$, and let $P(n)$ be the relation that $n$ is prime, $L(n,m)$ the relation that $n<m$, and $E(n)$ the relation that $n$ is even. 

\ 
\begin{enumerate}
\item Express the following symbolic statement as an English sentence:
\[ \forall n \exists m (L(n,m) \wedge P(m)) \]

\vspace{1em}
\emph{For all natural numbers $n$, there exists a natural number $m$ such that $n$ is less than $m$ and $m$ is prime.
Alternatively, there always exists a prime natural number greater than any given natural number.}

\vfill
\item Express the following English sentence as a symbolic statement. Do not use the $\exists !$ abbreviation.

``There is a unique natural number $n$ which is both even and prime.''

\vspace{1em}

\[
 \exists n ((E(n) \wedge P(n)) \wedge (\forall m (E(m) \wedge P(m)) \implies m=n))
\]


\vfill
\end{enumerate}

\newpage

\item Let $\oplus$ be a new connective representing exclusive or (XOR), so that $P \oplus Q$ is true precisely when exactly one of $P$ and $Q$ is true. Write a truth table for $\oplus$.

\vspace{1em}
\begin{tabular}{c|c|c}
    $P$ & $Q$ & $P \oplus Q$ \\
    \hline
    T & T & F \\
    T & F & T \\
    F & T & T \\
    F & F & F
\end{tabular}

\newpage

\item Consider the following three statements, where $P$, $Q$, and $R$ are three predicates:

\begin{itemize}
\item {\bf Statement 1:} $(P \rightarrow Q) \rightarrow R$
\item {\bf Statement 2:} $P \rightarrow (Q \rightarrow R)$
\item {\bf Statement 3:} $(P \wedge Q) \rightarrow R$
\end{itemize}

\begin{enumerate}
\item Show  Statement 2 is logically equivalent to Statement 3.

\vspace{1em}
\begin{minipage}{0.45\textwidth}
    \centering
    \begin{tabular}{c|c|c|c}
        $P$ & $Q$ & $R$ & $P \rightarrow (Q \rightarrow R)$ \\
        \hline
        T & T & T & T \\
        T & T & F & F \\
        T & F & T & T \\
        T & F & F & T \\
        F & T & T & T \\
        F & T & F & T \\
        F & F & T & T \\
        F & F & F & T
    \end{tabular}
\end{minipage}
\hfill
\begin{minipage}{0.45\textwidth}
    \centering
    \begin{tabular}{c|c|c|c}
        $P$ & $Q$ & $R$ & $(P \wedge Q) \rightarrow R$ \\
        \hline
        T & T & T & T \\
        T & T & F & F \\
        T & F & T & T \\
        T & F & F & T \\
        F & T & T & T \\
        F & T & F & T \\
        F & F & T & T \\
        F & F & F & T
        \end{tabular}
\end{minipage}

\vspace{1em}
\emph{
    The above truth tables show that the Statement 2 \& 3 are equivalent as for each permuation of $P$, $Q$, and $R$ the two statements have the same truth value.
}


\vfill
\item Show  Statement 1 is not logically equivalent to Statement 2.

\vspace{1em}
\begin{minipage}{0.45\textwidth}
    \centering
    \begin{tabular}{c|c|c|c}
        $P$ & $Q$ & $R$ & $(P \rightarrow Q) \rightarrow R$ \\
        \hline
        T & T & T & T \\
        T & T & F & F \\
        T & F & T & T \\
        T & F & F & T \\
        F & T & T & T \\
        F & T & F & F \\
        F & F & T & T \\
        F & F & F & F
    \end{tabular}
\end{minipage}
\hfill
\begin{minipage}{0.45\textwidth}
    \centering
    \begin{tabular}{c|c|c|c}
        $P$ & $Q$ & $R$ & $P \rightarrow (Q \rightarrow R)$ \\
        \hline
        T & T & T & T \\
        T & T & F & F \\
        T & F & T & T \\
        T & F & F & T \\
        F & T & T & T \\
        F & T & F & T \\
        F & F & T & T \\
        F & F & F & T
    \end{tabular}
\end{minipage}

\vspace{1em}
\emph{
    The above truth tables show that the Statement 1 \& 2 are not equivalent as there are some permuations of $P$, $Q$, and $R$ where the two statements have different truth values.
}

\vfill
\end{enumerate}

\newpage

\item The symmetric difference of two sets, $X \Delta Y$, is defined to consist of those sets which are elements of exactly one of the two sets.

Write a formula in the language of set theory which expresses the statement $Z = X \Delta Y$.

\vspace{1em}

    \[
        Z = {(m \in X : m \notin Y) \wedge (m \in Y : m \notin X)}
    \]
or
  \[
      Z = X \cup Y - X \cap Y
  \]

\end{enumerate}

\end{spacing}

\end{document}
