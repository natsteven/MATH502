\documentclass{article}

% Language setting
% Replace `english' with e.g. `spanish' to change the document language
\usepackage[english]{babel}

% Set page size and margins
% Replace `letterpaper' with `a4paper' for UK/EU standard size
\usepackage[letterpaper,top=2cm,bottom=2cm,left=3cm,right=3cm,marginparwidth=1.75cm]{geometry}

% Useful packages
\usepackage{amsmath}
\usepackage{graphicx}
\usepackage[colorlinks=true, allcolors=blue]{hyperref}
\usepackage{amsfonts}
\usepackage{amssymb}
\usepackage{upgreek}
\usepackage{amsthm}

\title{Set Theory and Logic Notes}
\author{Nat}

\newcommand{\defn}{\textbf{Def: }}
\newcommand{\dfn}{\textbf{Def: }}

\newcommand{\set}[1]{\{#1\}}


\begin{document}
    \maketitle

    \section{Introduction}

% Group theory. Group is a set of objects .

% G - a set

% . - binary function

% -1 - inverse function

% 1 - identity

% Axioms of group theory:

% For all a,b,c in G

% associative - a.b. . c =

    We can abbreviate statements, e.g. $\neg (x=y)$ and $x \neq y$
    there exists a unique x satisfying $P(x) \rightarrow \exists! x P(x)$
    expansion: $\exists x (P(x) \and \forall y (P(y) \rightarrow y=x))$
    note we can give a tree which shows how a formula is constructed.
    The connectives and quanitifers are the internal nodes and root, terminals are predicates/relations.

    Connectives are functions on truth values, can be described using truth tables.
    Examples:

    \begin{tabular}{c|c}
        P  &  $\neg$ P\\
        \hline
        T  & F \\
        F & T
    \end{tabular}

    \vspace{1em}

    \begin{tabular}{c|c|c}
        P & Q & P $\wedge$ Q\\
        \hline
        T & T & T \\
        T & F & F \\
        F & T & F \\
        F & F & F \\
    \end{tabular}

    \vspace{1em}

    \begin{tabular}{c|c|c}
        P & Q & P $\implies$ Q\\
        \hline
        T & T & T \\
        T & F & F \\
        F & T & T \\
        F & F & T \\
    \end{tabular}

    \vspace{1em}
    \begin{tabular}{c|c|c}
        P & Q & P $\iff$ Q\\
        \hline
        T & T & T \\
        T & F & F \\
        F & T & F \\
        F & F & T \\
    \end{tabular}

    \vspace{1em}
    \textbf{Logical Equivalence}
    \begin{itemize}
        \item Two statements are logically equivalent if for any truth assignments of truth values to the predicates, both statements have the same resulting truth value. $s_1 \equiv s_2$
    \end{itemize}

    \emph{Ex}: $P \implies Q$ is logically equivalent to $(\neg P ) \vee Q$

    \begin{tabular}{c|c|c}
        P & Q & $(\neg P) \vee Q$\\
        \hline
        T & T & T \\
        T & F & F \\
        F & T & T \\
        F & F & T \\
    \end{tabular}
    \vspace{1em}

    \textbf{Def:} A statement is \emph{valid} if it is true for any truth assignment to the predicates
    \emph{Ex:} $P \vee \neg P$


    \section{Set Theory}
    \begin{itemize}
        \item First to study this was Cantor
        \item \"size\" of infinite sets
        \item two sets have the same size if we can put a bijection between the two
        \item properties of sets.
    \end{itemize}

    \underline{Idea:} Sets are determined by their members. $\{1,2,3,4\} \equiv \{2,3,1,4\}$

    \underline{Goals}
    \begin{itemize}
        \item Formalize Sets
        \item Introduce Axioms of Zermelo-Frankel Set Theory with choice (ZFC)
        \item Cardinality
        \item represent standard mathematical objects as sets
        \item ZFC as a foundation for all of mathematics
    \end{itemize}

    \underline{Note}
    \begin{itemize}
        \item Everything will be a set
        \item Language of set theory will be built from two predicates!
        \begin{itemize}
            \item = : equality
            \item $\in$ : set membership. $x\in y$ will be true when x is an element of y
        \end{itemize}
    \end{itemize}

    \emph{Ex:} $\emptyset$ , empty set. $\{\emptyset\} \neq \emptyset$. Then we can build natural numbers. 2: $\{\emptyset,\{\emptyset\}\}$

    \vspace{1em}
    \noindent\textbf{\emph{The Axioms of ZFC}}

    \underline{Language} built from =, $\in$
    \begin{itemize}
        \item Axiom 0: \emph{set existence}, There is a set. $\exists x (x=x)$. (axiom 0 will be redundant later).
        \item Axiom 1: \emph{extensionality}, Sets are determined by their elements
        \item Axiom 2: \emph{Foundation} There are no infinite descending chains of sets with respect to elementhood. i.e. No $..... \in x_2 \in x_1 \in x_0$. implies no element is an element of itself
        \item Axiom 3: \emph{comprehension scheme}
        \begin{itemize}
            \item For every formula $\varphi(x)$ and every set $Z$ we can form the set $\{x \in Z : \varphi(x)\}$.
            Those x which are elements of $Z$ which satisfy $\varphi(x)$
            \item one axiom for each formula $\varphi(x)$
        \end{itemize}
        \item Axiom 4: \emph{pairing}, Given sets $x$ and $y$, there is a set containing both $x$ and $y$
        \item Axiom 5: \emph{union}, Given a family of sets, $F$, there is a set containing the union of all of the sets in $F$
        \item Axiom 6: \emph{replacement scheme}, Given a formula $\varphi$ which defines a function $f$ and a set $Z$, there is a set which consists of the range of $f$ applied to $Z$
        \item Axiom 7: \emph{infinity} (makes 0 redundant), there is an infinite set
        \item Axiom 8: \emph{power set}, given a set $X$ there is a set $\mathcal{P}(X)$ which contains as elements all of the subsets of $X$
        \item Axiom 9: \emph{axiom of choice (AC)}, given a collection of disjoint non-empty sets, there is a set which contains exactly one element from each set in the collection.
        \begin{itemize}
            \item \emph{Ex}: $\mathbb{R}$ $x~y$ if $x-y \in \mathbb{Q}$

            $[x] = \{y: y~x\}$

            $\mathcal{e}=\{[x]: x \in \mathbb{R} \}$

            if x not ~ y then [x] and [y] are disjoint$ [x] \neq \emptyset$
            \item AC says there is a set containing exactly one element of each equivalence set
            \item this is called a \emph{Vitali} set.
        \end{itemize}

        \underline{Note:}
        \begin{itemize}
            \item ZFC denotes Axioms 1-8
            \item ZF denotes Axioms 1-8
            \item ZC and Z denote ZFC and ZF with replacement scheme removed
            \item Z-, ZF-, ZC-, ZFC- denote removing foundation axiom
            \item \"most\" standard mathematics can be don in ZC-
        \end{itemize}

    \end{itemize}


    \subsection{Axiom 0 - Set Existence}
    \subsection{Axiom 1 - Extensionality}
    \subsection{Axiom 2 - Foundations}
    \subsection{Axiom 3 - Comprehension Scheme}
    \subsection{Axiom 4 - Pairing}

    \textbf{Recall} - Ordered Pair: $<x,y> = \{\{x,y\},\{x\}\}$

    \subsection{Axiom 5 - Union}
    \[\forall \mathcal{F} \exists A \forall Y \forall x ((x \in Y \wedge Y \in mathcal{F}) \implies x \in A
    \]
    \underline{Ex} $Y_1, Y_2, \dots, Y_k \in \mathcal{F}$

    $A \supseteq Y_1 \cup Y_2 \dots \bigcup Y_k$

    \vspace{1em}
    \underline{Note}: Comprehension allows us to form the et whose elements are exactly those x so that $x \in Y$ for some $Y \in \mathcal{F}$. Denote this set $\bigcup \mathcal{F}$, i.e. $Y^\cup \in \mathcal{F}^Y$

    \underline{Ex:} $x \cup y = \bigcup \{x,y\} =\{z : z \in x \cup z \in y\}$

    $x \cup y \cup z = \big \{x,y,z\}$

    \vspace{1em}
    \underline{Ex:} $\bigcup\{x\} = \{z : z \in x\} = x$

    $ \bigcup\{\{x\}\} = \{x\}$ and $\bigcup \bigcup\{\{x\}\} = x$

    \vspace{1em}
    \underline{Def}: When $\mathcal{F} \neq \emptyset$ we can define intersection. $\bigcap \mathcal{F} = \bigcup_{Y \in \mathcal{F}} Y$

    $=\{x : \forall Y \in \mathcal{F} (x \in y)\}$ (naive)

    $\{x\in Y_0 : \forall Y \in \mathcal{F}(x \in y)\}$ (proper use of comprehension.)

    where $Y_0$ is some element of $\mathcal{F}$

    \underline{Note} If $\mathcal{F} = \emptyset$ then $\bigcap \mathcal{F} would be a universal set$

    \vspace{1em}
    \underline{Def:} The \emph{ordinal successor function}

    $S(x) = x \cup \{x\}$

    \underline{Ex:} $S(\emptyset) = \emptyset \cup \{\emptyset\} = \{\emptyset\}$ and $S(\{\emptyset\}) = \{\emptyset\} \cup \{\{\emptyset\}\} = \{\emptyset,\{\emptyset\}\}$

    $0 = \emptyset$

    $1 = S(0)$

    $2 = S(1) = S(S(0))$

    $n+1 = S(n)$

    \underline{Note} $n = \{ k : k < n\}$, n has exactly n elements. $n \leq m iff n \subseteq m$ and $n < m iff n \in m$

    \underline{Informally} a natural number is a set obtained by some finite number of iteration of the successor applied to $\emptyset$


    \subsection{Relations and Functions}

    \underline{Def} R is a (binary) relation if R is a set of ordered pairs, i.e. $\forall z \in R \exists x \exists y z = <x,y>$

    \underline{Note} If $\phi(z) is a formula the abbreviation \forall z \in R \phi(z) and \exists z \in R \phi(z)$ means

    $\forall Z \in R \phi(z) means \forall z (z \in R \implies \phi(z))$

    $\exists z \in R \phi(z) means \exists z (z \in R \and \phi(z))$

    \vspace{1em}
    \underline{Def} $dom(R) =^* \{x : \exists y <x,y> \in R\}$

    $ran(R) =^* \{y: \exists x <x,y> \in R\}$

    * need to see that there is a set $Z$ so that all $x$ and $y$ needed in these definitions are elements of $Z$

    If $<x,y> \in R$

    $\{\{x\},\{x,y\}\}$

    $\{x\} \in \bigcup R$ and $\{x,y\} \in \bigcup R$ also
    $x \in \bigcup \bigcup R$ and $x,y \in \bigcup \bigcup R$

    So $dom(R) = \{x \in \bigcup \bigcup R : \exists y <x,y> \in R\}$
    $ran(R) = \{y \in \bigcup \bigcup R : \exists x <x,y> \in R\}$

    \vspace{1em}
    \underline{Def} Properties of Relations

    Let $R$ be a relation and $A$ some set. We will say that $R$ is\ldots
    \begin{itemize}
        \item \emph{transitive on A} if $\forall x,y,z \in A$
        \[((x R y \and y R z) \implies x R z\]
        where $x R y$ means $<x,y> \in R$
        \item \emph{reflexive on A} if $\forall x \in A (x R x)$
        \item \emph{irreflexive on A} if $\forall x \in A \neg(x R x)$
        \item \emph{symmetric on A} if $\forall x,y \in A (xRy \leftrightarrow y R x)$
        \item \emph{partially orders A strictly} if R is irreflexive on A and transitive on A
        \item \emph{satisfies trichotomy on A} if  $\forall x,y \in A (x R y \lor y R x \lor x=y)$
        \item \emph{totally orders A strictly} if R partially orders A strictly and R satisfies trichotomy on A
        \item \emph{equivalence relation on A} if R is reflexive on A, symmetric on A, and transitive on A.
    \end{itemize}

    \underline{Ex} $A = \mathbb{N}$ and  $x R y$ if $x \equiv_3 y$. i.e. x-y is a multiple of 3. R is an equivalence relation on A.

    e.g. $x R y$ and $y R z$ then $x-y =3j$ for  some j and $y-z = 3k$ for some k
    $x-z=(x-y) + (y-z) = 3j + 3k = 3(j+k)$ so $x R z$

    \underline{Ex} $A = \mathbb{N}$  $x R y$ if x divides y is reflexive on A and transitive on A.

    \underline{Ex} $A = \mathbb{N}$ $x R y$ if $x < y$ R totally order A strictly

    \underline{Def} Let R be a relation and A a set. The restriction of R to A, $R \upharpoonright A = \{<x,y> \in R : x \in A \}$


    \underline{Def}
    \begin{itemize}
        \item R is a \emph{function} if R is a relation $\forall x \in dom(R) \exists! y <x,y> \in R$
        \item F, $x \in dome(R)$ we will write R(x) to mean the unique y with $<x,y> \in R$
    \end{itemize}

    \underline{Ex} $F = \{<0,0> ,<1,0>,<2,2>\}$ is a function $dom(F) = \{0,1,2\}$ and $ran(F) = \{0,2\}$ $F(0) = 0, F(1) = 0, F(2) = 2$

    \underline{Def} $F : A \rightarrow B$ means F is a function $dom(F) = A$ and $ran(F) \subseteq B$
    F above $F: \{0,1,2\} \rightarrow \{0,1,2\}$
    $F : A \rightarrow_{onto} B$ or $F: A \twoheadrightarrow B$
    if $F: A \rightarrow B$ and $ran(F) = B$ i.e. F is surjective or onto.
    $F: A \rightarrow^1-1 B$ or $F : A \hookrightarrow B$ if $F: A \rightarrow B$ and $\forall x_1, x_2 \in A$ then $(F(x_1) = F(x_2) \implies x_1=x_2)$ i.e. F is injective or one-to-one.
    $F: A \rightarrow^{1-1}_{onto} B$ or $F : A \leftrightarrow B$ as F is surjective and injective

    \underline{Ex}: $F(x) \sin X$
    $F : \mathbb{R} \rightarrow \mathbb{R}$
    $F : $

    \underline{Ex} $\sin([0,\pi / 2]) = [0,1]$ but we need to be more careful because everything is a set.

    \underline{Ex:} $F=\{<0,0>,<1,0>,<2,2>\}$ Recall $2 = \{0,1\}$
    $F(2)=2$
    the range of F applied to the set 2 is $\{F(0), F(1)\}=\{0,0\}=\{0\}$

    \underline{Def:} For a function F and a set A. $F" A$ (F applied to A) is $F" A = ran(F\upharpoonright A)=\{F(x) : x \in A \}$
    \underline{Ex} above F(2)=2, $F"2=\{0\}$


    \underline{\textbf{\emph{Jan 28}}}

    \emph{Note} axioms we have so far don't allow us to build many functions or relations.
    Given set A,B a function f: a $\rightarrow$ b will be a subset of the direct product A X B but axioms so far don't allow us to construct A X B
    even $\{\emptyset\}$ X B can't be produced

    \underline{Ex} Suppos ewe had the set of natural numbers $\mathbb{N}$
    And we want ot define $F(n) = S(n)$ (successor function).
    we need the set $\{<n,S(n)> : n \in \mathbb{N}\}\}$
    Let $\varphi(x,y) : y=S(x)$ i.e. $y = x \cup \{x\}$
    woule like to see the \"range\" of $\varphi$ exists applied to $\mathbb{N}$

    \subsection{Axiom 6 - Replacement Scheme}
    For every formula $\varphi(x,y)$ so that the variable $B$ does not occur in $\varphi$ we have the axiom.
    \[ \forall A ((\forall x \in A \exists! y \varphi(x,y)) \implies (\exists B \forall x \in A \exists y \in B \varphi(x,y)))\]

    \emph{Note} called replacement since we replace each x in A by the unique y with $\varphi(x,y)$
    Comprehension gives us the set range (f/A) where f is the function defined by $\varphi$

    \underline{Def: Direct Product} (Naive)
    \[S \times T = \{x : \exists s \in S \exists t \in T x=<s,t>\}\]
    need to see there is a set containing all such x's
    \underline{Lem:} Given set S and T there is a set Z so that for every $s \in S$ and $t \in T <s,t> \in Z$
    \underline{Cor:} THe direct prodcut exists: $S \times T = \{x \in Z : \exists s \in S \exists t \in T x=<s,t>\}$

    \underline{\emph{Proof of Lem:}}
    \begin{itemize}
        \item Fix $s \in S$.
        \item Let $\varphi(x,y)$ be $y=<s,x>$
        \item Then $\{s\} \times T = \{<s,t> : t \in T\} = \{<w \in B : \exists t \in T w = <s,t>\}$ where B is obtained from replacement applied to the set $A = T$ and $\varphi(x,y)$
        \item Next Let $\varphi(x,y)$ be $y = \{s\} \times T$ and $A=S$
        \item By replacement the set $n = \{\{s\} \times T : s \in S\}$
        \item Then $\bigcup n = \bigcup_{s \in S}\{s\} \times T = S \times T$
    \end{itemize}

    \textbf{Note:} We can now also define functions by formulas.

    \underline{Lemma:} Suppose $\forall x \in A \exists! y \varphi(x,y)$ then there is a function $f$ with $dom(f)=A$ so that for every $x \in A$ $f(x)$ is equal to the unique y so that $\varphi(x,y)$

    \emph{Pf:} Let $B$ be the set given by Replacement.
    Then $f = \{w \in A \times B : \exists x \exists y (w=<x,y>) \wedge \varphi(x,y))\}$

    \underline{Ex:} $A = \mathbb{N} \varphi(x,y) is y=S(x)$ then there is a function $f$ so with $dom(f) = \mathbb{N}$ and $f(n) =S(n)$ for all $n \in \mathbb{N}$

    \textbf{Note:} Multivariable function are relations can be defined as well

    \underline{Ex:} $f(x,y)$ can be viewed as $f: X \times Y \rightarrow Z$

    \textbf{Def:} Given a relation $R$ the \underline{inverse relation} $R^{-1}=\{<y,x> \in ran(R) \times dom(R) : <x,y> \in R\}$

    \underline{Note:} If f is a function, $f^{-1}$ won't necessarily be a function unless f is one-to-one

    \textbf{Def:} Given relations $F$ and $G$ the \underline{composition} $G \circ F$ is given by $G \circ F = \{<x,z> \in dom(F) \times ran(G): \exists y (<x,y> \in F \wedge <y,z> \in G)\}$
    \underline{Note:} If $F$ and $G$ are functions and $ran(F) \subseteq dom(G)$ then $G \circ F$ is the usual composition with $(G \circ F)(x)=G(F(x))$

    \underline{Def:} Suppose $(S,<) and (T,\prec)$ are relations, i.e. < is a relation with $dom(<)$ and $ran(<)$ subset of $S$ and similarly for $(T,\prec)$.
    Then the \underline{lexicographic product} on $S \times T$ is defined by relation $<_{lex}$ given by $<s,t>  <_{lex} <s',t'>$ when $s < s' \vee (s=s' \wedge t \prec t')$

    \underline{Ex:} $S=F=\{0,1,2\}$ < and $\prec$ be the usual ordering $0<1<2, 0 \prec 1 \prec 2$
    $<0,0> <_{lex} <0,1> <_{lex} <0,2> <_{lex} <1,0> \dots <_{lex} <2,2>$

    \emph{Note:} (homework) If < and $\prec$ are strict total orderings then so is $<_{lex}$.

    \underline{Def:} let $R$ be a relation on the $A$  and $S$ a relation on $B$.
    A function $F$ is an \underline{isomorphism} from $(A,R)$ onto $(B,S)$.
    if $F: A \rightarrow B$ so that $\forall x,y \in A$ $x R y \iff F(x) S F(y)$
    we say $(A,R)$ is isomorphic to $(B,S)$, $(A,R,) \cong (B,S)$ if there is a function $F$ which is an isomorphism of $(A,R)$ into $(B,S)$

    \underline{Def:} Let $R$ be an eqiuvalence relation on $A$ (domain and range of R are subsets of A).
    For $x \in A$ let the \underline{equivalence class} of $x$ be $[x] = \{y \in A : y R x\}$
    \emph{Note } $x R y \iff [x] = [y]$

    \underline{Def: } $A/R = \{[x] : x \in A\}$
    i.e. $\{w \in B : \exists x \in A w \ [x]\}$
    where $B$ is the set produced by applying Replacement scheme to the formula $\varphi(x,y) : y=[x]$

    \subsection{Interlude}
    Finite models of parts of Set Theory
    \emph{Note:} can describe a structure on a finite domain where we interpret $\in$ explicitly

    \textbf{Ex:} $X = \{a,b,c\}$ and $a \in b, b \in c$
    i.e. $\in$ is interpreted by relation $E$ where $E = \{ <a,b> , <b,c>\}$

    we can draw a directed graph where nodes are elements of x and arrow represent ordered pairs in $E$.

    Given such a structure, which axioms are satisfied when we interpret $\in$ by $E$

    \underline{Extensionality} two sets are the same exactly if they have the same elements:
    a has no elements, b has a single element a, and c has a single element b.
    satisfies extensionality as they are seperate.

    \underline{Pairing} given two sets then a set exists containing both:
    not satisfied - no set has both a and b as elements.

    \underline{Foundation} no infinite chain of membership:
    satisfied

    \emph{Ex:} $X = \{a,b,c\}$
    $E = \{<a,b>,<b,c>,<c,a>\}$
    (cyclic triangle)

    \underline{Extensionality}: a has element c, b has a, and c has b, satisfying
    \underline{Pairing} fails
    \underline{Foundation} fails

    \emph{Ex} $E =\{<a,c>,<b,c>\}$ (binary tree)
    Foundation and Pairing Fails, and so does Extensionality as a nd b are both empty sets.

    \emph{Ex} $X=\{a\}$ and $E=\{<a,a>\}$ (self-loop)
    Foundation fails, extensionality is trivially satisfied as there is only one set, and it satisfies pairing.

   \emph{\underline{\textbf{Jan 30}}}

    recall let r be a relation
    r is a strict toal order on A if - R is transitive on A, irreflixive on A and satisfies trichotomy
    \textbf{Def}: Let R be a relation. We say $a \in X$ is R-minimal in X if $\neg \exists z ( z \in X \wedge Z R a)$

    \emph{Note:}    a is not necessarily least, i.e. there may be more than one element of x which is R-minimal in X.

    \underline{EX:} $x=\mathbb{N}$ $n R m$ when $n<m$ 0 is the unique \underline{R-minimal} element in x.

    Let $A \subseteq \mathbb{N}$ $A \neq \emptyset$ then A has an R-minimal in A.

    \underline{Ex:} $x = \{ n \in \mathbb{N} : n \geq 2\} n R m$ when m is a proper multiple of n.
    p is R-minimal in X exactly when p is prime.
    infinite R-minimal

    \underline{Ex:} $x=\mathbb{Z}$ $n R m$ when $n < m$ No R-minimal element in X.
    no R-minimal

    \emph{Note:} we can also define R-maximal

    \textbf{Def} A relation R is \underline{well-founded} on X if for every non-empty $A \subseteq X$ there is an $a \in A$ which is R-minimal in A

    \underline{Ex:} $x=\mathbb{N}$ $n R m$ when $n < m$ R is well-founded on X.

    \underline{Ex:} $x = \{n \in \mathbb{N} : n \geq 2\}$ $n R m$ when m is a propers multiple of n.
    R is well-founded on X.
    Let A be a non-empty subset of X\@.
    pick $a_0 \in A$
    if $a_0$ is R-minimal in A we are done (no integer (element in A) is a proper divisor of it)
    else if not there is $a_1 \in A$ so that $a_1$ is a proper divisor of $a_0$
    Note $a_1 < a_0$
    if $a_1$ R-minimal in A we are done \dots etcetera
    This must stop at some point as we can't go below two since the sequence would be decreasing.
    The last term will be R-minimal in A.

    \underline{Ex:} $x \in \mathbb{Z}$ $n R m$ when $n < m$ R is not well-founded
    $A = \mathbb{Z}$ has no R-minimal element

    \underline{Ex:} $x=[0,1]$ $x R y$ when $x < y$.
    R is not well-founded on X.
    $A = (0,1)$ has no R-minimal element because if $a \in A$ $a>0$ so let $b=a/2$ and $b<a$ showing we can find something smaller.

    \textbf{Def:} A relation R is \underline{well-orders} a set A if R totally orders A strictly (transitive and satisfies trichotomy) and R is well-founded on A.

    \underline{Ex:} < well-orders $\mathbb{N}$.

    \underline{Ex:} < does not well-order $\mathbb{Z}$ or $[0,1]$

    \underline{Ex:} proper multiple relation does not well-order $\{n \in \mathbb{N} : n \geq 2\}$

    \textbf{Def}: A \underline{well-ordering} is a pair $(A ; R)$ where A is a set and R is a relation which well-orders R.

    \underline{Ex:} $(\mathbb{N};<)$ is a well-ordering

    \emph{Note:} For a well-ordering $(A;R)$ every non-empty $X \subseteq A $ has an \underline{R-least} element, i.e. there is a unique element of X which is R-minimal in X.
    suppose both x and y are R-minimal in X
    either x=y or $x R y$ meaining y is not R-minmal or vice versa.

    \emph{Note:} If $(S;<)$ and $(T;\prec)$ are both well-orderings then so is their lexicographic product.

    \emph{Note:} If R well-orders A and $B \subseteq A$ is a non-empty subset then R well-orders B.

    \underline{ex:} $(N;<)$

    B = primes

    < is a well-ordering of B
    
    \underline{Ex:} $0=\emptyset$ , $1 = \{\emptyset\}$, $2 = \{\{\emptyset\}\}\}$
    
    $x = \{0,1,2\}$ and $R = \in$
    
    Then $\in$ well-orders X.
    
    \emph{Note:} If X is a finite set then any strict total ordering of X is actually a well-ordering of X.
    
    
    \textbf{\underline{Ordinals}}
    Recall: $0=\emptyset$ and $n+1=S(n)$ where $S(x) = x \cup \{x\}$
    Informally: Natural numbers are 0, S(0), S(S(0))
    
    \textbf{Def}: A set $Z$ is \underline{transitive} if $\forall y \in Z (y \subseteq Z)$
    i.e. $\forall x \forall y ((x \in y \wedge y \in Z) \rightarrow x \in z)$
    
    \underline{Ex:} $z=2=\{\emptyset\{\emptyset\}\}$
    $y=\emptyset$
    $\emptyset \subseteq Z$
    $y=\{\emptyset\}$
    $\{\emptyset\} \subseteq Z$ since $\emptyset \in Z$
    so Z is transitive
    
    \underline{Ex:} $Z=\{\{\emptyset\}\}$ is not transitive
    $y=\{\emptyset\}$
    $y \in Z$
    $x=\emptyset$
    $x \in y$
    but $\emptyset \notin Z$
    
    \emph{Note:} The $\in $ relation behaves transitively with respect to the set Z.
    Z is transitive of $(x \in y \wedge y \in z) \implies x \in z$ for any x and y.

    (natural numbers are transitive)
    
    \textbf{Def} Z is a (von Neumann) \underline{ordinal} if Z is transitive, and Z is well-ordered by $\in$
    
    \underline{Ex:} $2 = \{\emptyset\{\emptyset\}\}$ is an ordinal
    elements: $\emptyset, \{\emptyset\}$
    
    $3=\{0,1,2\}$ is an ordinal
    
    every natural number is an ordinal (see this ?later on?)
    
    \emph{Note: } we usually use Greek letters to denote ordinal.
    
    \textbf{Def} For ordinal $\alpha$ and $\beta$:
    
    $\alpha < \beta$ means $\alpha \in \beta$
    
    $\alpha \leq \beta$ means $\alpha \in \beta$ or $\alpha = \beta$
    
    \textbf{Def} ON or Ord is the class of all ordinals.
    
    \emph{Note:} $ON$ is not a set
    
    $"x \in ON"$ abbreviates \("\)x is transitive and $\in$ well-order x(\")
    
    $" \forall x \in ON \varphi(x)"$ abbreviates $" \forall x (( x is transitive \wedge \in well-order x) \rightarrow \varphi(x))"$
    
    $"x \subseteq ON"$ , $"x \cap ON"$ are similarly abbreviated
    
    \textbf{\underline{Theorem}} $ON$ is well-ordered by $\in$, i.e. 
    \begin{enumerate}
        \item $\forall \alpha, \beta, \gamma \in ON ((\alpha < \beta \wedge \beta < \gamma) \rightarrow (\alpha < \gamma))$
        \item $\forall \alpha \in ON (\alpha \neg < \alpha)$
        \item $\forall \alpha, \beta \in ON ( \alpha < \beta \vee \beta < \alpha \vee \alpha = \beta)$
        \item $\forall z ((z \notin \emptyset \wedge z \subseteq ON) \rightarrow z has an \in - minimal element)$
        \end{enumerate}

    \underline{cor} There is no set which contains all ordinals.
    \underline{pf of cor} suppose X is a set which contains all ordinal
    By comprehension let $Y=\{Z \in X : Z is an ordinal\}$
    Then $Y=ON$ is a set.
    Theorem implies Y is transitive
    (1) If $alpha \in \beta$ and $\beta \in Y$ then $(\alpha < \beta)$ and $\alpha \in Y$ Y is well-ordered by $\in$
    Hence Y is an ordinal so $Y \in Y$
    but since Y is an ordinal by (2) of Theorem $Y \notin Y$ so contradiction

    \emph{Note} This is known as the Burali-Forti Paradox.

    we need some lemmas before proving the theorem

    \underline{Lem 1} ON is a transitive class, i.e. if $\alpha \in ON$ and $z \in \alpha$ then $z \in ON$

    \underline{pf} let $\alpha$ be an ordinal.
    Then $\alpha$ is transitive.
    Since $z \in \alpha$ then $z \subseteq \alpha$.
    $\in$ well-orders $\alpha$, so $\in$ well-orders $z$.
    To see $z$ is transitive: suppose $x \in y$ and $y \in z$
    so $y \in \alpha$ so $y \subseteq \alpha$ (since $\alpha$ is transitive and $z \subseteq \alpha$)
    so $x \in \alpha$
    hence $x,y,z \in \alpha$ and $x \in y$ and $y \in z$ so $x \in z$
    since $\in$ well-orders $\alpha$ and hence $\in$ is a transitive relation in $\alpha$
    which is what we wanted.
    Hence $z$ is transitive and well-ordered by $\in$ , so $z$ is an ordinal

    \emph{ordinals} \underline{feb 4}

    Recall: A set z is a transitive if whenever $y \in z$ then $y \subseteq$, i.e. if $x \in y$ and $y \in z$ then $x \in z$.
    A set $\alpha$ is an \underline{ordinal} if $\alpha$ is transitive and $\in$ well-orders $\alpha$.
    $\alpha < \beta \leftrightarrow \alpha \in \beta$ and $alpha \leq \beta \leftrightarrow \alpha < \beta \vee \alpha = \beta$

    \underline{Theorem} $ON$ is well-ordered by $\in$

    \underline{Lem 1} $ON$ is a transitive class, i.e. $\alpha \in ON$ and $z \in \alpha$ then $z \in ON$ where $ON$ is the class of ordinals. $\alpha$ is an ordinal (is what the in means)

    \underline{Lem 2} For $\alpha , \beta \in ON$ $\alpha \leq \beta $ iff $\alpha \subseteq \beta$.
    \begin{proof}

    $\alpha \leq \beta$ means $\alpha \in \beta \vee \alpha = \beta$
    if $\alpha = \beta$ immediately $\alpha \subseteq \beta$
    if $\alpha \in \beta$ then $\alpha \subseteq \beta$ since $\beta$ is transitive.

    suppose $\alpha \subseteq \beta$
    if $\alpha = \beta$ we are done, so suppose not.
    let $X = \beta \\ \alpha \neq \emptyset$ , $X \subseteq \beta$
    $\beta$ is wel-ordered by $\in$,
    X has an $\in$-minimal element $\xi$
    show $\xi = \alpha$
    For $\mu \in \xi$ we have $\mu \in \beta$ and $\mu \notin X$ so $\mu \in \alpha$
    Hence $\xi \leq \alpha$.
    Suppose $\xi \neq \alpha$.
    Then there is a$\mu \in \alpha \setminus \xi$
    $\mu$ and $\xi$ both in $\beta$.
    $\beta$ is totally ordered by $\in$ so one of $\mu \in \xi$, $\xi \in \mu$, or $\xi = \mu$.
    $\mu \neq \xi$ by choice of $\mu$
    since $\mu \in \xi$ and $\xi \notin \alpha$ $(\xi \in X = \beta \setminus \alpha)$ so $\mu = \xi$
    if $\xi \in \mu$ then $\xi \in \alpha$ (since $\mu \in \alpha$) but $\xi \notin \alpha$, hence $\xi \notin \mu$
    Hence $\xi = \alpha$, so $\alpha \in \beta$.
    \end{proof}

    \underline{Lem 3} If $\alpha, \beta \in ON$ then $\alpha \cap \beta \in ON$.
    (we will see $\alpha \cap \beta = \min(\alpha, \beta)$)

    \begin{proof}
        $\alpha \cap \beta \subseteq \alpha$ so $\alpha \cap \beta$ is well-ordered by $\in$.
        $\alpha \cap \beta$ is transitive since if $x \in \alpha \cap \beta$.
        then $x \in \alpha$ so $x \subseteq \alpha$, and $x \in \beta$ so $x \subseteq \beta$.
        Hence $x \subseteq \alpha \cap \beta$.
        So $\alpha \cap \beta$ is an ordinal.
    \end{proof}

Proof of Theorem:
\begin{proof}
    \begin{enumerate}
        \item $\forall \alpha, \beta, \gamma \in ON$ and $\alpha < \beta \wedge \beta < \gamma$ then $\alpha < \gamma$
        \begin{itemize}
            \item $\alpha$ is a transitive set.
            \item $\beta < \gamma$ means $\beta \in \gamma$.
            \item $\alpha < \beta$ means $\alpha \in \beta$.
            \item So $\alpha \in \gamma$ i.e. $\alpha < \gamma$.
        \end{itemize}
        \item $\forall \alpha \in ON$ $\alpha \notin \alpha$
        \begin{itemize}
            \item $\alpha$ is well-ordered by $\in$.
            \item If $\alpha \in \alpha$ then $\forall x \in \alpha$ ($x \notin x$).
            \item Hence $\alpha \notin \alpha$.
        \end{itemize}
        \item $\forall \alpha, \beta \in ON$ ($\alpha \in \beta \vee \beta \in \alpha \vee \alpha = \beta$)
        \begin{itemize}
            \item Let $\delta = \alpha \cap \beta$.
            \item $\delta \in ON$ by Lemma 3.
            \item $\delta \subseteq \alpha$ so $\delta \leq \alpha$ by Lemma 2.
            \item So either $\delta = \alpha$ or $\delta \in \alpha$.
            \item If $\delta = \alpha$ then $\alpha \subseteq \beta$ so either $\alpha = \beta$ or $\alpha \in \beta$ and we are done.
            \item So assume $\delta \in \alpha$
            \item $\delta \subseteq \beta$ so either $\delta = \beta$ or $\delta \in \beta$.
            \item if $\delta = \beta$ then $\beta \subseteq \alpha$ so we are done.
            \item so assume $\delta \in \beta$
            \item so $\delta \in \alpha \cap \beta = \delta$ but $\delta \notin \delta$, contradiction
            \item so either $\delta = \alpha$ or $\delta = \beta$ and we are done
        \end{itemize}
        \item $\forall z$ ($z \neq \emptyset \wedge z \subseteq ON \rightarrow$ $z$ has an $\in$-least element.
        \begin{itemize}
            \item Let $\alpha \in z$ be arbitrary.
            \item If $\alpha$ is $\in$-minimal in $z$ we are done.
            \item Suppose not.
            \item Consider $\alpha \cap z = \{\beta \in z : \beta < \alpha\}$
            \item $\alpha \cap z \subseteq \alpha$
            \item if $\alpha \cap z = \emptyset$ we are done as $\alpha = \emptyset$ would be $\in$-least in $z$
            \item if $\alpha \cap z \neq \emptyset$ then there is an $\in$-minimal element of $\alpha \cap z$, call this $\gamma$
            \item $\gamma \in \alpha \cap z$ and for any $\beta \in \alpha (\beta \notin \alpha \cap z)$
            \item if $\beta \in \gamma$ then $\beta \in \alpha$ hence $\beta \notin z$
            \item Hence $\gamma$ is $\in$-minimal element of $z$ which is what we wanted
        \end{itemize}
    \end{enumerate}
\end{proof}

    \underline{Ex}
    \begin{itemize}
        \item $0=\emptyset$ is the least ordinal
        \item $1 = \{\emptyset\}$ is the next biggest ordinal (no ordinal $\alpha$ with $0 < \alpha$ and $\alpha < 1$)
    \end{itemize}

    \emph{Note} If $\alpha \in ON$ then $\alpha = \{\gamma \in ON : \gamma < \alpha\}$

    Recall $S(x) = x \cup \{x\}$

    \underline{Lem} If $\alpha \in ON$ then $S(x) \in ON$.
    \begin{proof}
        If $\beta \in S(\alpha)$ then either $\beta \in \alpha$ or $\beta = \alpha$.
        In both cases $\beta \subseteq \alpha$ hence $\beta \subseteq S(\alpha)$.
        So $S(x)$ is transitive.
        If $z \subseteq S(x)$ and $z \neq \emptyset$.
        If $z = \{\alpha\}$ then $\alpha$ is an $\in$-minimal element of $z$ (since $\alpha \notin \alpha$)
        Otherwise, $z \cap \alpha \neq \emptyset$ has an $\in$-minimal element since $\alpha \in ON$ which will be an $\in$-minimal element of $z$.
    \end{proof}

    \underline{Lemma} For $\alpha \in ON$, $S(\alpha)$ is the immediate succesor of $\alpha$, i.e. $\alpha < S(x)$ and there is no ordinal $\beta$ with $\alpha < \beta$ and $\beta < S(\alpha)$
    \begin{proof}
        $\alpha \in S(\alpha) = \alpha \cup \{\alpha\}$ so $\alpha < S(\alpha)$
        If $\beta < S(alpha)$ then $\beta \in \alpha \cup \{alpha\}$
        So either $\beta \in \alpha$ or $\beta = \alpha$
        i.e. either $\beta < \alpha$ or $\beta = \alpha$ so can't have $\alpha < \beta$
    \end{proof}

    \textbf{Def} An ordinal $\beta$ is a \underline{successor ordinal} if $\beta = S(\alpha)$ for some $\alpha$.
    Otherwise $\beta$ is a \underline{limit ordinal} if $\beta \neq 0$ and $\beta$ is not a successor ordinal.
    A \underline{finite ordinal} or \underline{natural number} is if every $\alpha \subseteq \beta$ is either 0 or a successor ordinal.

    \emph{Note:} Axioms so far only produce finite ordinal e.g. 0,1,2,....

    \emph{Note:} $\beta \neq 0$ is a limit ordinal $\iff$ $\forall \alpha < \beta$ $S(\alpha) < \beta$

    \subsection{Axiom 7: Axiom of Infinity}
    $\exists X (\emptyset \in X \wedge \forall y (y \in X \rightarrow S(y) \in X))$

    \emph{Note: } We want to see that this X is an infinite set containing every finite ordinal.

    \underline{Lemma: } If $n$ is a natural number then $S(n)$ is a natural number.
    \begin{proof}
        $S(n)$ is an ordinal from earlier.
        If $\alpha \leq S(n)$ then
        If $\alpha = S(n)$ then $\alpha$ is a successor (done)
        Otherwise if $\alpha < S(n)$ then $\alpha \in n \cup \{n\}$
        So $\alpha \leq n$ so either $\alpha = 0$ or $\alpha$ is a successor.
    \end{proof}

    \underline{Theorem: } \underline{Principle of Ordinary Induction}
    For any set $X$, if $\emptyset \in X$ and $\forall y \in X (S(y) \in X)$ then $X$ contains all natural numbers.

    \begin{proof}
        Suppose the above is true.
        Suppose $n$ is a natural number with $n \notin X$.
        Let $Y = S(n) \setminus X \neq \emptyset$ since $n \in Y$.
        Y is a set of natural numbers.
        Y has an $\in$-minimal element, call it $k$.
        $k \leq n$
        Either $k=0$ or $k = S(j)$ for some $j$.
        $k \neq 0$ since $k \notin X$ and $0 \in X$
        $k \neq S(j)$ since $j \notin Y$ ($j \in k$ and $k \in$-minimal in $Y$)
        if $k = S(j)$ then $j \in S(n)$ so $j \in X$ so $S(j) = k \in X$, contradiction
        Hence no natural number $n \notin X$
    \end{proof}

    \textbf{Def} Let $\omega = \{n \in X: n is a natural number\}$ where $X$ is the set from the Axiom of Infinity.
    So $\omega$ is the set of natural numbers sometimes denoted $\mathbb{N}$

    \emph{Note:} we will see that $\omega$ is an ordinal.
    $\omega$ is the least infinite ordinal.
    $\omega$ is the least limit ordinal.

    \underline{\emph{Feb 6.}}
    Recall $\omega = \{n \in X : n is a natural number\}$ where $X$ is the set from the Axiom of Infinity so that $\emptyset \notin X \wedge \forall Y (Y \in X \rightarrow S(X) \in X)$

    \underline{Lemma: } Suppose $X$ is a set of ordinals which is an initial segment of $ON$. i.e. $\forall \beta \in X \forall \alpha \in \beta (\alpha \in X)$ then $X \in  ON$

    \begin{proof}
        $X$ is well-ordered by $\in$ since any set of ordinals is well-ordered by $\in$

        X is transitive since if $\alpha \in \beta$ and $\beta \in X$ then $\alpha < \beta$ so $\alpha \in X$

        Hence $X$ is an ordinal.

    \end{proof}

    \underline{Lemma: } $\omega$ is the least-limit ordinal.
    \begin{proof}
        $\omega$ is an ordinal since if $n \in \omega$ then $n$ is a natural number and if $m < n$ then m is a natural number, hence $m \in \omega$

        so $\omega$ is an initial segment of $ON$

        $\omega \neq \emptyset$

        suppose $\omega = S(y)$ for some $y$ then $y \in S(y) = \omega$ so $y$ is a natural number so $S(y) = \omega$ would be a natural number, so $\omega \in \omega$ contradicts that $\alpha \notin \alpha$ for any ordinal $\alpha$

        hence $\omega$ is a limit ordinal.

        every $\alpha < \omega$ is a natural number so either 0 or $S(n)$ some n

        hence $\omega$ is the least limit ordinal

    \end{proof}

    \emph{Note: } Now we have $0,1,2,\dots,\omega,S(\omega), S(S(\omega)),\dots$.
    Like the think of $S(\omega)$ as $\omega + 1$, etcetera.
    Need to define ordinal arithmetic $\alpha + \beta$, $\alpha \dot \beta$, $\alpha^{\beta}$.
    we will define these in terms of certain well-orderings we define.
    See that every well-ordering corresponds to a unique ordinal.

    \underline{Ex} subsets of $\mathbb{R}$ with the usual ordering <.
    Let $A = \{1-\frac{1}{n} : n \geq 1\}$
    $(A ; <) \cong (\omega,\in)$
    Let $B = \{ 1-\frac{1}{n}: n \geq 1\}\cup \{1\}$
    $(B;<) \cong (S(\omega); \in)$ where $S(\omega) = \omega + 1$
    Let $C = \{1-\frac{1}{n}:n\geq1\}\cup\{2-1/n:n\geq1\}$
    $(C;<) \cong (\omega + \omega;\in)$


    \underline{Lemma} If $\alpha$ and $\beta$ are ordinals and $f : \alpha \leftrightarrow \beta$ is an isomorphism from $(\alpha,\in)$ onto $(\beta,\in)$ then $\alpha = \beta$ and $f$ is the identity function.

    \begin{proof}
        Fix $\xi \in \alpha$, sp $f(\xi) \in \beta$.
        If $\zeta < \xi$ then $f(\zeta) < f(\xi)$ and vice versa.
        Then $\{\nu \in \beta : \nu < f(\xi)\}=\{f(\mu) : \mu \in \alpha \wedge \mu < \xi\}$.
        Hence $f(\xi) = f``\xi = \{f(\gamma) : \gamma \in \xi\}$
        Claim that $f(\xi) = \xi$ for all $\xi \in \alpha$.
        Let $X = \{\xi \in \alpha : f(\xi) \neq \xi\}$
        If $X = \emptyset$ we are done so suppose not.
        Let $X \neq \emptyset$ then $X$ is a set of ordinals so there is a least $\xi \in X$.
        If $\mu < \xi$ then $f(\mu)=\mu$.
        So $f(\xi)=\{f(\mu) : \mu < \xi\} = \{\mu : \mu < \xi\} = \xi$ contradicting that $f(\xi) \neq \xi$.
        Hence $f(\xi) = \xi $ for every $\xi \in \alpha$ i.e. $f$ is the identity map, and $\beta = \alpha$.
    \end{proof}

    \emph{Note: } There can be bijections between unequal ordinals, but they won't be order preserving.

    \underline{Ex: } $\omega = 0,1,2,\dots$ and $\omega + 1 = 0,1,2,\dots,\omega$.
    Pairing $\omega$ with 0 and the rest with their successors, so we have a bijection

    \emph{Note: } In the above proof we used the \underline{principle of transfinite induction}.
    Every non-empty set of ordinals has a least element.
    So fi $\varphi(x)$ is a formula and for all $\alpha \in ON$
    $(\forall \beta < \alpha \varphi(\beta)) \rightarrow \varphi(\alpha)$
    Then $\varphi(\alpha)$ holds for all ordinals $\alpha$

    \emph{Note: } Often split into cases for $\alpha$.
    show $\varphi(0)$.
    show $\varphi(\beta) \rightarrow \varphi(S(\beta))$.
    for a limit ordinal $\gamma$ show $(\forall \beta < \gamma \varphi(\beta)) \rightarrow \varphi(\gamma)$
    (orders are rigid we cant have an order isomorphism? maybe?)

    \underline{\textbf{Theorem}}: If R is a well-ordering of A then there is a unique ordinal $\alpha$ so that $(A;R) \cong (\alpha;\in)$
    (ordinal gives a representation of all orderings up to isomorphism)

    \begin{proof}
        $\alpha$ is unique from the previous lemma since if $f : (A;R) \leftrightarrow (\alpha;\in)$ and $g : (A;R) \leftrightarrow (\beta;\in)$.
        Then $g \circ f^{-1} : (\alpha;\in) \leftrightarrow (\beta;\in)$.
        So $g \circ f^{-1} = identity$ i.e. $f=g$ and $\alpha = \beta$
        For $a \in A$ let $a \downarrow = \{x\in A : x R a\}$.
        Then $a \downarrow$ is also well-ordered by R.
        Call $a \in A$ \underline{good} if $(a \downarrow;R) \cong (\xi; \in)$ for some ordinal $\xi$.
        Let $G = \{a \in A : a is good\}$.
        Define $f : G \rightarrow ON$ by setting $f(a) = $ the unique $\xi \in ON$ so that $(a\downarrow;R)\cong(\xi;\in)$
        Also for $a \in G$ let $h_a$ be the unique isomorphism $h_a=(a\downarrow;R)\cong(f(a);\in)$ ($f(a)=\xi$)
        This is justified by replacement
        If $a\in G$ and $c R a$ (i.e. $c \in a\downarrow$) then $c \in G$ since $c \downarrow \subseteq a\downarrow$
        $h_c = h_a \upharpoonright (c\downarrow): (c\downarrow;R)\cong(h_a(c);\in)$
        ($\upharpoonright$ is 'restricted')
        (Lecture drawing 54 mins into class)
        So $f(c) = h_a(c) < f(a)$
        Hence $f : G \rightarrow ON$ is order-preserving, i.e. $cRa \leftrightarrow f(c) < f(a)$.
        $ran(f)$ is an initial segment of $ON$ since if $\xi \in  ran(f)$ there is some $a \in G$ with $f(a) = \xi$
        So if $\mu < \xi`$ there is $c \in a\downarrow$ with $h_a(c) = \mu$
        But then $c \in G$ and $f(c) = h_a(c) = \mu$ so $\mu \in ran(f)$
        Hence $ran(f)$ is an ordinal $\alpha$ and f is an isomorphism from $(G;R)$ onto $(\alpha;\in)$.
        Want to show $G=A$.
        Suppose not, then let $e$ be the R-least elements of $A \setminus G$
        Then $e \downarrow = G$ since G is an R-initial segment of A.
        Hence $(e \downarrow;R) = (G;R) \cong (\alpha;\in)$.
        Hence $e \in G$, contradiction.
        Hence $G=A$ so $(A;R)\cong(\alpha;\in)$
    \end{proof}

    \textbf{Def: } For a well-ordering $(A;R)$ the \underline{order type} of $(A;R)$ , $type(A;R)$ is the unique ordinal $\alpha$ such that (A;R)$\cong$($\alpha;\in$)

    Recall: For (A;R) and (B;s) the \underline{lexicographic product} is the ordering of $A \times B$ given by $(a_1,b_1) <_{lex}(a_2,b_2) \leftrightarrow a_1 R a_2 \vee (a_1 = a_2 \wedge b_1 s b_2)$.

    \textbf{Def: } \underline{Ordinal arithmetic}
    For $\alpha,\beta \in ON$
    $\alpha \dot \beta = type(\beta \times \alpha, <_{lex})$
    $\alpha + \beta = type(\{0\}\times \alpha \cup \{1\} \times \beta, <_{lex})$

    \underline{Ex: } $\omega + 1$: $\{0\} \times \omega \cup \{1\}\times 1$ (see drawing at feb 6. 72 mins)

    Recall \underline{Ordinal Arithmetic}
    - $\alpha \dot \beta = type(\beta \times \alpha, <_{lex})$
    - $\alpha + \beta = type(\set{0} \times X \cup..??)$

    \underline{Ex: }
    $\omega \dot \omega = \langle 0,0 \rangle < \langle 1 , 0 \rangle < ....$

    multiplication and addition not commutative

    $\omega \dot 2 = \omega + \omega$
    $2 \dot \omega = 2 + 2 + ... = \omega$

    \emph{Note: } We can give recursive definitions.
    \textbf{Def: }
    - $\alpha + 0 = \alpha$
    - $\alpha + S(\beta) = S(\alpha + \beta)$
    - $\alpha + \lambda = Sup_{\beta < x} \alpha + \beta, \lambda limit$

    \textbf{Def: }
    - $\alpha \dot 0 = 0$
    - $\alpha \dot S(\beta) = \alpha \dot \beta + \alpha$
    - $\alpha \dot 2 = Sup(\beta < \lambda) (\alpha \dot \beta)$

    \textbf{Def: } \underline{Ordinal Exponentiation}
    - $\alpha^0 = 1$
    - $\alpha^{S(\beta)} = \alpha^\beta \dot \alpha$
    - $\alpha^\lambda = Sup(\beta < \lambda) \alpha^\beta$

    \underline{Ex: }
    - $\alpha^1 = 1 \dot \alpha = \alpha$
    - $\omega^2 = \omega \dot \omega$

    \subsection{Properties of Ordinal Arithmetic}
    \underline{Lemma}: addition and multiplicaiton are associative (distribute only on the left)
    - $(\alpha + \beta) + \gamma = \alpha + (\beta + \gamma)$
    - $(\alpha \dot \beta) \dot \gamma = \alpha \dot (\beta \dot \gamma)$
    - $\alpha \dot (\beta + \gamma) = \alpha \dot \beta + \alpha \dot \gamma$
    - $\alpha + \beta = \alpha + \gamma \rightarrow \beta = \gamma$
    - $\alpha \leq \beta \rightarrow \exists! \gamma (\alpha + \gamma = \beta)$
    - $\alpha > 0 \rightarrow \exists! \gamma, \delta \beta= \alpha \dot \gamma + \delta \wedge \delta < \alpha)$
    - $\alpha^{\beta + \alpha}$.... etc. seee feb 11. 15 mins

    But:
    $\omega + 1 \neq 1 + \omega = \omega$
    $\omega \dot 2 = \omega + \omega \neq 2 \dot \omega = \omega$
    $(1+1) \dot \omega = 2 \dot \omega = \omega \neq 1 \dot \omega + 1 \dot \omega = \omega + \omega$
    $1 + \omega = \omega = 2 + \omega$ but $1 \neq 2$

    \emph{Note:} We can prove these i n a couple ways.
    Transfinite induction (later).
    Comparing Order Types.

    \underline{Ex: } Show multiplicaiton is associative:
    $ (\alpha \dot \beta) \dot \gamma = \alpha \dot (\beta \dot \gamma)$
    \begin{proof}
        Find order-preserving bijection between two orderings.
        $\alpha \dot \beta = type(\beta \times \alpha, <_{lex})$
        $(\alpha \dot \beta) \dot \gamma = type(\gamma \times (\beta \times \alpha), <_{lex})$
        $a_1, a_2 \in \alpha b_1, b_2 \in \beta c_1,c_2 \in \gamma$
        $\langle c_1 \langle b_1, a_1 \rangle \rangle R < \langle c_2, \langle b_2,a_2 \rangle \rangle$
        $\iff c_1 < c_2 \vee ( c_1 = c_2 \wedge \langle \langle b_1, a_1 \rangle <_{lex} \langle b_2, a_2 \rangle ))$
        $iff c_1 < c_2 \vee ( c_1 = c_2 \wedge (b_1 < b_2 \vee (b_1 = b_2 \wedge a_1 < a_2)))$
        R corresponds to $(\alpha \dot \beta) \dot \gamma$
        Now $\alpha \dot (\beta \dot \gamma) = type ((\beta \times \gamma, <_{lex})\times \alpha, <_{lex})$
        $\langle \langle c_1,b_1\rangle,a_1\rangle S \langle\langle c_2,b_2,\rangle a_2\rangle$
        $iff (\langle c_1,b_1\rangle <_{lex})$ \dots etcetera

        (*) $\langle c_1,\langle b_1, a_1 \rangle \rangle R \langle c_2, \langle b_2, a_2 \rangle \rangle$
        $\iff \langle \langle c_1, b_1 \rangle , a_1 \rangle S \langle \langle c_2, b_2 \rangle , a_2 \rangle$

        Let $f : \gamma \times (\beta \times \alpha) \rightarrow (\gamma \times \beta) \times \alpha$
        be defined by:
        $f(\langle c, \langle b, a \rangle \rangle) = \langle \langle c, b \rangle , a \rangle$
        f is a bijection.
        f is order-preserving i.e. * from above
        shows $\langle c_1, \langle b_1, a_1 \rangle \rangle R \langle c_2, \langle b_2, a_2 \rangle \rangle$
        $\iff f(\langle c_1, \langle b_1, a_1 \rangle \rangle) S f(\langle c_2, \langle b_2, a_2 \rangle \rangle )$
    \end{proof}

    \subsection{Induction and Recursion on the Ordinals}
    \underline{induction:} used to prove statements.
    \underline{recursion:} used to define functions.

    \underline{Ex: } (natural numbers).
    Prove with induction that $\sum{i=1}{n} i = \frac{n(n+1)}{2}$
    $n=0 0=0$
    assume $\sum{i=1}{n} = \frac{n(n+1)}{2}$
    show $\sum{i=1}{n+1} = \frac{(n+1)(n+2)}{2}$

    define $n!$ by recursions
    $0! = 1$
    $(n+1)! = (n+1) \dot n!$

    \underline{\textbf{Theorem} Transfinite Induction on ON}
    For each formula $\varphi (x)$ if $\varphi(\alpha)$ holds for some ordinal $\alpha$, then there is a least ordinal $\xi$ so that $\varphi(\xi)$ holds

    \emph{Note: } We can replace `` holds `` by `` fails ``
    \begin{proof}
        Fix $\alpha$ sp that $\varphi(\alpha)$ holds
        Let $x = \set{\xi \in \alpha : \varphi(\xi)}$
        If $ x = \emptyset$ then $\alpha$ is the least ordinal with $\varphi(\alpha)$
        If $x \neq \emptyset$ then $x$ has an $\in$-least elements $\xi$, which is the least ordinal with $\varphi(\xi)$
    \end{proof}

    \underline{Corr} Suppose $\varphi(\xi)$ satisfies:
    (*) $\forall a \in ON (( \forall \beta < \alpha \varphi(\beta)) \rightarrow \varphi(\alpha))$
    Then $\varphi(\alpha)$ holds for every $\alpha \in ON$

    \begin{proof}
        If $\varphi(\alpha)$ fails for some $\alpha$ then there is a least $\alpha$ so $\varphi(\alpha)$ fails.
        Then $\forall \beta < \varphi(\alpha)$
        So by (*) we have $\varphi(\alpha)$ holds, a contradiction
    \end{proof}

    \emph{Note: } Odten we split (*) into cases
    $\alpha = 0$
    $\alpha = S(\beta)$ show if $\varphi(\alpha)$ holds then $\varphi(S(\beta))$ holds.
    $\alpha = \lambda$ is a limite: Show if $\varphi(\beta)$ holds for all $\beta < \alpha$ then $\varphi(\alpha)$ holds.

    \underline{Ex: } Show for every $\alpha \in ON$
    There are unique $\lambda$ and $n$ so that either $\lambda = 0$ or $\lambda$ is a limit ordinal.
    n is a natural number, and $\alpha = \lambda + n$
    prove by transfinite induction
    \begin{proof}
        base case:
        $\alpha = 0 = 0 + 0$
        So $\lambda = 0$ and $n = 0$
        case:
        $\alpha = S(\beta)$ Assume $\beta = \lambda + n$ some unique $\lambda$ and $n$
        $S(\beta) = \lambda + n + 1 = \lambda + S(n)$, $S(n)$ is a natural number.
        case:
        $\alpha = \lambda$ is a limit ordinal
        $\alpha = \lambda + 0$ 0 is a natural number.
    \end{proof}

    \underline{Transfinite Recursion on ON}
    We want to define funcitons on ON.
    we mean to find a formula $\varphi(x,y)$ so that $\forall x \exists! y \varphi(x,y)$.
    We can define $G(x) = !y \varphi(x,y)$ (not literallt a function)
    For any ordinal $\xi$ $G \upharpoonright \xi$ is a function (G restricted to xi)

    \underline{\textbf{Theorem} Primitive Recursion on ON}
    Suppose $\varphi(x,y)$ is a formula so that $\forall x \exists! y \varphi(x,y)$.
    Let $G(s) = $ the unique $y$ with $\varphi(s,y)$.
    We can define a formula $\psi(x,y)$ so that
    (1) $\forall x \exists! y \psi(x,y)$
    (2) $\forall \xi \in ON F(\xi) = G(F \upharpoonright \xi)$ where $F(\xi)$ is the unique $y$ with $\psi(\xi, y)$

    \underline{Ex: } Fix an ordinal $\alpha \in ON$
    Define $E_\alpha(\beta) = \alpha^\beta$
    Let $\varphi(x,y)$ be :
    ($x$ is not a function with $dom(x) \in ON \wedge y = 0 ) \vee$
    ($x$ is a function with $dom(x) = \xi \in ON \wedge$
    $((\xi = 0 \wedge y=1)) \vee (\exists \beta (\xi = S(\beta) \wedge y = x(\beta) \dot \alpha) \vee (\xi is a limit ordinal \wedge y = \set{x(\beta) : \beta < \xi})))$ (?)

    Then there is a formula $\psi(x,y)$ so that setting $F(\xi) = G(F \upharpoonright \xi)$ where $G(x) = !y$ with $\varphi(x,y)$ (x is a function)
    Then $F(\xi) = \alpha^\xi$ as we defined earlier $E_\alpha(\xi)$

    \emph{Note: } $F \upharpoonright \xi$ will be a function and equals $\set{\langle \beta ,F(\beta) \rangle : \beta < \xi}$ a function with domain $\xi$


    \section{Feb 13}
    \underline{\textbf{Theorom:} Primitive Recursion on ON}
    (See slides)
    \begin{proof}
        For $\delta \in ON$ let $App(\delta,h)$ be the formula that says that $h$ is a $\delta$-approximation to $F$, i.e.
        $h$ is a funciton
        $dom(h)=\delta$
        $h(\xi)=G(h \upharpoonright \xi)$ for all $\xi < \delta$
        where $G(s)$ is the unique $y$ with $\varphi(s,y)$

        Let $\psi (x,y)$ be: $(x \notin ON \wedge y = \emptyset) \vee (x \in ON \wedge \exists \delta > X \exists h(App(\delta,h) \wedge h(x) = y))$.
        (We need to see that there are enough of the approximiation and that if we have diff approxes that go up to different things, we need ot see that hey agree?
        that for every x there is a unique y so that psi of x,y.)

        Need to see that this $\psi$ works.
        Verify $\forall x \exists! y \psi(x,y)$.

        \underline(Show two claims:)
        \underline{Claim(unique)}: $(\delta \leq \delta` \wedge App(\delta,h) \wedge App(\delta`,h`)) \rightarrow h=h` \upharpoonright \delta$
        \underline{Claim(Exist)}: $\forall \delta \exists h App(\delta,h)$

        Granting these two claims:
        We have $\forall x \exists! y \psi(x,y)$.
        Given $\xi \in ON$ let $\delta > \xi$ and let $h$ be the unique function with $App(\delta,h)$.
        Then $F \upharpoonright \delta = h$ so $F \upharpoonright \xi = h \upharpoonright \xi$.
        So $F(\xi) = h(\xi) = G(h \upharpoonright \xi) = G(F \upharpoonright \xi)$.
        as wanted so we need to verify these two claims.

        \underline{Proof of claim(u)} using transfinite induction
        Assuming $\delta \leq \delta` \wedge App(\delta,h) \wedge App(\delta`,h`)$ that $h(\xi)=h`(\xi) \forall \xi < \delta$
        (doing this by induction on $\xi$)
        Suppose not (there cant be a least failure)
        Let $\xi$ be the least element of $\set{\alpha < \delta : h(a) \neq h`(\alpha)}$ (want this to be empty)
        Then $h \upharpoonright \xi = h` \upharpoonright \xi$.
        Hence $h(\xi) = G(h \upharpoonright \xi) = G(h` \upharpoonright \xi) = h`(\xi)$
        since $App(\delta,h)$ and $App(\delta`,h`)$
        contradiction $\xi$ satisfied $h(\xi) \neq h`(\xi)$

        \underline{Proof of claim(e)} (also by transfinit induction)
        Show $\forall \delta \exists h App(\delta,h)$
        Suppose not and let $\delta \in ON$ be least so that $\neg \exists h App(\delta,h)$

        Three Cases:
        $\delta = 0$ is impossible since $App(0,\emptyset)$

        $\delta = S(\beta)$ for some $\beta$.
        There is $g_{\beta}$ so that $App(\beta,g_{\beta})$.
        Let $f = g_{\beta} \cup \set{\langle \beta, G(g_{\beta})}$
        Notice $g_{\beta} = f \upharpoonright \beta$ so $f(\beta) = G(g_{\beta}) = G(f \upharpoonright \beta)$.
        So $f(\xi) = G(f \upharpoonright \xi)$ for all $\xi < S(\beta) = \delta$
        Hence $App(\delta,f)$ holds, contradicting that $\neg \exists h App(\delta,h)$

        $\delta$ is a limit ordinal.
        For all $\beta < \delta$ there is a unique $g_{\beta}$ with $App(\beta,g_{\beta})$ since claim(u) holds.
        Let $f = \bigcup \set{g_{\beta} : \beta < \delta}$
        Because of claim(u), $f$ is a function. (unique valued determined)
        $dom(f) = \delta$
        Since for every $\xi < \delta$ there is a $\beta$ with $\xi < \beta < \delta$ ($\delta$ is a limit ordinal)
        $f(\xi) = g_{\beta}(\xi) = G(g_{\beta} \wedge \xi) = G(f \upharpoonright \xi)$.
        Hence $App(\delta,f)$ contradiction.

    \end{proof}


    \textbf{Def:} For $\alpha \in ON$ an \underline{$\alpha$-sequence} is a fucntion with domain $\alpha$.
    Write $S_{\beta}$ for $S(\beta)$ for $\beta < \alpha$ and $S$ is an $\alpha$-sequence.

    \underline{Ex:} $\alpha = \omega$ $S_0,S_1,S_2, \dots$

    \textbf{Def:} Let $\bigcup^0 x = x$
    $\bigcup^{n+1} x = \bigcup \bigcup^n x$ for $n \in  \omega$
    The \underline{transitive closure} of $x$, $trcl(x)$ or $TC(x)$, as
    $TC(x) = \bigcup \set{\bigcup^n x : n \in \omega}$

    \underline{Ex:} $ x = \set{\set{\emptyset,}\set{\set{\emptyset}}}$
    $\bigcup^0 x = x$
    $\bigcup^1 x = \set{\emptyset,\set{\set{\emptyset}}}$
    $\bigcup^2 x = \emptyset \cup \set{\set{\emptyset}} = \set{\set{\emptyset}}$
    $\bigcup^3 x = \set{\emptyset}$
    $\bigcup^4 x = \emptyset$
    $\bigcup^n x = \emptyset$ for all $n > 4$
    $TC(x) = \set{\emptyset , \set{\set{\emptyset}},\set{\emptyset}}$

    \underline{Lemma:}
    (1) $x \subseteq TC(x)$
    (2) $TC(x)$ is a transitive set
    (3) $TC(x)$ is the smallest transitive set $t$ with $x \leq t$
    (4) If $y \in x$ then $TC(y) \subseteq TC(x)$

    \begin{proof}
    (1) True since $x = \bigcup^0 x \subseteq TC(x)$
    (2) If $y \in TC(x)$ then there is $n \in \omega$ with $y \in \bigcup^n x$ so $y \subseteq \bigcup \bigcup^n x$
        So $y \subseteq \bigcup ^n+1 x \subseteq TC(x)$.
        So $TC(X)$ is transitive.
    (3) Suppose $x \subseteq t$ and $t$ is a transitive set.
        Suppose for contradiction $TC(x) \nsubseteq t$.
        Let $n$ be least so that $\bigcup^n x \nsubseteq t$.
        $n \neq 0$ since $x \subseteq t$ $x=\bigcup^0 x$.
        So $n=k+1$ for some $k$.
        if $y \in \bigcup^n x$ then $y \in \bigcup^{k+1} x = \bigcup \bigcup^k x$.
        So there is $z \in \bigcup^k x$ with $y \in z$.
        $\bigcup^k x \subseteq t$ since $n$ was least.
        So $y \in z \in \bigcup^k x \subseteq t$.
        $t$ is transitive so $y \in t$.
        Hence $\bigcup^n x \subseteq t$ contradiction.
        Hence $t$ is the smallest transitive set with $x \subseteq t$.
    (4) Holds since if $y \in x$ then $\bigcup^n y \subseteq \bigcup^{n+1} x$ for all $n \in \omega$.
        So $TC(y) \subseteq TC(x)$.
    \end{proof}


    \subsection{Power Sets}
    Want to see that we can collect all subsets of a given set.
    This will allow us to build "bigger" sets in terms of cardinality.

    \textbf{Axiom 8: Power Set Axiom}
    $\forall x \exists y \forall z (z \subseteq x \rightarrow z \in y)$

    \textbf{Def:} The \underline{power set} of a set $x$ is $\mathcal{P}(x) = \set{z : z \subseteq x}$

    \underline{Ex:} $x = \set{0,1,2}$ then $\mathcal{P} = \set{\emptyset, \set{0}, \set{1}, \set{2}, \set{0,1}, \set{0,2}, \set{1,2}, \set{0,1,2}}$

    \emph{Note:} If x is a finite set with $n$ elements then $\mathcal{P}(x)$ has $2^n$ elements.

    \underline{Ex:} $\mathcal{P}(\omega)$ is the set of all sets of natural numbers.

    \emph{Note:} If $f : A \rightarrow B$ is a function then $f \subseteq A \times B$ so $f \in \mathcal{P}(A \times B)$.
    So we can make the following definitions.

    \textbf{Def:} $B^A$ or $prescript(A)B$ is the set of all functions $f$ with $dom(f) = A$ and $ran(f) \subseteq B$

    \emph{Note} $B^A \subseteq \mathcal{P}(A \times B)$

    \emph{Note: } When $A$ and $B$ are ordinals, this is not the same as ordinal exponentiation.
    In this case usually write $prescript(\beta)\alpha$ for functions $f : \beta \rightarrow \alpha$ different from the ordinal $\alpha^{\beta}$

    \underline{Ex: } Ordinal exponentiation $\omega^{\omega} = \omega \dot \omega \dot \omega \dots = Sup(n < \omega) \omega \dot \omega \dots \omega$ (n times)
    $prescript(\omega)\omega = $ all functions $f : \omega \rightarrow \omega$

    \textbf{Def:} When $\alpha$ is an ordinal,. $A^{<\alpha} = \bigcup(\xi < \alpha) A^{\xi}$
    this is the set of all sequences of elements of A of length $<\alpha$.

    \underline{Ex: } $A^{<\omega}$ is the set of all finite sequences of elements of A.

    \textbf{Def: } For $a_0,a_1,\dots,a_{m-1} \in A$ we write $(a_0,a_1,\dots,a_{m-1})$ for the sequence of length $m$ $S$ with $S \in A^m$ with $S(i) = a_i$ for $i<m$.

    \emph{Note: } Direct product can be justified using poer set axiom without needing replacement axiom.


    \section{Feb 18}

    \underline{Cardinality}

    \defn $X \preceq Y$ if there is an injection $f: X \rightarrow Y$.
    $X \approx Y$ if there is a bijection if $X \leftrightarrow Y$.
    $X \prec Y$ if $X \preceq Y$ but $Y \neq X$.

    \underline{Ex: } $\omega \approx \omega + 1 = \omega \cup \set{\omega}$
    $f(0) = \omega$
    $f(n) = n-1$ for $n \geq 1$

    \underline{Ex: } $\omega \approx \omega + \omega$
    $f(2n) = n$
    $f(2n +1) = \omega + n$ for $n \in \omega$

    \underline{Ex: } $\omega \times \omega \approx \omega$
    $f(n,m) = \frac{(n+m)(n+m+1)}{2} + n$
    they have the same cardinality

    (see graph in slides)

    \underline{Ex: } $\mathbb{P}(\mathbb{N}) \preceq \mathbb{R}$
    for $A \leq \mathbb{N} f(A) = \sum{n \in A} \frac{1}{3^n}$

    \underline{Ex: } $\mathbb{R} \preceq \mathbb{P}(\mathbb{N})$
    Let $\langle q_n : n \in \omega \rangle$ enumerate $\mathbb{Q}$
    Let $f(x) = \set{n \in \mathbb{N} : q_n < x}$
    If $x \neq y$ there is some $q_n$ with either $x < q_n < y$ or $y < q_n < x$
    So either $n \notin f(x) n \in f(y)$ or the opposite, so $f(x) \neq f(y)$

    \underline{\textbf{Question}} is there a bijection between $\mathbb{R}$ and $\mathbb{P}(\mathbb{N})$?

    \underline{Lemma: }
    (1) $\preceq$ is transitive
    (2) If $X \subseteq Y$ then $X \preceq Y$
    (3) $\approx$ is an equivalence relation

    %! Author = nathanaelsteven
%! Date = 2/18/25

%% Preamble
%\documentclass[11pt]{article}
%
%% Packages
% file already inlcudes from parent:
%
%\usepackage{amsmath}
%\usepackage{graphicx}
%\usepackage[colorlinks=true, allcolors=blue]{hyperref}
%\usepackage{amsfonts}
%\usepackage{amssymb}
%\usepackage{upgreek}
%\usepackage{amsthm}
%% Document
%\begin{document}

\newcommand{\set}[1]{\{#1\}}

\emph{Note: } If there is an injection $f : A \leftrightarrow B$ then there is a surjection $g : B \leftrightarrow A$ (A,B non-empty) (these arrows may be incorrect :)
pick $a_0 \in A$
define $g(b) = $ the unqiue $a \in A$ with $f(a) = b)$ if $b \in ran(f)$ $a_0$ if $b \notin ran(f)$

\empth{Note: } (Without Axiom of Choice) Having a suhrection $g : B \twoheadrightarrow A$ does not necessarily imply that there is an injection $f : A \hookrightarrow B$

\emph{Note: } we can have $A \subseteq B$ $A \neq B$ and $A \approx B$
$B = \omega A = \omega \setminus \{0\}$
$f : B \rightarrow A f(n) = n+1$
$f$ is a bijection.

\underline{Lemma: } If $B \subseteq A$ and $A \preceq B$ then $A \approx B$

\begin{proof}
    Let $f : A \hookrightarrow B$.
    Note $A \subseteq B \subseteq f(A) = ran(f) = f``A$
    $f^n = f \dot ... \dot f$ n times and $f^0 = identity$
    Let $H_n = f^n(A) \setminus f^n(B)$
    $K_n = f^n(B) \setminus f^{n+1}(A)$
    (diagram in slides)
    $B = f^0(B)$ and also for $A$
    we get slices of A
    Note that thr $H_n$'s and $K_n$'s are pairwise disjoint.
    Let $P = \bigcap (n \in \omega) f^n(A) = \bigcup (n \in \omega) f^n(B)$
    P disjoint from Hns and Kns
    $P_1$ $H_n$ for $n \in \omega$ and Kn for $\n \in \omega$
    partition $A$ into disjoint pieces.
    $A = P \cup H_0 \cup H_1 \cup \dots \cup K_0 \cup K_1 \dots$
    $B = P \cup H_1 \cup \dots K_0 \dots$
    Define $g : A \leftrightarrow B$ by
    $g(x) = \{f(x) if x \in \cup (n) H_n , x if x \in P \cup cup (n) K_n\}$
    (diagram in slides)
    $f(H_n) = H_{n+1}$
    $f : H_n \leftrightarrow H_{n+1}$
    So $g$ is a bijection from $A$ to $B$.
\end{proof}

\underline{Ex: } $A = [0,1] B = (0,1]$
$f : A \hookrightarrow B$ by $f(x) = \frac{x+1}{2}$.
$f$ is an injection.
(diagram in slides)
$H_0 = A \setminus B$
$K_0 = B \setminus f(A)$
$K_0 = (0,1/2)$
$H_0 = f(A) \setminus f(B)$
$H_0 = 0$
$K_1 = (1/2,3/4)$
$H_1 = 1/2$
$H_2 = 3/4$
$g(x) = \frac{x+1}{2}) if x = 1-\frac{1}{n} for n \geq 1, x if not$.


\underline{\textbf{Schroder-Bernstein Theorem:}} $A \approx B$ if and only if $A \preceq B$ and $B \preceq A$
\begin{proof}
    $\implies$ is immediate.
    $\impliedby$ is more difficult.
    Let $f : A \hookrightarrow B$ and $g : B \hookrightarrow A$
    Let $B^~ = h(B) \subseteq A$
    $B \approx B^~$ via $h$.
    $B^~ \subseteq A$ and $g \circ f : A \hookrightarrow B^~$
    By Lemma there is a bijection $h : A \leftrightarrow B^~$
    \end{proof}
    (went ot bathroom)
    an example and two notes missed
\textbf{\underline{Cantor's Theorem:}} For any set $A$ $A \prec \mathbb{P}(A)$
\begin{proof}
    Let $f(x) = \set{x}$ $f : A \hookrightarrow \mathbb{P}(A)$ so $A \preceq \mathbb{P}(A)$
    Show there is no bijection from $A$ to $\mathbb{P}(A)$
    Let $f : A \rightarrow \mathbb{P}(x)$
    show $f$ is not a surjection.
    Let $D = \set{x \in A : x \notin f(x)}$
    $D \in \mathbb{P}(A)$
    claim $D \notin ran(f)$
    Suppose $D = f(x)$ for some $x$.
    $x \in D \iff x \notin f(x)$.
    $\iff x \notin D$.
    contradiction.
    so $D \notin ran(f)$.
    \end{proof}

\underline{Ex: } $\mathbb{N} \prec \mathbb{P}(\mathbb{N})$ so $\mathbb{N} \prec \mathbb{R}$.

\underline{Ex: } $ A = \mathbb{N}$.
$f : \mathbb{N} \rightarrow \mathbb{P}(\mathbb{N})$
(illustration)
$D$ can'r appear as a row in table.

\underline{Cantor's Paradox: } There is no universal set.
Suppose $V$ were a universal set.
Then $\mathbb{P}(V) \subseteq V$
$\mathbb{P}(V) \preceq V$
conradiction.

\emph{Note: } $\mathbb{P}(A) \approx 2 (left super A) = \set{f : f is a function f: A \rightarrow \set{0,1}}$
$X \subseteq A \hookrightarrow f_x(a) = \set{1 if a \in X, 0 if a \notin X}$

\emph{Note: } $2^{\mathbb{N}}$

\emph{Note: } $A (left super B) (all left super C) \approx A (left super C \times B)$ similar to $(x^y)^z = x^y^z$
$A (left super B \cup C) \approx A (left super B) \times A (left super C)$ for (something)

\defn A set $A$ is \underline{countable} if $A \preceq \omega$
$A$ is \underline{finite} if $A \preceq n$ for some $n \in \omega$
$A$ is \underline{infinite} if $A$ is not finite.
$A$ is \underline{uncountable} if $A$ is not countable.

\underline{Ex: } $\mathbb{Q}$ is countable.

\underline{EX: } $\mathbb{P}(\omega)$ is uncountable

\defn A (von-Neumann) \underline{cardinal} is an ordinal $\alpha$ so that $\xi \npreceq \preceq \alpha$ for all $\xi < \alpha$

\emph{Note: } Equivalent to saying $\nexists \xi < \alpha (\xi \approx \alpha)$
call these \underline{initial ordinals}

\underline{Ex: } $n \in \omega$ 0,1,2 \dots are cardinals.
$\omega$ is a cardinal.

\underline{Ex: } $\omega + 1, \omega + \omega$ are not cardinals.


\section{Feb 20}

\underline{Def: } A \underline{Cardinal} is an ordinal $\alpha$ so that $\forall \xi < \alpha \xi \prec \alpha$

\textbf{Thm}
(1) Every cardinal $\geq \omega$ is a limit ordinal
(2) Every natural number is a cardinal
(3) If $A$ is a set of cardinal then $supA = \bigcup A$ is a cardinal
(4) $\omega$ is a cardinal

\begin{proof}
(1) Suppose $\alpha \geq \omega$ is a cardinal
    Suppose $\alpha$ is a successor $\alpha = \delta + 1$
    So $\delta < \alpha$
    Define $f: \alpha \rightarrow \delta$ ($\alpha = \delta \cup \set{\delta}$)
    $f(\delta) = 0$
    $f(n) = n+1$ for $n \in \omega$
    $f(\xi) = \xi$ for $\omega \leq \xi < \delta$
    $f$ is a bijection, so $\alpha \appox \delta$ contradicting that $\alpha$ is a cardinal.
(2) use ordinary inducitno
    $0 = \emptyset$ is a cardinal vacuously.
    Suppose $n$ is a cardinal ($n \in \omega$).
    Suppose $S(n)=n+1$ is not a cardinal.
    Then there is some $\xi < n+1$ with $\xi \approx S(n)$
    Let $f : \xi \leftrightarrow S(n) = n \cup \set{n}$
    Can't have $\xi = 0$ so $\xi=S(m)$ for some $m$.
    $f : S(m) =m \set{m} \leftrightarrow S(n) = n \set{n}$
    Note $m < n$
    Let $a$ be such that $f(a) = n$
    If $a=m$ then $f \upharpoonright m : m \leftrightarrow n$ contradicting that $n$ is a cardinal.
    If $a \neq m$ define $g : m \leftrightarrow n$ by $g(a) = f(m)$ and $g(b) = f(b)$ for $b \neq a$
(diagram in slides)
    so $g$ is a bijection fro m to n contradicting that $n$ is a cardinal.
(3) Let $A$ be a set of  ardinal if $sup A$ is not a cardinal.
    Let $\xi < sup A$ so that $\xi \approx sup A$ (Supremum of A is least upper bound)
    There is $\alpha \in A$ with $\xi < \alpha \leq sup A$
    $\xi \preceq \alpha$ and $\alpha \preceq sup A \approx \xi$ so $\alpha \preceq \xi$
    By Schroder-Bernstein $\alpha \approx \xi$ so $\alpha$ is not a cardinal.
(4) $\omega$ is a cardinal by (2) and (3)
    Let A be $\omega$.
    $sup A = \omega$ (omega least infinite ordinal)
    so $\omega$ is a cardinal.
\end{proof}

\underline{Def:} A set is \underline{well-orderable} if there is a relation R so that R well-orders A.

\emph{Note:} A is well-orderable if and only if there is an ordinal $\alpha$ so that $A \approx \alpha$
For $\leftarrow$ If $A \approx \alpha$ let $f : A \leftrightarrow \alpha$
Define R on A by $a_1 R a_2 \leftrightarrow f(a_1) < f(a_2)$
Then $(A;R) \cong (\alpha, \in)$
So A is well-ordered by R.

\emphe{Note:} Every ordinal is in bijection with some unique cardinal.
Given $\alpha \in Ord$ let $\beta$ be the $\in$-least element of
\set{\xi \leq \alpha : \alpha \approx \xi}
So $\alpha \approx \beta$ and $\beta$ is a cardinal.
If $\xi < \beta$ then $\xi \notint$ (above set) so $\beta \neg \approx \xi$

\underline{Def:} If A is well-orderable, let $|A|$ be the least ordinal $\alpha$ so that $A \approx \alpha$ call $|A|$ the \underline{cardinality} of A.

\underline{Ex: } If $\alpha$ is a cardinal $|\alpha| = \alpha$
$|omega + omega| = \omega$

\underline{Ex: } Any set of ordinals is well-orderable.

\textbf{Lemma: } Let $A,B$ be well-orderable sets then:
(1) $|A|$ is a cardinal
(2) $A \preceq B \iff |A| \leq |B|$
(3) $A \approx B \iff |A| = |B|$
(4) $A \prec B \iff |A| < |B|$

\begin{proof}
(1) above
(2) $\implies$ suppose $f : A \hookrightarrow B$
    Let $g : B \approx |B|$
    then $g \circ f : A \hookrightarrow |B|$
    $(g \circ f)`` A \subseteq |B|$
    so well-orderable
    Type $((g \circ f)``A;\in) \leq type(|B|,\in) = |B|$
    $\impliedby$ suppose $|A| \leq |B|$
    $A \approx |A| \subseteq |B| \approx B$
    So $A \preceq B$ (an injection exists).
(3) $\implies$ Suppose $A \approx B$
    $|A| \approx A \approx B \approx |B|$
    can't have $|A| < |B|$ or $|B| < |A|$
    so $|A| = |B|$
    $\impliedby$ if $|A| = |B|$ then $A \approx |A| = |B| \approx B$
(4) Follows from (2) and (3).
\end{proof}

\emph{Lemma:} If $A$ is well-orderable and there is a surjection $f: A \twoheadrightarrow B$ then $B$ is well-orderable.
\emphe{proof} Let R be a well-ordering of A.
Define S on B by $b_1 S b_2 \iff min(f^-1(\set{b_1})) R min(f^-1(\set{b_2}))$
Then S is a well-ordering of B.

\emph{Note:} We haven't yet seen that there are uncountable cardinals.
We saw $\omega \prec \mathbb{P}(\omega)$ so $\mathbb{P}(\omega)$ is uncountable, but we don't know $\mathbb{P}(\omega)$ is well-orderable.
Need axiom of choice to see that $\mathhbb{P}(\omega)$ is well-orderable.

\textbf{Thm \underline{Hartog's Theorom} : } For every set $A$ there is a cardinal $k$ so that $k \npreceq A.$
Hence if A is a well-orderable set then there is a cardinal $k$ so that $|A| < k$.

\begin{proof}
    Let $W$ be the set of pairs of the form $(x,R)$ in $\mathbb{P}(A) \times \mathbb{P(A \times A)}$ so that $x \subseteq A, R \subseteq \A \times \A$
    and $R$ well-orders $x$.
    Thus $W$ is the set of all well-orderings of subsets of $A$.
    Note for $\alpha \in ON$
    $\alpha \preceq A \iff \alpha = type(x;R)$ for some $(x,R) \in W$
    Let $\beta = sup \set{type(x,R) + 1 : (x,R) \in W}$
    If $\alpha \preceq A$ then $\alpha < \beta$
    Let $\kappa = |\beta|$ then $\kappa \approx \beta$, $\kapp$ is a cardinal
    $\kappa \npreceq A$
\end{proof}

\underline{Def: } Let $\aleph(A)$ be the least cardinal $k$ so that $k \npreceq A$
Called the \underline{Hartog's number or aleph} of A.

\underline{Def: } For an ordinal $\alpha$ let $\alpha^+ = \aleph(\alpha)$

\emph{Note: } $|\alpha| < \alpha^+$
For a cardinal $\kappa$, $\kappa^+$ is the next biggest cardinal.

\textbf{Def: } \underline{The Aleph Sequence}
Define by recursion of $\xi \in ON$
$\aleph_0 = \omega_0 = \oemga$
$\aleph_{\xi+1} = \omega_{\xi + 1} = (\aleph_\xi)^+$
$\aleph_\lambda = \omega_\lambda = sup \set{\aleph_\xi : \xi < \lambda}$ for $\lambda$ a limit ordinal.

\emph{Note:} Generally use $\aleph_\xi$ when talking about cardinality, use $\omega_\xi$ when talking about order type (they are the same).

We shall see that all infinite cardinal appear in this sequence somewhere

\emph{Note:} $\aleph_1$ is the least uncountable ordinal/cardinal.
$\omega_1 = \set{\alpha : \alpha is a countable ordinal}$

\emph{Note:} ordinal arithmetic does not increase cardinality for infinite ordinals.
For $\alpha, \beta \geq \omega$
$|\alpha + \beta| = |\alpha \dot \beta| = |\alpha^\beta| = max(|\alpha|,|\beta|)$
cardinal exponentiation will be different than ($2^omega = 2 \dot 2 \dots = \omega$)

\textbf{Thm:} If $\alpha \geq \omega$ then $|\alpha \times \alpha| = |\alpha|$

\emph{Note:} $\alpha \times \alpha$ is well-orderable by $<_{lex}$

\begin{proof}
    It suffices to prove that when $\kappa$ is a cardinal.
    Since if $|\alpha| = \kappa$ then $\alpha \approx \kappa$
    and $\alpha \times \alpha \approx \kappa \times \kappa$
    $\kappa \preceq \kappa \times \kappa$ by $f(\xi) = (\xi, \xi)$ for $\xi < \kappa$
    Need to show $|\kappa \times \kappa | \leq \kappa$ for $\kappa$ a cardinal.
    Produce a well-ordering of $\kappa \times \kappa$ of order type $\kappa$.
    Define a relation on $ON \times ON, \triangleleft$ that is a well-ordering of $ON \times ON$
    in the sense $\in$ is a well-ordering of $ON$.
    Such that $type(\kappa \times \kappa, \triangleleft) = \kappa$ when $\kappa \subseteq \omega$ is a cardinal
    Define $(\xi_1,\xi_2) \triangleleft (\eta_1,\eta_2)$
    if $max(\xi_1,\xi_2) < max(\eta_1,\eta_2)$ \vee
    $[max(\xi_1,\xi_2) = max(\eta_1,\eta_2) \wedege max(\xi_1,\xi_2) <_{lex} max(\eta_1,\eta_2)]$
    (diagram in slides).
    $\triangleleft$ is different than $<_{lex}$

    \section{Feb 25}
    want to show $\triangleleft$ is a well-ordering on $ON \times \ON$ so rthat $type(\kappa \times \kappa, \triangleleft) = \kappa$ for $\kappa \subseteq \omega$ a cardinal.

    Straightforward to show $\triangleleft$ is a well-ordering of $ON \times ON$ (in the sense that $\in$ well-orders $ON$)

    Let $S \subseteq ON \times ON$ let $type(S) = type(S, \triangleleft \upharpoonright S \times S)$

    \underline{Ex:} $type(omega \times \omega) = \omega$ (drawing in slides, the weird moving around where max element is limit)
    Now for $type(\omega + 1 \times \omega + 1) = \omega \cdot 3 + 1$ (similar at first during omega. final pair is omega,omega)

    Show $type(\kappa \times \kappa) = \kappa$
    Suppose not and let $\kappa$ be the least infinite cardinal so that $\delta=type(\kappa \times \kappa) \neq \kappa$
    If $\alpha < \kappa$ is an ordinal then $|\alpha \times \alpha| < \kappa$
    Let $F : \delta \leftrightarrow \kappa \times \kappa$ be the isomorphism $(\delta ; \in)$ to $(\kappa \times \kappa, \triangleleft)$
    If $\delta > \kappa$ let $(\xi_1, \xi_2) = F(\kappa)$ (dis)
    Let $\alpha = max(\xi_1,\xi_2) + 1 < \kappa$ since $\kappa$ is a limit ordinal.
    $F``\kappa$ (image of kappa under F) is a subset of $\alpha \times \alpha$ from definition of $\triangleleft$
    So $\kappa \preceq \alpha \times \alpha \prec \kappa$ (injection in to alpha x alpha)
    Contradiction.
    If $\delta < \kappa$ then $\kappa \npreceq \kappa \times \kappa \approx \delta \prec \kappa$ another contradiction
    So $\delta = \kappa$
\end{proof}


Ordinal arithmetic doesnt increase cardinality for infinite cardinals

\underline{Question} Is every set well-orderable?
Alot are but not simple to show.

ZF e.g.\ ($\neg$ AC) does not prove that every set is well-orderable.

\subsection{Axiom 9: Axiom of Choice}
$\forall F [[\emptyset \notin F \wedge \forall x ,y \in F (x \neq y \rightarrow x \cap y \neq \emptyset)]$
(F is a family fo disjoint non-empty sets)
$\rightarrow \exists C \forall x \in F \exists! y (y \in C \cap X]$
(there is a choice set for F)
(dis)

\dfn For a family F of sets, a \underline{choice set} for F is a set C so that $C \cap X$ is a singleton for each $X \in F$

\emph{Note:} If ,e.g., UF is well-orderable (example of getting AC from well-orderable)
We don't need AC to find a choice set.

We want to see various other statements follow from and are equivalent to AC.

\underline{Lemma}
AC implies: if there is a surjection $f: A \twoheadrightarrow B$ then there is an injection $g: B \hookrightarrow A$

\begin{proof}
    Let $F = \set{f^{-1}(\set{b}): b \in B}$ (dis)
    F is a family of disjoint of non-empty sets.
    AC implies there is a choice set C for F so that $C \cap f^{-1}(\set{b})$ is a singleton for each $b \in B$
    Define $g: B \hookrightarrow A$ by $g(b) = $ the unique element of $C \cap f^{-1}(\set{b})$

\end{proof}

\dfn We say two statements are $\Phi$ and $\Psi$ are \underline{equivalent in ZF} if
$ZF + \Phi$ proves $\Psi$ and $ZF+\Psi$ proves $\Phi$

\underline{Theorem} The following are equivalent in ZF:
(1) Axiom of Choice
(2) Every family of non-empty sets has a choice function.
(2`) The direct product of any family of non-empty sets  is non-empty.
(2``) Every set has a choice function.
(3) Every set is well-orderable.
(4) $\forall x \forall y (x \preceq y \vee y \preceq x)$.
(5) Tukey's Lemma.
(6) The Hausdorf maximality principal.
(7) Zorn's Lemma.

\dfn Let F be a family of non-empty sets.
A \underline{choice funciton} for F is a function g with $dom(g) = F$ and so $g(x) \in x$ for ecah $x \in F$

\dfn A choice function for a set X is a choice function fro the family $F = \mathcal{P}(x) \setminus\set{\emptyset}$

\dfn For a family of sets F the \underline{direct product} of F, $\Pi F$ consists of all function whose $dom(f) = F$ and $f(x) \in X$ for all $x \in F$

\underline{Ex: } $\Pi \set{X, Y} = $ all function f where $dom(f) = \set{X,Y}$ with $f(z) \in Z for z \in \set{X,Y}$
i.e.\ $f(x) \in X f(y) \in Y$
$f = \set{\langle X, f(x) \rangle,\langle y, f(y) \rangle} \approx \langle f(x), f(y) \rangle$
There is a natural bijection betwwn this and $X \times Y$

\define{\RR}{\mathbb{R}}

\underline{Ex:} For $x \in \RR$ let $[x] = \set{y \in \RR : y-x \in \mathbb{Q}}$
Let $F = \set{[x] : x \in \RR}$
F is a family of disjoint non-empty sets.
Let C be a choice set for $\neq$.
C contains exactly one element from each $[x]$.
Known as a Vitali set.
No explicitly definable such set.

\textbf{\underline{Thm}} $(1) \leftrightarrow (2) \leftrightarrow (2`) \leftrightarrow (2``)$
\begin{proof}
$(1) \implies (2)$ Let F be a family of non-empty sets.
    Show F has a chocie function.
    Let $A = \set{\set{x} \times X : X \in F}$
    A is a family of non-empty sets disjoint sets.
    Using AC there is a choice set for A.
    $C \cap \set{x}\times X$ is a singleton $\langle x, y \rangle$ with $y \in X$ for each $x \in F$.
    C is a function since for each $x \in F$ there is a unique y with $\langle x, y \rangle \in C$.
    $dom(X) = F$
    $f(x) \in X$ for each $x \in F$
    Hence C is a choice function for F.
$(2) \leftrightarrow (2`)$ is immediate just in different terminology.
    elements of $\Pi F$ are exactly the choice functions for F.
$(2) \implies (2``)$ Let X be a set.
    Show there is achocie fucntion for x, i.e. a choice function for $F = \mathcal{P} \setminus \set{\emptyset}$.
    immediate from $(2)$.
$(2``) \implies (1)$ Let F be a family of disjoint non-empty sets.
    Show that there is a choice set for F.
    Let $A = \bigcup F$
    By (2``) there is a choice funciton for $A: g : \mathcal{P}\setminus \set{\emptyset} \rightarrow A$.
    $g(x) \in X$ for $x \in \mathcal{P}(A)\setminus \set{\emptyset}$
    Let $C = \set{g(x) : x \in F}$
    $C \cap X$ is a singleton for each $X \in F$.
    So C is a choice set for F.
\end{proof}

\textbf{\underline{Thm}} AC is equivalent to "every set is well-orderable".
\begin{proof}
    Show for any set $A$ that A is well-orderable $\iff$ there is a choice function for $\mathcal{P}(A) \setminus \set{\emptyset}$
    $\implies$ Let R be a well-ordering of A.
    Define $g$ by $g(x) = R-least$ element of x for $x \in \mathcal{P}(A)\setminus \set{emptyset}$.
    So g is a choice function for $\mathcal{P}(A) \setminus \set{\emptyset}$.
    $\impliedby$ Let g be a choice function for $\mathcal{P}(A) \setminus \set{\emptyset}$.
    Show A is well-orderable (by bijection with an ordinal).
    Let $\kappa = \aleph(A)$ (i.e.\ $\kappa$ is athe least cardinal with $k \neq A$)
    Let S be any set $S \notin A$. "stop".
    Let $f: \kappa \rightarrow A \cup \set{S}$ be defined by transfininte induction.
    $f(\alpha) = $ \begin{cases} $g(A \setminus \set{f(\xi) : \xi < \alpha})$  $\alpha < \kappa$\\ S \text{otherwise} \end{cases}
    Note if $\alpha < \beta$ and $f(\beta) \neq S$ then $f(alpha) \neq S$ and $f(\alpha) \neq f(\beta)$
    Note $\kappa \npreceq A$ there is some least $\alpha < \kappa$ with $f(\alpha) = S$.
    But then $\set{f(\xi) : \xi < \alpha} = A$
    and $f \upharpoonright \alpha : \alpha \leftrightarrow A$
    So $\alpha \approx A$ so A is well-orderable.
    \end{proof}

%\end{document}
    %! Author = nathanaelsteven
%! Date = 2/27/25

\section{Feb 27}

Recall Axiom of Choice: Every family of disjoint non-empty sets has a choice set.
Equivalent to: Every set is well-orderable.

\textbf{Theorem} AC is equivalent to:
\[\forall x \forall y (x \preceq y \vee y \preceq x)\]

\begin{proof}
    $\implies$ Every set can be well-ordered so there are cardinal $|x| \approx x$ and $|y| \approx y$
    Either $|x| \leq |y|$ or $|y| \leq |x|$, so either $x \preceq y$ or $y \preceq x$.
    $\impliedby$ Given a set $x$ let $\kappa = \aleph(x)$ so $\kappa \npreceq x$
    Then $x \preceq \kappa$ so x is well-orderable
\end{proof}

\dfn For $\mathcal{F} \subseteq \mathcal{P}(A)$, say $X \in \mathcal{F}$ is \underline{maximal in $\mathcal{F}$} if x is maximal wirth respect to $\subseteq$
i.e. $\neg \exists Y (Y \in \mathcal{F} \wedge X \subseteq Y \wedge X \neq Y)$

\underline{EX:} $\mathcal{F} = \mathcal{P}(A)$.
$X=A$ is the unique maximal element in $\mathbb{F}$

\underline{EX:} $\mathcal{F} \subseteq \mathcal{P}(\omega)$ consists of all finite subsets of $\omega$.
No maximal element in $\mathbb{F}$

\underline{Ex:} Let $V$ be a vector space $\dim(V) \geq 1$
Let $\mathcal{F} \subseteq \mathcal{P}(V)$ consist of all linearly independent subsets of $V$.
$X \in \mathbb{F}$ is maximal iff X is a basis for V.
Many different maximal elements.

\dfn A family $\mathcal{F} \subseteq \mathcal{P}(A)$ is of \underline{finite character} if for any
$X \subseteq A$ $X \in \mathbb{F} \iff$ every finite subset of X is in $\mathbb{F}$

\underline{Ex: } $\mathcal{F} = \mathcal{P}(A)$ trivially of finite character
\underline{Ex: } $\mathcal{F} \subseteq \mathcal{P}(\omega)$ not of finite character
\underline{Ex: } $\mathcal{F} \subseteq \mathcal{P}(V)$ consists of all linearly independent subsets of $V$ is of finite character


\underline{Ex:} Let $K \subseteq \mathbb{R}$ be compact
Let $A = $ open subsets pf $\mathbb{R}$
Let $\mathbf{F} = \set{g \in \mathcal{P}(A) : \bigcup g \text{does not cover} K}$
$\mathcal{F}$ is of finite character

\emph{Note: } If $\mathcal{F}$ is of finite character and $X in \mathcal{F}$ and $Y \subseteq X$ then $Y \in \mathcal{F}$

\dfn \textbf{Tukey's Lemma} Whenever $\mathcal{F} \subseteq \mathcal{P}(A)$ is of finite character and $X \in \mathcal{F}$
then there is a maximal $Y \in \mathcal{F}$ with $X \leq Y$

\underline{Ex: } Tukey's Lemma implies that any linearly independent set of vectors can be extneded to a basis

\textbf{Thm} AC is equivalent to Tukey's Lemma.

\newcommand{\FF}{\mathcal F}
\newcommand{\PP}{\mathcal{P}}

\begin{proof}
    \textbf{$AC \implies TL$}
    Let $\FF$ $\subseteq$ $\PP(A)$ be of finite character.
    A can be well-ordered, so let
    $\kappa = |A|$ anf let $f: \kappa \leftrightarrow A$
    $A = \set{ x_\alpha : \alpha < \kappa}$ where $x_\alpha = f(\alpha)$ for $\alpha < kappa$
    Define sets $Y_\alpha$ for $\alpha < \kappa$ by transfinite recursion:
    Given $X \in \FF$, let
    $Y_0 = X$
    $Y_{\xi +1} = \begin{cases} Y_\alpha \cup \set{x_\alpha} & \text{ if } Y_{\alpha} \cup \set{X_\alpha} \in \FF \\ Y_\alpha & \text{ otherwise } \end{cases}$
    $Y_\lambda = \bigcup \set{Y_\alpha : \alpha < \lambda}$ for $\lambda$ a limit ordinal.
    Let $Y =  \bigcup_{\alpha < \kappa} Y_\alpha$
    Then $X \leq Y$
    Each $Y_\alpha \in \FF$ because $\FF$ is of finite character.
    If $Y_{\lambda} \notin \FF$ there is some finite subset $Z \subseteq Y_\lambda$ with $Z \notin \FF$ there would
    be some $\alpha < \lambda$ with $Z \subseteq Y_\alpha$ contradicting $Y_\alpha \in \FF$
    Y is maximal in $\FF$
    Suppose not, let $Z \in \FF$ with $Y \subseteq Z$ but $Y \neq Z$
    some element $X \in Z \setminus Y$
    $X = X_\alpha$ for some ordinal $\alpha$
    $Y_{\alpha+1} = Y_{alpha} \cup \set{x_\alpha} \in \FF$ subset of Z.
    Contradicting $x_\alpha \notin Y$
    \textbf{$TL \implies AC$}
    Let $\FF$ be a family of disjoint non-empty sets.
    See ther eis a choice set for $\FF$.
    Let $A = \bigcup \FF$ and let $\mathcal{G} \subseteq \PP(A)$ consist of all partial choice sets for $\FF$.
    i.e. $D \in \mathcal{G} \iff D \subseteq \bigcup \FF$ and $|X \cap D| \geq 1$ for all $X \in \FF$
    Note $\emptyset \in \mathcal{G}$
    $\mathcal{G}$ is of finite character since $D \in \PP(A)$ and $D \notin \mathcal{G}$ then there is $X \in \FF$ with $|X \cap D| > 1$
    Let $B \subseteq X \cap D$ contain 2 elements of $X \cap D$ then $B$ si a finite subset of $D$ with $B \notin \mathcal{G}$.
    By Tukey's Lemma there is a maximal $C \in \mathcal{G}$.
    Then C is a choice set for $\mathcal{G}$
    Since if there were $X \in \FF$ with $C \cap X = \emptyset$  then
    let $z \in X$ then $C \cup \set{z} \in \mathcal{G}$
    contradicting that C was maximal.
\end{proof}

\dfn Let < be a strict partial order of a set A.
Then $C \subseteq A$ is a \underline{chain} if C is totally ordered by <.
We say that a chain $C \subseteq A$ is a \underline{maximal chain} if there is no chain $D$ with $C \leq D$ and $C \neq D$

\underline{Ex: } $A = \mathbb{N} \setminus \set{0}$ set $n \prec m$ if m is a proper multiple of n.
$\set{3,6,30}$ is a chain.
not maximal since $\set{3,6,30,60}$ is a chain with $C \leq D$ and $C \neq D$
$\set{2^k : k \geq 0}$ is a maximal chain since $n \neq 2^k$ for any $k \geq 0$
let k so $n < 2^k$ n is not a multiple of $2^k$ and $2^k$ is not a multiple of n
Hence neither $n \prec 2^k$ not $2^k \prec n$ and $n \neq 2^k$
So $\set{2^k : k \geq 0} \cup \set{n}$ is not a chain.

\dfn \underline{Hausdorff Maximal Principle}.
Whenever < is a strict partial order of a set A, there is a maximal chain $C \subseteq A$ with $C \leq A$
(probably missing stuff)

\dfn \underline{Zorn's Lemma} ?If every chain in a partially ordered set has an upper bound, then there is a maximal element.?
(missing stuff)

\underline{Ex: } $n < m$ if m is a proper multiple doesn't satisfy (*)

\underline{Ex: } $A = \PP(\omega)$ x<y if $x \subset Y$
Let C be a chain in A
Let $Z = \bigcup C$ then Z is an upper bound for C.
so this satisfies (*)

\textbf{Thm} AC is equivalent in ZF to both Hausdorff Maximality Principle and Zorn's Lemma.

\begin{proof}
    \textbf{Tukey's Lemma $\implies$ HMP}
    Given a strict partial order of set A.
    Let $\FF$ consist of all chains in A.
    $\FF$ is of finite character.
    If C is not a chian there are $x,y \in C$ s.t. $x < y, y < x, y \neq x$ then $n=\set{x,y}$ is not a chain
    By Tukey's Lemma there is a maximal element of $\FF$.
    i.e. a maximal chain in A.
    \textbf{HMP $\implies$ Zorn's Lemma}
    < is a strict partial order of a set A satisfying (*) (every chain has an upper bound)
    Let $a \in A$
    Let $A` = \set{\text{all} b \in A \text{with} a \leq b}$
    Apply HMP to A` and <.
    We get a maximal chain $C \in A`$.
    By (*) there is an upper bound b for chain.
    so $a \leq b$ and b is a maximal element of A` and hence of A.
    Since if b < d then d is an upper bound for C and b is not maximal.
    so $C \cup \set{d}$ would be a chain contradicting C is a maximal chain.
    \textbf{Zorn's Lemma $\implies$ Tukey's Lemma}
    Let $\FF \subseteq \PP(A)$ be of finite character.
    Let X < Y if $X \subset Y$.
    This satisfies (*)
    Given a chain its union is an upper bound
    C in $\FF$ since $\FF$ is of finite character.
    by Zorn's Lemma there is a maximal element w.r.t. <.
    i.e. a maximal element in $\FF$.
\end{proof}

\textbf{Thm} (AC) Let $\kappa \geq \omega$ be a cardinal.
If $\FF$ is a family of sets wtih $|\FF| \leq \kappa$ and $|X| \leq \kappa$ for each $X \in \FF$ then $|\bigcup \FF| \leq \kappa$
\underline{Cor} the union of countably many countable sets is countable.


\newcommand{\leftsuperscript}[2]{{\vphantom{{#1}}}^{#2}{#1}}

\begin{proof}
    We can assume $\FF \neq \emptyset$ and $\emptyset \notin \FF$
    Have $f : \kappa \twoheadrightarrow \FF$
    Using AC we can choose $g_\alpha : \kappa \twoheadrightarrow f(\alpha) \in \FF$
    well-order $\leftsuperscript{\kappa}{\left(\bigcup \FF\right)} $ and let $g_\alpha$ be least with $ran(g_\alpha)=f(\alpha)$
    Define $h: \kappa \times \kappa \twoheadrightarrow \bigcup \FF$ by
    $h(a,b) = g_{f(a)}(b)$
    Proveing $|\kappa \times \kappa| = \kappa$ so there is a surjection for $\kappa$ onto $\bigcup \FF$ so $|\bigcup \FF| \leq \kappa$
    \end{proof}

\section{March 4th}

\subsection{Cardinal Arithmetic}
\emph{Note:} We'll be assuming AC throughout.
Every set can be well ordered.
$\forall x |x|$ exists i.e. $|x|$ is a cardinal with $|x| \approx x$

Define $\kappa + \lambda, \kappa \cdot \lambda, \kappa^\lambda$
use boxes to distinguish cardinal arithmetic from ordinal arithmetic.

\dfn For cardinal $\kappa$ and $\lambda$:
$\kappa + \lambda = | \set{0} \times \kappa \cap \set{1} \times \lambda |$
similar ot ordinal but now instead of lookign at the order type we look at the cardinality

$\kappa \cdot \lambda = |\kappa \times \lambda|$



$\kappa^\lambda = |\leftsuperscript{\lambda}{\kappa} $
The set of all function $f : \lambda \rightarrow \kappa$

\emph{Note:} $2^\omega$ the ordinal $= \sup\set{2^n : n < \omega} = \omega$
the cardinal = $\leftsuperscript{\omega}{2} = |\PP(\omega)| > \omega$

\emph{Note: } cardinal $\leftsuperscript{0}{0} = | \set{f : \emptyset \rightarrow \emptyset}| = |\set{\emptyset}| = 1$

\textbf{Lemma}
For Cardinals $\kappa \leq \kappa`$ and $\lambda \leq \lambda`$
(1) $\kappa + \lambda \leq \kappa` + \lambda`$
(2) $\kappa \cdot \lambda \leq \kappa` \cdot \lambda`$
(3) $\kappa^\lambda \leq \kappa`^{\lambda`}$ unless $\kappa = \kappa` = \lambda = 0$


    Straightwforward proof, produce injections from the underlying sets.
    e.g. $f : \kappa \times \lambda \hookrightarrow \kappa` \times \lambda`$

\textbf{Lemma}
For cardinal $\kappa, \lambda, \theta$
(1) $\kappa + \lambda = \lambda + \kappa$
(2) $\kappa \cdot \lambda = \lambda \cdot \kappa$
(3) ($\kappa \cdot \lambda) \cdot \theta = \kappa \cdot \theta + \lambda \cdot \theta$
(5) $\kappa^{\lambda \cdot \theta} = \kappa^{\lambda^\theta}$
(6) $\kappa^{\lambda + \theta} = \kappa^\lambda \cdot \kappa^\theta$

proof: produce bijection

\textbf{Lemma}
For ordinals $\alpha, \beta$
(1) $|\alpha + \beta| = |\alpha| + |\beta|$
(2) $|\alpha \cdot \beta| = |\alpha| \cdot |\beta|$

proof: immediately from the definitions, bijection

\textbf{Lemma}
For finite ordinals or cardinals
arithemetic is the same but not for infinite ordinal. i.e. cardinal + equals ordinal +

\textbf{Lemma}
For cardinals $\kappa, \lambda$ with at least one infinite
(1) $\kappa + \lambda = \max(\kappa,\lambda)$
(2) $\kappa \cdot \lambda = \max(\kappa,\lambda)$ if neither 0.

proof: say $\kappa \leq \lambda$ so $\lambda$ is infinite.
(1) $\kappa + \lambda = |\set{0} \times \kappa \cup \set{1} \time \lambda | \leq |\set{0}\times\lambda \cup \set{1} \times \lambda$ = $|\set{0,1}\times \lambda| \leq |\lambda \times \lambda| = |\lambda| = \max(\kappa,\lambda)$
(2) similar

\textbf{Lemma}
For $\kappa$ cardinal
$2^\kappa = |\PP(\kappa)| > \kappa$
pf: $f : \kappa \rightarrow 2 \mapsto \set{\alpha < \kappa : f(\alpha) = 1} \in \PP(\kappa)$ gives a bijection
Cantor's theoremL $|\PP(\kappa)| \npreceq \kappa$ so $|\PP(\kappa)| \nleq \kappa$
So $|\PP(\kappa)| > \kappa$

Recall $\aleph(\kappa) =$ least cardinal $\lambda$ so that $\lambda \npreceq \kappa$.
with AC $\aleph(\kappa) = $ least cardinal $> \kappa$
$\kappa^+ = \aleph(\kappa)$
$\aleph_0 = \omega$
$\aleph_{\alpha + 1} = \aleph_\alpha^+$
$\aleph_\lambda = \sup\set{\aleph_\alpha : \alpha < \lambda}$ for limit ordinal $\lambda$

\emph{Note: } $2^{\aleph_0}$ is uncountable
$2^{\aleph_0} > \aleph_0$
$2^{\aleph_0} \geq \aleph_1$
ZFC doesn't determine whether or not $2^{\aleph_0} = \aleph_1$
AC needed to define $2^{\aleph_0}$

\dfn \underline{The Continuum Hypothesis}(CH) is the statement $2^{\aleph_0} = \aleph_1$
\underline{The Generalized CH} is the statement that for every ordinal $\alpha$ $2^{\aleph_\alpha} = \aleph_{\alpha+1}$

\emph{Note:} CH is independent of ZFC

\newcommand{\NN}{\mathbb N}
\newcommand{\ZZ}{\mathbb Z}
\newcommand{\QQ}{\mathbb Q}
\newcommand{\RR}{\mathbb R}
\emph{Note: } $|\RR| = 2^{\aleph_0}, |\NN| = \aleph_0$
CH is equivalent to: every set of real numbers is either countable or has the same cardinality as $\RR$

\emph{Note: } Knowing $2^\lambda$ for all infinite $\lambda$ allows us to compute $\kappa^\lambda$ for many other $\kappa$.
Assuming GCH we can give an explicit formula for $\aleph_\alpha^{\aleph_\beta}$

\textbf{Lemma} If $2 \leq \kappa \leq 2^\lambda$ and assume $\lambda$ is infinite
then $\kappa^\lambda = 2^\lambda$

pf: $2^\lambda \leq \kappa^\lambda \leq (2^\lambda)^\lambda = 2^{\lambda \cdot \lambda} = 2^\lambda$ so $\kappa^\lambda = 2^\lambda$

What about $\kappa^\lambda$ when $\kappa > 2^\lambda$ ?

\emph{Note: } we will see that under GCH $\aleph_1^0 = \aleph_1, \aleph_5^0 = \aleph_5$
$\aleph_{\omega_1}^{\aleph_0} = \aleph_{\omega_1}$
but $\aleph_\omega^{\aleph_0} = \aleph_{\omega + 1}$.
So what is difference about $\aleph_\omega$ from the others?
$\aleph_\omega = \sup\set{\aleph_0, aleph_1, \dots , \aleph_n n \in \omega}$

\dfn For a limit ordinal $\gamma$, the \underline{cofinality} of $\gamma$, $cf(\gamma)$ is
$cf(\gamma) = \min\set{type(x) : x \subseteq \gamma \wedge \sup x = \gamma}$

\emph{Note: } $cf(\gamma) =\min\set{\xi \in ON : \text{there is a function} f : \xi \rightarrow \gamma \text{so that} ran(f) \text{ is unbounded in } \gamma}$

\emph{Note: } $cf(\gamma) \leq \gamma$ for any $\gamma$
Ex: $cf(\omega) = \omega$
Ex: $cf(\omega_1) = \omega_1$
Any countable subset of $\omega_1$ is bounded (supremum is countable)

Ex: $cf(\omega + \omega) = \omega$
$x = \set{\omega + n : n \in \omega}, type(x) = \omega$

Ex: $cf(\aleph_\omega) = \omega$
$x = \set{\aleph_n : n \in \omega}, type(x) = \omega$

\dfn $\gamma$ is \underline{regular} if $cf(\gamma) = \gamma$
A cardinal $\kappa$ is \underline{singular} if $cf(\kappa) < \kappa$

Ex: $\aleph_0, \aleph_1$ are both regular.
$\aleph_\omega$ is singular.

\textbf{Lemma}
(1) If $A \subseteq \gamma$ and $\sup(A) = \gamma$ then $cf(\gamma) = cf(type(A))$ (supA equals gamma means A unbounded in gamma?)
(2) $cf(cf(\gamma)) = cf(\gamma)$ i.e. $cf(\gamma)$ is regular
(3) $\omega \leq cd(\omega) \leq |\gamma| \leq \gamma$
(4) If $\gamma$ is regular then $\gamma$ is a cardinal, so $cf(\gamma)$ is a cardinal
(5) If $\gamma = \aleph_\alpha$ where either $\alpha = 0$ or $\alpha$ is a successor ordinal, then $\gamma$ is regular.
(6) If $\gamma = \aleph_\alpha$ where $\alpha$ is a limit ordinal then $cf(\aleph_\alpha) = cf(\alpha)$

\begin{proof}
(1) show $cf(\gamma) = cf(type(A))$
    Let $\alpha = type(A) \leq \gamma$.
    $\alpha$ is a limit ordinal since A is unbounded in $\gamma$.
    Let $f : \alpha \leftrightarrow A$ an isomorphism.
    show $cf(\gamma) \leq cf(\alpha)$.
    If $Y \subseteq \alpha$ is unbounded, then $f``Y$ is an unbounded subset of A of the same order type.
    Hence $f``Y$ is an unbounded subset of $Y$ of the same order type.
    Consider $Y$ unbounded in $alpha$ of order type $cf(\alpha)$ then $f``Y$ is unbounded subset of $\gamma$ of order type $=cf(\alpha)$
    So $cf(\gamma) \leq cf(\alpha)$
    Now show $cf(\alpha) \leq cf(\gamma)$
    Let $X \subseteq \gamma$ of order type $cf(\gamma)$ (unbounded in $\gamma$)
    For $\xi \in X$ let $h(\xi) = $ least element of $A \geq \xi$
    If $\xi < \eta$ then $h(\xi) \leq h(\eta)$
    Let $X` = \set{\eta \in X : \forall \xi \in X \cap \eta : h(\xi) < h(\eta)}$
    So $h \upharpoonright X` : X` \hookrightarrow A$ is order-preserving.
    $h``X`$ is unbounded in A and has order type $\alpha$ so $cf(\alpha) \leq type(X`) \leq type(X) \leq cf(\gamma)$

    \end{proof}

\section{March 6}
For a limit ordinal $\gamma$, the \underline{cofinality} of $\gamma$, $cf(\gamma)$ is
$cf(\gamma) = \min\set{type(x):x \subseteq \gamma and \sup x = \gamma}$

Ex: $cf(\aleph_0) = \omega$
$cf(\aleph_1) = \aleph_1$
$cf(\aleph_\omega)=\aleph_0$

Last time we proved:
(1) If $A \subseteq \gamma$ and $\sup(A) = \gamma$ then $cf(\gamma) = cf(type(A))$

Now
(2) $cf(cf(\gamma)) = cf(\gamma)$, so $cf(\gamma)$ is regular when $cf(\gamma) = \gamma$ (is a lmiit ordinal?)
pf:
    Let $A \subseteq \gamma$ be unbounded so $type(A) = cf(\gamma)$
    Applying (1) $cf(\gamma) = cf(type(A)) = cf(cf(\gamma))$


(3) $\omega \leq cf(\omega) \leq |\gamma| \leq \gamma$
\begin{proof}
    $\omega \leq cf(\omega)$ because if A is unbounded in $\gamma$ then type(A) can't be a successor ordinal.
    $|\gamma| \leq \gamma$ is immediate
    To show $cf(\gamma) \leq |\gamma|$:
    Let $\kappa = |\gamma|$ (because cardinality of an ordinal we know there is a bijection)
    Let $f: \kappa \twoheadrightarrow \gamma$ ber a surjection
    Define by recursion $g : \kappa \rightarrow ON$:
    $g(\eta) = \max\set{f(\eta), \sup\set{g(\xi) + 1 : \xi < \eta}}$
    So $g(\eta) \geq f(\eta)$
    If $\xi < \eta$ then $g(\xi) < g(eta)$
    So g is an isomorphism from $\kappa$ to $ran(g)$
    If $A = ran(g) \subseteq \gamma$ of order type $\kappa$ which is unbounded in $\gamma$ since $g(\eta) \geq f(\eta) \forall \eta$ and f is surjective.
    So $cf(\gamma) \leq type(A) = \kappa$
    If $ran(g) \nsubseteq \gamma$.
    So let $\eta$ be the least ordinal < $\kappa$ so $g(\eta) \geq \gamma$
    $\eta$ will be a limit ordinal, since if $g(\xi) < \gamma$ then $g(\xi) + 1 = \max(f(\xi+1), g(\xi)+1) < \gamma$
    So $g``\eta$ is unbounded subset of $\gamma$ of order type $\eta < \kappa$
    So $cf(\gamma) \leq \eta < \kappa$

\end{proof}

(4) If $\gamma$ is regular then $\gamma$ is a cardinal, so $cf(\gamma)$ is always a cardinal
pf:
    By (3) $\gamma = cf(\gamma) \leq |\gamma| \leq \gamma$ so $\gamma = |\gamma|$ so $\gamma$ is a cardinal

Recall $\aleph_0 = \omega, \aleph_{\xi + 1} = \aleph_\xi^+, \aleph_\lambda = \sup\set{\aleph_\xi : \xi < \lambda} \text{ for } \lambda$  is a limit ordinal

(5) If $\gamma = \aleph_\alpha$ where either $\alpha = 0$ or $\alpha$ is a successor ordinal, then $\gamma$ is regular
\begin{proof}
    Let $\alpha = \beta + 1$ be a successor ordinal.
    Suppose $A \subseteq \aleph_{beta+1} $ with $type(A) < \aleph_{\beta + 1}$
    Then $|A| \leq \aleph_\beta$
    $\sup(A) = \bigcup A$ is the union of $\leq \aleph_\beta$ sets each of cardinality $\leq \aleph_\beta$
    So $|\sup(A)| \leq \aleph_\beta < \aleph_{\beta+1}$
    So A is not unbounded in $\aleph_{\beta + 1}$
    So $cf(\aleph_{\beta + 1}) = \aleph_{\beta + 1}$
\end{proof}

(6) If $\gamma = \aleph_\alpha$ where $\alpha$ is a limit ordinal then $cf(\aleph_\alpha) = cf(\alpha)$
pf:
    Apply (1) to $A=\set{\aleph_\xi : \xi < \lambda}$
    So $cf(\aleph_\alpha) = cf(type(A)) = cf(\gamma)$


\textbf(Theorem) Let $\Theta$ be a cardinal
(1) If $\Theta$ is regular and $\mathbb{F}$ is a family of sets with $|\mathbb{F}| < \Theta$ and $|S|<\Theta$ for all $S \in \mathbb{F}$
then $|\bigcup \mathbb{F}| < \Theta$
(2) If $\Theta$ is singular (not regular) and $cf(\Theta) = \lambda < \Theta$ then there is a family $\mathbb{F}$ of sets with
$|\mathbb{F}| = \lambda$ and $|S| < \Theta$ for all $S \in \mathbb{F}$ with $\bigcup \mathbb{F} = \Theta$

\begin{proof}
(1) Let $X = \set{|S| : S \in \mathbb{F}} \subseteq \Theta$ and $|X| < \Theta$.
    So $type(X) < \Theta$ hence $\sup(X) < \Theta$ (type has same size as cardinality, then use that theta is regular)
    Let $\kappa=\max(\sup(x), |\mathbb{F}) < \Theta$.
    If $\kappa$ is finite then $\bigcup \mathbb{F}$ is as well.
    If $\kappa$ is infinite then $|\bigcup \mathbb{F}| \leq \kappa$
    So $|\bigcup \mathbb{F}| \leq \kappa < \Theta$ in both cases.
(2) Let $cf(\Theta) = \lambda < \Theta$.
    Let $\mathbb{F}$ be a subset of $\Theta$ with $type(\mathbb{F}) = \lambda$ and $\sup(\mathbb{F}) = \Theta$
    Then $|\mathbb{F}| = \lambda$ and $|S| < \Theta$ for all $S \in \mathbb{F}$
    and $\bigcup \mathbb{F} = \Theta$
\end{proof}

\textbf{Theorem (Koenig)}
If $\kappa \geq 2$ and $\lambda$ is infinite then $cf(\kappa^\lambda) > \lambda$
\underline(corr) $2^\kappa$ for any cardinal $\kappa$
since $cf(2^\kappa)>\kappa$ and $2^\kappa \geq cf(2^\kappa)$

\begin{proof}
    Let $\Theta = \kappa^\lambda$ (lambda is infinite so always infinte even when kappa is finite)
    Then $\Theta > \lambda$ by Cantor's Theorem since $\kappa^\lambda \geq 2^\lambda > \lambda$
    $\Theta^\lambda = (\kappa^\lambda)^\lambda = \kappa^{\lambda \cdot \lambda} = \kappa^\lambda = \Theta$
    Enumerate $\leftsuperscript{\lambda}{\Theta} = \set{f_\alpha : \alpha < \Theta}$
    If $cf(\Theta) \leq \lambda$ then we would have (2)
    $\Theta = \bigcup(\xi < \lambda) S_{\xi} with |S_\xi| < \Theta$.
    Can define $g: \lambda \rightarrow \Theta$ by (maybe error here? check slides)
    $g(\xi) = \min(\Theta \setminus \set{f_\alpha(\xi) : \alpha \in S_\xi})$
    (diagram in slides, uses diagonalization)
    (the minned set is not empty because...)
    Thenfor each $\alpha < \Theta$ $g(\xi) \neq f_\alpha(\xi)$where $\alpha \in S_\xi$
    Since $g(\xi) \neq f_\alpha(\xi)$ for all $\alpha \in S_\xi$
    Hence $g \neq f_\alpha$ for any $\alpha < \Theta$
    Contradiction that $\lambda_\Theta = \set{f_\alpha : \alpha < \Theta}$
    so $cf(\kappa^\lambda) > \lambda$

\end{proof}

\textbf{Cor} $\kappa^{cf(\kappa)}>\kappa$
if $\kappa^{cf(\kappa)}\geq cf(\kappa^{cf(\kappa)}) > cf(\kappa)$
So $\kappa^{cf(\kappa)} \neq \kappa$ in fact is >

Ex:
$\aleph_\omega ^ {\aleph_0} > \aleph_\omega$
$\aleph_2 ^ {\aleph_0} = \aleph_2$ under GCH
$\aleph_{\omega_1}^{\aleph_0} = \aleph_{\omega_1}$ under GCH


(missing stuff)

\textbf{Thm} Assuming GCH. Let $\kappa, \lambda$ be cardinal with $\max(\kappa,\lambda)$ infinite then
(1) If $2 \leq \kappa \leq \lambda^+$ then $\kappa^\lambda = \lambda^+$
(2) If $1 \leq \lambda \leq \kappa$ then $\kappa^\lambda =\begin{cases} \kappa & \text{if } cf(\kappa) > \lambda \\ \kappa^+ & \text{if } cf(\kappa) \leq \lambda \end{cases}$
Note If $\kappa = \lambda$ or $\kappa  = \lambda^+$ then
overlap $\kappa = \lambda$ (1) $\kappa^\lambda = \lambda^+$ (2) $\kappa^\lambda = \lambda+$
$\kappa=\lambda^+$ (1) $\kappa^\lambda = \lambda^+$ (2) $= \kappa = \lambda^+$

\begin{proof}
(1) $\kappa^\lambda > \lambda$ so $\kappa^\lambda \geq \lambda^+$
    $\kappa^\lambda \leq (\lambda^+)^\lambda = (2^\lambda)^\lambda = 2^\lambda = \lambda^+$
(2) $1 \leq \lambda \leq \kappa$
    $\kappa \leq \kappa^\lambda \leq \kappa^\kappa = 2^\kappa = \kappa^+$
    So $\kappa^\lambda$ is eihter $\kappa$ or $\kappa^+$
    If $cf(\kappa) \leq \lambda$ then $\kappa^\lambda \geq \kappa^{cf(\kappa)} > \kappa$ so $\kappa^\lambda = \kappa^+$
    If $cf(\kappa) > \lambda$ then every $f: \lambda \rightarrow \kappa$ is bounded
    $\leftsuperscript{\lambda}{\kappa} = \bigcup(\alpha<\kappa) \leftsuperscript{lambda}{\alpha}$
    $\leftsuperscript{\lambda}{\alpha} \subseteq \PP(\lambda \times \alpha)$
    $|\lambda \times \alpha| < \kappa$
    So $|\PP(\lambda \times \alpha)| \leq \kappa$ by GCH
    So $|\leftsuperscript{\lambda}{\alpha}| \leq \kappa$ for $\alpha < \kappa$
    $|\leftsuperscript{\lambda}{\kappa} \leq \kappa$ so $= \kappa$
\end{proof}

Ex: $\aleph_\omega^{aleph_{omega}}$ under GCH
$\kappa = \lambda$
$=\aleph_\omega^+=\aleph_{\omega+1}$
(check left super is correct)
    %! Author = nathanaelsteven
%! Date = 3/11/25

\section{Mar 11}

Recall: GCH: $2^k = k^+$ for al infinite cardinals $k$ i.e. $2^{\aleph_{alpha}} = \aleph_{\alpha+1}$ for all $\alpha$
If we assume GCH we can show any $\kappa^\lambda$, partly thanks to knowing cofinality.

\textbf{Theorem:} Assuming GCH:
(1) If $2 \leq \kappa \leq \lambda^+$ then $\kappa^\lambda = \lambda^+$
(2) If $1 \leq \lambda \leq \kappa$ then $\kappa^\lambda = \begin{cases} \kappa & \text{if } $\lambda < cf(\kappa)$ \\ \kappa^+ & \text{if } $\lambda \geq cf(\kappa)$ \end{cases}$

\underline{Ex: } Assuming GCH:
$\aleph_1^{\aleph_0}, \aleph_1 \leq \aleph_0^+$ so $= \aleph_0^+ = \aleph_1$ by case 1
and $\aleph_0 \leq \aleph_1, \aleph_0 \leq cf(\aleph_1) =aleph_1$ so $=\aleph_1$ by case 2

$\aleph_{\omega_1}^{\aleph_0} = \aleph_{\omega_1}$

$\aleph_\omega^{\aleph_0}, \aleph_0 \geq cf(\aleph_{\omega_1})$ so $=\aleph_\omega^+ = \aleph_{\omega+1}$

$\aleph_\omega^{\aleph_1}$ (case 2) $\aleph_1 > cf(\aleph_\omega) = \aleph_0$ sp $=\aleph_{\omega+1}$

$\aleph_{\omega}^{\aleph_\omega}$ (both) $= \aleph_{\omega+1}$

$\aleph_\omega^{\aleph_{\omega+1}} = \aleph_{\omega+1}^+ = \aleph_{\omega+2}$

$\aleph_{\omega+1}^{\aleph_\omega} = \aleph_\omega^+ = \aleph_{\omega+1}$

Recall: defined aleph sequnce by recursion, enumeration of all infinite cardinals
$\aleph_0 = \omega, \aleph_{\alpha+1} = \aleph_\alpha^+, \aleph_\lambda = \sup \set{\aleph_\alpha : \alpha < \lambda}$ for $\lambda$ a limit ordinal.

\dfn \underline{The beth sequence} is defined by recursion on the ordinals.
$\beth_0 = \aleph_0 = \omega, \beth_{\alpha+1} = 2^{\beth_\alpha}, \beth_\lambda = \sup \set{\beth_\alpha : \alpha < \lambda}$ for $\lambda$ a limit ordinal.

\emph{Note } $\aleph_\alpha \leq \beth_\alpha$ for all $\alpha$
GCH is equivalent to $\aleph_\alpha = \beth_\alpha$ for all $\alpha$

Recall: $\kappa$, a cardinal (or a limit ordinal), is regular if $cf(\kappa) = \kappa$ otherwise singular.

$\aleph_0$ and $\aleph_{\alpha+1}$ are all regular.
$cf(\aleph_\omega) = \aleph_0$ so $\aleph_\omega$ is singular
$cf(\aleph_{\omega+1}) = cf(\omega_1) = \aleph_1 < \aleph_{\omega_1})$ so $\aleph_{\omega_1}$ is singular.

Are there limit ordinal so $\aleph_\lambda$ is regular, meaning $\aleph_\lambda = cf(\aleph_\lambda) = cf(\lambda) \leq \lambda$
Always have $\aleph_\lambda \geq \lambda$, so need $\aleph_\lambda = \lambda$.
!fixed point we can find them

\underline{Ex:} $\kappa_0 = \aleph_0$, $\kappa_{n+1} = \aleph_{\kappa_n}$
$\aleph_0, \aleph_{\aleph_0}, \dots$ (see slides) aleph sub aleph sub alpeh sub 0
Let $\kappa = \sup (\set{\kappa_n : n \in \omega})$
$\aleph_\kappa = \sup \set{ \aleph_\alpha : \alpha < \kappa} = \sup \set{ \aleph_{\kappa_n}: n \in \omega}$
$= \sup \set{\kappa{n+1}: n \in \omega}= \sup \set{\kappa_n  n \in \omega} = \kappa$ but $\kappa$ is not regular

$cf(\aleph_\kappa) = cf(\kappa) = \omega = \aleph_0$ since $\set{\kappa_n : n \in \omega}$ is unbounded in $\kappa$
so $\kappa$ is singular

\dfn A cardinal $\kappa$ is a \underline{limit cardinal} if $\lambda^+ < \kappa$ for all $\lambda < \kappa$
(all infinite cardinals have to be limit ordinals, but not all limit ordinals are a cardinal)
equivalently $\kappa = \aleph_\lambda$ for $\lambda$ a limit ordinal.

\dfn A cardinal $\kappa$ is a \underline{strong limit cardinal} if $2^\lambda < \kappa$ for all $\lambda < \kappa$
(2 to the lamdba is at least as big as lambda plus: strong implies limit)
Under GCH strong limits are the same as limit cardinals.

\dfn A cardinal $\kappa > \aleph_0$ ($\aleph_0$ is a limit cardinal?) is a \underline{weakly inaccessible cardinal} if $\kappa$ is a regular limit cardinal.
(inacc because you cant reach it from below, regular because cant reach from successor and something about making it from subsets?)
A cardinal $\kappa > \aleph_0$ is \underline{strongly inaccessible} if $\kappa$ is a regular strong limit cardinal.
(cant reach from belpw, even if we are allowed to use power sets)
Under GCH these are the same.

\underline{Fact :} ZFC does not prove that there are weakly inaccessible cardinals.

\dfn For any set $A$ we define $[A]^\kappa = \set{x \subseteq A : |x| = \kappa}$
and $[A]^{\subset \kappa} = \set{x \subseteq A : |x| < \kappa}$

\subsection{Axiom of Foundation}
(epsilon relation is well founded on all sets, previously only on ordinals)
\dfn \underline{Axiom of Foundation}
$\forall x [\exists y (y \in x) \implies \exists y (y \in x \wedge \neg x (z \in x \wedge z \in y))]$
i.e. $x \neq \emptyset \implies \exists y \in x (y \cap x = \emptyset)$
i.e. $x \neq \emptyset \implies$ x contains an $\in$-minimal element.

\emph{Note: } Axiom of Foundation is not necessary for the development of ``ordinal`` mathematics.
All explicitly defined sets we use will be well-founded with respect to $\in$

\emph{Note: } Foundation rules out ``pathological`` sets like $x = \set{x}$
We could consider the class of ``well-founded sets`` even wihtout Foundation.
Foundation says all sets can be built iteratively from $\emptyset$ and taking collections of subsets.

Allows us to assign ordinal ranks to sets.

\underline{Ex: } Foundation $\implies \neg \exists x (x \in x)$
Suppose there were a set where it is an element of itself.
Consider $\set{x} \neq \emptyset$
Foundation $\implies \exists y \in \set{x}, y \cap \set{x} = \emptyset$
Need $y=x$ as it is theonly elements, but $x \cup \set{x} = x$ since $x \in x$ and $x \neq \emptyset$ contradicting $y \cap x = \emptyset$

\underline{Ex: } Similarly, Foundation $\implies$ no infinite descending $\in$-chains or cycles.
No $x_0 \in x_1 \in \dots \in x_n \in x_0$ or $x_{n+1} = x_n $ for all $n \in \omega$

\emph{Note: } diagram in slides
Foundation $\implies$ every branch through the tree terminates at the $\emptyset$

\underline{Ex: } $\set{\emptyset, \set{\set{\emptyset}}, \set{\emptyset,\set{\emptyset}}}$
draw listing out elements of the sets \dots
(can be infinite, both branches and depth)
\underline{Ex: } $x_n = \set{\set{\dots \set{\emptyset} \dots}}$ n times $\set{x_n \in \omega}$ diagram in slides.
have a brnach of length n for each element.
all arbitrarily long branches aree finite (not infinite) allowed by foundation.

\dfn \underline{Cumulative Hierarchy}
Define collection of sets $R(\alpha)$ by recursion on $\alpha \in ON$
$R(0) = \emptyset$
$R(\alpha+1) = \PP(R(\alpha))$
$R(\lambda) = \bigccup \set{R(\alpha) : \alpha < \lambda}$ for $\lambda$ a limit ordinal

et $WF =  \bigccup(\alpha \in ON) R(\alpha)$

Note $WF$ is not a set, but defined by a formula $x \in w^F \equiv \exists \alpha x \in R(\alpha)$

\dfn Say x is \underline{well-founded} if $x \in WF$

\underline{Ex: }
$R(0) = \emptyset$
$R(1) = \PP(\emptyset) = \set{\emptyset}$
$R(2) = \set{\emptyset,\set{\emptyset}}$
$R(3) = \set{\emptyset, \set{\emptyset}, \set{\set{\emptyset}},\set{\emptyset,\set{\emptyset}}} $

Note $|R(n)| = 2^{n-1}$ for $n \geq 1$
$R(\alpha+1) = 2^{|R(\alpha)|}$

Note $R(\omega) = \bigcup(n \in omega) = R(n)$
$|R(\omega)| = \aleph_0$
$|R(\omega+1)| = 2^{\aleph_0}$
$|R(\omega + \alpha) = \beth_\alpha$

\emph{Note: } $0 = \emptyset \in R(1), 1 = \set{\emptyset} \in R(2)$
Each $n \in \omega$ has $n \in R(\omega)$ but $\omega \notin R(\omega)$ but $\omega \subseteq R(\omega)$ so $\omega \in R(\omega+1) = \PP(R(\omega))$

\emph{Note: } Call $R(\omega) = HF = $ Hereditarily Finite sets.

\dfn For $x \in WF$ the \underline{rank} of x is $rank(x) =$ the least $\alpha$ so that $x \in R(\alpha + 1)$

\emph{Note: } $x \in R(0) = \emptyset$
If $x \in R(\lambda)$ for a limit ordinal then $x \in R(\alpha)$ for some $\alpha < \lambda$.
So th4e least $\delta$ where $x \in R(\delta)$ must be $\delta = \alpha+1$ for some $\alpha$

\underline{\textbf{Lemma}}
(1) Every $R(\alpha)$ is a transitive set
(2) If $\alpha \leq R$ then $R(\alpha) \subseteq R(\beta)$
(3) $R(\alpha+1) \setminus R(\alpha) = \set{x \in WF : rank(x) = \alpha}$
(4) $R(\alpha) = \set{x \in WF : rank(x) < \alpha}$
(5) If $x \in y$ and $y \in WF$ then $x \in WF$ and $rank(x) < rank(y)$


\begin{proof}
(1) Every $R(\alpha)$ is a transitive set.
    By transfinite induction on $\alpha$.
    $R(0) = \emptyset$ is transitive.
    $R(\lambda)$ immediate as the union of a collection of transitive sets is transitive.
    Suppose $R(\alpha)$ is transitive then show $R(\alpha+1) = \PP(R(\alpha))$ is transitive.
    $R(\alpha) \subseteq R(\alpha)$ since if $x \in R(\alpha)$ then $x \subseteq R(\alpha)$ since $R(\alpha)$ is transitive.
    So $x \in \PP(\R(\alpha)) = R(\alpha + 1)$.
    So if $y \in R(\alpha+1)$ then $y \subseteq R(\alpha) \subseteq R(\alpha+1)$ so its transitive.
(2) If $\alpha \leq \beta$ then $R(\alpha) \subseteq R(\beta)$.
    Fix $\alpha$ and prove by induction on $\beta \geq \alpha$.
    Assume $R(\alpha) \subseteq R(\beta)$.
    From (1) we have $R(\beta) \subseteq R(\beta+1)$
    So $R(\alpha) \subseteq R(\beta) \subseteq R(\beta+1)$.
    If $R(\alpha) \subseteq R(\beta)$ all $\alpha \leq \beta < \lambda$ ($\lambda$ a limit ordinal)
    then $R(\alpha) \leq \bigcup(\beta < \lambda) R(\beta) = R(\lambda)$
(3) $R(\alpha+1)\setminus R(\alpha) = \set{x \in WF : rank(x) = \alpha}$
    For $x \in WF$ $rank(x)=$ least $\alpha$ so that $x \in R(\alpha+1)$.
    Sp $x \in R(\alpha + 1) \wedge x \notin R(\alpha) \leftrightarrow rank(x) = \alpha$.
(4) $R(\alpha) = \set{x \in WF : rank(x) < \alpha}$.
    Immediate from (2) and (3)
%    $x \in R(\alpha) \leftrightarrow x \in R(\beta)$ for some $\beta < \alpha$.
%    $x \in WF \leftrightarrow \exists \beta < \alpha x \in R(\beta)$
(5) If $x \in y$ and $y \in WF$ then $x \in WF$ and $rank(x) < rank(y)$.
    Let $\alpha = rank(y)$ so $y \in R(\alpha+1)$ so $y \subseteq R(\alpha)$.
    So $x \in R(\alpha)$ so $rank(x) < \alpha = rank(y)$.
    $y \in WF \implies \exists \alpha y \in R(\alpha)$
    $x \in y \subseteq R(\alpha)$ so $x \in WF$ and $rank(x) \leq \alpha$.
    If $rank(x) = \alpha$ then $x \in R(\alpha+1) = \PP(R(\alpha))$ so $x \subseteq R(\alpha)$
    So $x \in R(\alpha)$ so $rank(x) < \alpha$
    \end{proof}

   \underline{Lemma}
   (1) $ON \cap R(\alpha) = \alpha$ for all $\alpha \in ON$.
(2) $ON \subseteq WF$
(3) $rank(\alpha) = \alpha$ for $\alpha \in ON$

\begin{proof}
(1) by induction on $\alpha$.
    $R(0) = \emptyset = 0$.
    so $ON \cap R(0) = \emptyset = 0$
    For $\lambda$ a limit.
    $ON \cap R(\lambda) = \bigcup_{\alpha < \lambda} ON \cap R(\alpha) = \bigcup(\alpha < \lambda) \alpha = \lambda$
    Suppose $ON \cap R(\alpha) = \alpha$
    $\alpha \subseteq R(\alpha) \subseteq R(\alpha+1)$
    $\alpha \in \PP(\R(\alpha)) = R(\alpha+1)$.
    So $\alpha + 1 = \alpha \cup \set{\alpha} = \subseteq R(\alpha+1)$.
    Suppose $\delta \in ON \cap R(\alpha+1)$
    $\delta \subseteq R(\alpha) \cap ON = \alpha$
    So $\delta \leq \alpha$ so $\delta \in \alpha + 1$.
    So $R(\alpha + 1) \cap ON = \alpha + 1$.
(2) Each $\alpha \subseteq R(\alpha)$ so $\alpha \in R(\alpha + 1)$
    so $\alpha \in WF$
(3) $\alpha \in R(\alpha+1) \setminus \R(\alpha)$ iff $\alpha \subseteq R(\alpha)$.
    So $rank(\alpha) = \alpha$
    \end{proof}

diagram in slides? potentially

all sets in $R(\omega)$ are finite.
We see infinite sets in $R(\omega+1)$ as $\omega$ is in it, in fact all subsets of $\omega$.
each real number can be represented at $R(\omega+1)$
and at +2 we have all the subsets of $\RR$.
infinite descending chains and self-referencing sets are not in $WF$ (without foundation).

\underline{Lemma}
(1) $y \in WF \leftrightarrow y \subseteq WF$
(2) If $y \in WF$ then $rank(y) = \sup\set{\rank(x) + 1: x \in y}$

\begin{proof}
(1) Suppose $y \in WF$ then $y \in R(\alpha)$ so $y \subseteq R(\alpha)$ so $y \subseteq WF$
    Suppose $y \subseteq WF$.
    Let $\beta = \sup\set{rank(x) +1: x \in y}$ (replacement tells us it is a set).
    Notice $y \subseteq R(\beta)$ so $y \in R(\beta + 1)$ so $y \in WF$.
    Since each $x \in y$ has $x \in R(\rank(x) + 1)$
(2) From last step $rank(y) \leq \sup\set{\rank(x)+1 : x \in y }$.
    If $x \in y  $ then $rank(x) < rank(y)$
    So $rank(y) \geq rank(x) + 1 \forall x \in y$.
    So $rank(y) \geq \sup\set{\rank(x) + 1 : x \in y}$.
    \end{proof}

\underline{Ex:} $x = \set{1,3}$
$rank(x) = \sup\set{rank(1) + 1, rank(3) + 1}=4$  (rank + 1 = rank, and sup is +1 of that? so 2,4 so 4)
$rank(\langle 1,3 \rangle) = rank(\set{\set{1},\set{3}, \set{1,3}})$
$=\sup\set{rank(\set{1}),...}=5$ (rank of 1,3 is 4) so sup = 5

\underline{Lemma: } If $z \subseteq y$ and $y \in WF$
then $z \in WF$ and $rank(z) \leq rank(y)$

pf: if $y \in WF$ then it is in some $R(\alpha+1)$ where $\alpha = rank(y)$ and $y \subseteq R(\alpha)$ so $z \subseteq R(\alpha)$
So $z \in R(\alpha+1)$ and $rank(z) \leq \alpha = rank(y)$.


\underline{Lemma:} Let $x,y \in WF$
(1) $\set{x,y} \in WF$ and $rank(\set{x,y}) = \max(rank(x), rank(y)) + 1$
(2) $\langle x,y \rangle$ and its rank $= \max(rank(x),rank(y)) + 2$
(3) $\PP(x) \in WF$ and $rank(\PP(x)) = rank(x) + 1$
(4) $\bigcup x \in WF$ and $rank(\cup x) \leq rank(x)$
(5) $x \cup y \in WF$ and $rank(x \cup y) = \max(rank(x),rank(y))$
(6) $trcl(x) \in WF$ and $rank(TC(x)) = rank(x)$

$TC(x) : \bigcup^0 x = x, \bigcup^{n+1} x = \bigcup (\bigcup^n x), trcl(x) = \bigcup_{n \in \omega} (\bigcup^n x) $

pf: immediate from $rank(\omega) = \sup\set{rank(\omega) + 1: \omega \in x} $ (? rank of x or omega for initial one)

\dfn $ZF^-$ concsits of all Axioms of ZF except foundation.
\dfn V is all sets?

\underline{Theorem:} In $ZF^-$ the Axiom of Foundation is equivalent to the statement the ``$V = WF$``
i.e. $\forall x \exists \alpha x \in R(\alpha)$

\begin{proof}
    $\impliedby$ Suppose $\forall x \exists \alpha x \in R(\alpha)$.
    If $x \neq \emptyset$ then let $y \in x$ has least rank among all elements of $x$.
    Then $y$ si an $\in$-minimal element of $x$.
    no elements of $y$ which are elements of $x$ since they would have lower rank.
    So Axiom of Foundation holds.
    $\implies$ Assume Axiom of Foundation.
    Fix $x$ and show $x \in WF$.
    let $t = TC(x)$ so $x \subseteq t$
    If $t \susbeteq WF$ then $x \in t \in WF$ so $\x in WF$
    Suppose $t \nsubseteq WF$.
    $t \setminus WF$ would be non-empty.
    So by Foundation we can find an $\in$-minimal element $y$.
    Every element of $y$ is an element of $t$ since $t$ is transitive.
    But no element of $y$ is in $t \setminus WF$.
    So everey element of $y$ is in $WF$ so $y \subseteq WF$ so $y \in WF$.
    Contradicting our choice.
    \end{proof}





%! Author = nathanaelsteven
%! Date = 4/1/25

\documentclass[11pt]{amsart}
\usepackage{amsmath}
\usepackage{amssymb}
\usepackage{amsthm}

\begin{document}

    Missing definitions of lexicon, atomic formula, terms, structure, etcetera.

    closure with quantifiers? universal clsoures
    \\ \\
    \newcommand{\FF}{\mathcal{F}}
    \newcommand{\LL}{\mathcal{L}}
    \newcommand{\TT}{\mathcal{T}}

    \underline{terms} built from $\FF \cup VAR$
    \underline{atomic formulas} $P^{\TT_0 ...}$ or $= \TT_1 \TT_2$
    \underline{formulas} use connectives, quantifiers, etc.


    - $V(\TT)$ is the set of variables which occur in $\TT$.
    - $\varphi$, let $V(\varphi)$ be the set of variables which have a free occurrence $\varphi$.
    \\
    For an $\LL$-structure $\mathfrak{A}=(A,I)$, a term $\TT$, and an assignment $\sigma$ for $\TT$ in $A$, define $val_{\mathfrak{A}(\TT)[\sigma]}$
    \\
    $\LL = \{+,\cdot\},\{<\}$ \\
    $\mathfrak{A}: A = \mathbb{R}$ \\
    $\TT: \cdot z + x y$ \\
    $V(\TT) = \{x,y,z\}$ \\
    $\sigma: [x y z][3 4 5]$ \\
    $val_\mathfrak{A}(x)[\sigma]=3$ \\

    For an $\LL$-structure $\mathfrak{A}$, an atomic formula $\varphi$, and an assignment $\sigma$ for $\varphi$ in $A$, define $val_{\mathfrak{A}(\varphi)[\sigma]}$ in $\{F,T\}=\{0,1\}$ by \\
    $val_\mathfrak{A}(P)[\sigma] = P_\mathfrak{A}$ if $P \in P_0$
    etcetera

    $\mathfrak{A}$ satisfies $\varphi$, $\mathfrak{A} \models \varphi$ when $V(\varphi) = \emptyset$?
    \\
    A set $\Sigma$ of $\LL$-sentences is semantically consistent or satisfiable $con_\models(\Sigma)$, if there is an $\LL$-structure $\mathfrak{A}$ such that $\mathfrak{A} \models \Sigma$
    \\
    Reductio ad Absurdum: Let $\Sigma$ be a set of $\LL$-sentences and $\Psi$ an $\LL$-sentence then:
    (1) $\Sigma \models \Psi \iff \Sigma \cup \{\neg \Psi\}$ is semantically inconsistent.

    A formula is logically valid if a structure $\mathfrak{A}$ satisfies it for $\sigma$ or all strucutre and assignment for psi in the structure
    if two formulas are both universal closures of a formula then they are logically equivalent (irrelevant what assignments are used)
    formula are equivalent with respect to a set of logical sentences if the universal closure is true in all models of the sentences.
    two terms are equivalent with respect ot logical sentences if for all structures that satisfy the sentences and all assignment for the two terms in the structure, the two terms have the same value.

    \\ axiom of choise and zorn's lemma are equivalenet with respect to ZF.
    supppse sigma includes associativity and the two terms are the sides of the equivalence, then they are equivalent with respect to the set of sentences.
    \\
    let beta and tau be terms and x a variable.
    then beta of x to tau, or beta of x / tau, is the result of replcing all occurences of x in beta with tau.
    \\
    let psi be a formula, x a variable, tau a term, then psi x assigned to tau, is the resutl of replacing all free occurences of x by tau in psi.
    example problem is if exists y where x < y, and we substitute in y for x we get a potential problem.
    if we tried ot replace y itself nothing would change because it isn't free, it has a quantifier.
    \\
    a term tau is free for x in a formula phi if no free occurrence of x is inside the scope of a quantifier exists y or forall y, where y is a variable in tau.
    tau is y for above is not free for x because there is an occurence of x inside the scope of a quantifier applied to y.
    If tau is free for x in phi, then the following are logically valid: forall x, phi(x) implies phi(tau), phi(tau) implies exists x, phi(x).
    note that the above example that was not free for x fails.
    \\
    a reduct of the originsl strucutre is a restriction to the symbols in some other structure that is a substructure of the original and vice verse an expansion.
    \\
    isomporhpisms and shit.
    \\
    a set of axiom (L-sentences) sigma is complete with respect to the strucutre(L) if sigma is semantically consistent and for all L-sentences, phi, either sigma logically implies(models/satisfies) phi or not phi.
    \\
    ZFC is not complete.
    there are some natural complete theories.
    any `sufficiently complicated` (able to interpret arithmetic) and `nicely axiomatized` (finite list or sufficiently concrete) set of axioms will not be complete.
    \\
    let a be an l-structure.
    the theory of a, Th(a) is the set of all l-sentences true in a.
    Th(a) is complete.
    for any formula(l-sentence) phi exactly one : a satisfies phi or a satisfies not phi.
    \\
    \subsection{Tautologies}
    a basic formula is one which (when expressed in polish/prefix notation) does not begin with a propositional connective. \\
    ex: forall x P(x): forall x P x, or forall x (P(x) then q(x)) : forall x implies p x q x, are both basic\\
    ex (otoh): p(x) implies q(x) : implies p x q x is not basic.
    \\ A truth assignment for L is a function wich assigns a truth value to every basic formula.
    given such a function, we can extend it to all formulas by recursion.
    note that we don't require consistency so truth assignments can be semantically inconsistent (no structures or meaning to symbols).
    \\ a formula phi is a propositional tautology if it is true under every truth assignment.

\end{document}

\end{document}

