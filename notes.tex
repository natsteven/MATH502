\documentclass{article}

% Language setting
% Replace `english' with e.g. `spanish' to change the document language
\usepackage[english]{babel}

% Set page size and margins
% Replace `letterpaper' with `a4paper' for UK/EU standard size
\usepackage[letterpaper,top=2cm,bottom=2cm,left=3cm,right=3cm,marginparwidth=1.75cm]{geometry}

% Useful packages
\usepackage{amsmath}
\usepackage{graphicx}
\usepackage[colorlinks=true, allcolors=blue]{hyperref}
\usepackage{amsfonts}
\usepackage{amssymb}
\usepackage{upgreek}
\usepackage{amsthm}

\title{Set Theory and Logic Notes}
\author{Nat}

\begin{document}
    \maketitle

    \section{Introduction}

% Group theory. Group is a set of objects .

% G - a set

% . - binary function

% -1 - inverse function

% 1 - identity

% Axioms of group theory:

% For all a,b,c in G

% associative - a.b. . c =

    We can abbreviate statements, e.g. $\neg (x=y)$ and $x \neq y$
    there exists a unique x satisfying $P(x) \rightarrow \exists! x P(x)$
    expansion: $\exists x (P(x) \and \forall y (P(y) \rightarrow y=x))$
    note we can give a tree which shows how a formula is constructed.
    The connectives and quanitifers are the internal nodes and root, terminals are predicates/relations.

    Connectives are functions on truth values, can be described using truth tables.
    Examples:

    \begin{tabular}{c|c}
        P  &  $\neg$ P\\
        \hline
        T  & F \\
        F & T
    \end{tabular}

    \vspace{1em}

    \begin{tabular}{c|c|c}
        P & Q & P $\wedge$ Q\\
        \hline
        T & T & T \\
        T & F & F \\
        F & T & F \\
        F & F & F \\
    \end{tabular}

    \vspace{1em}

    \begin{tabular}{c|c|c}
        P & Q & P $\implies$ Q\\
        \hline
        T & T & T \\
        T & F & F \\
        F & T & T \\
        F & F & T \\
    \end{tabular}

    \vspace{1em}
    \begin{tabular}{c|c|c}
        P & Q & P $\iff$ Q\\
        \hline
        T & T & T \\
        T & F & F \\
        F & T & F \\
        F & F & T \\
    \end{tabular}

    \vspace{1em}
    \textbf{Logical Equivalence}
    \begin{itemize}
        \item Two statements are logically equivalent if for any truth assignments of truth values to the predicates, both statements have the same resulting truth value. $s_1 \equiv s_2$
    \end{itemize}

    \emph{Ex}: $P \implies Q$ is logically equivalent to $(\neg P ) \vee Q$

    \begin{tabular}{c|c|c}
        P & Q & $(\neg P) \vee Q$\\
        \hline
        T & T & T \\
        T & F & F \\
        F & T & T \\
        F & F & T \\
    \end{tabular}
    \vspace{1em}

    \textbf{Def:} A statement is \emph{valid} if it is true for any truth assignment to the predicates
    \emph{Ex:} $P \vee \neg P$


    \section{Set Theory}
    \begin{itemize}
        \item First to study this was Cantor
        \item \"size\" of infinite sets
        \item two sets have the same size if we can put a bijection between the two
        \item properties of sets.
    \end{itemize}

    \underline{Idea:} Sets are determined by their members. $\{1,2,3,4\} \equiv \{2,3,1,4\}$

    \underline{Goals}
    \begin{itemize}
        \item Formalize Sets
        \item Introduce Axioms of Zermelo-Frankel Set Theory with choice (ZFC)
        \item Cardinality
        \item represent standard mathematical objects as sets
        \item ZFC as a foundation for all of mathematics
    \end{itemize}

    \underline{Note}
    \begin{itemize}
        \item Everything will be a set
        \item Language of set theory will be built from two predicates!
        \begin{itemize}
            \item = : equality
            \item $\in$ : set membership. $x\in y$ will be true when x is an element of y
        \end{itemize}
    \end{itemize}

    \emph{Ex:} $\emptyset$ , empty set. $\{\emptyset\} \neq \emptyset$. Then we can build natural numbers. 2: $\{\emptyset,\{\emptyset\}\}$

    \vspace{1em}
    \noindent\textbf{\emph{The Axioms of ZFC}}

    \underline{Language} built from =, $\in$
    \begin{itemize}
        \item Axiom 0: \emph{set existence}, There is a set. $\exists x (x=x)$. (axiom 0 will be redundant later).
        \item Axiom 1: \emph{extensionality}, Sets are determined by their elements
        \item Axiom 2: \emph{Foundation} There are no infinite descending chains of sets with respect to elementhood. i.e. No $..... \in x_2 \in x_1 \in x_0$. implies no element is an element of itself
        \item Axiom 3: \emph{comprehension scheme}
        \begin{itemize}
            \item For every formula $\varphi(x)$ and every set $Z$ we can form the set $\{x \in Z : \varphi(x)\}$.
            Those x which are elements of $Z$ which satisfy $\varphi(x)$
            \item one axiom for each formula $\varphi(x)$
        \end{itemize}
        \item Axiom 4: \emph{pairing}, Given sets $x$ and $y$, there is a set containing both $x$ and $y$
        \item Axiom 5: \emph{union}, Given a family of sets, $F$, there is a set containing the union of all of the sets in $F$
        \item Axiom 6: \emph{replacement scheme}, Given a formula $\varphi$ which defines a function $f$ and a set $Z$, there is a set which consists of the range of $f$ applied to $Z$
        \item Axiom 7: \emph{infinity} (makes 0 redundant), there is an infinite set
        \item Axiom 8: \emph{power set}, given a set $X$ there is a set $\mathcal{P}(X)$ which contains as elements all of the subsets of $X$
        \item Axiom 9: \emph{axiom of choice (AC)}, given a collection of disjoint non-empty sets, there is a set which contains exactly one element from each set in the collection.
        \begin{itemize}
            \item \emph{Ex}: $\mathbb{R}$ $x~y$ if $x-y \in \mathbb{Q}$

            $[x] = \{y: y~x\}$

            $\mathcal{e}=\{[x]: x \in \mathbb{R} \}$

            if x not ~ y then [x] and [y] are disjoint$ [x] \neq \emptyset$
            \item AC says there is a set containing exactly one element of each equivalence set
            \item this is called a \emph{Vitali} set.
        \end{itemize}

        \underline{Note:}
        \begin{itemize}
            \item ZFC denotes Axioms 1-8
            \item ZF denotes Axioms 1-8
            \item ZC and Z denote ZFC and ZF with replacement scheme removed
            \item Z-, ZF-, ZC-, ZFC- denote removing foundation axiom
            \item \"most\" standard mathematics can be don in ZC-
        \end{itemize}

    \end{itemize}


    \subsection{Axiom 0 - Set Existence}
    \subsection{Axiom 1 - Extensionality}
    \subsection{Axiom 2 - Foundations}
    \subsection{Axiom 3 - Comprehension Scheme}
    \subsection{Axiom 4 - Pairing}

    \textbf{Recall} - Ordered Pair: $<x,y> = \{\{x,y\},\{x\}\}$

    \subsection{Axiom 5 - Union}
    \[\forall \mathcal{F} \exists A \forall Y \forall x ((x \in Y \wedge Y \in mathcal{F}) \implies x \in A
    \]
    \underline{Ex} $Y_1, Y_2, \dots, Y_k \in \mathcal{F}$

    $A \supseteq Y_1 \cup Y_2 \dots \bigcup Y_k$

    \vspace{1em}
    \underline{Note}: Comprehension allows us to form the et whose elements are exactly those x so that $x \in Y$ for some $Y \in \mathcal{F}$. Denote this set $\bigcup \mathcal{F}$, i.e. $Y^\cup \in \mathcal{F}^Y$

    \underline{Ex:} $x \cup y = \bigcup \{x,y\} =\{z : z \in x \cup z \in y\}$

    $x \cup y \cup z = \big \{x,y,z\}$

    \vspace{1em}
    \underline{Ex:} $\bigcup\{x\} = \{z : z \in x\} = x$

    $ \bigcup\{\{x\}\} = \{x\}$ and $\bigcup \bigcup\{\{x\}\} = x$

    \vspace{1em}
    \underline{Def}: When $\mathcal{F} \neq \emptyset$ we can define intersection. $\bigcap \mathcal{F} = \bigcup_{Y \in \mathcal{F}} Y$

    $=\{x : \forall Y \in \mathcal{F} (x \in y)\}$ (naive)

    $\{x\in Y_0 : \forall Y \in \mathcal{F}(x \in y)\}$ (proper use of comprehension.)

    where $Y_0$ is some element of $\mathcal{F}$

    \underline{Note} If $\mathcal{F} = \emptyset$ then $\bigcap \mathcal{F} would be a universal set$

    \vspace{1em}
    \underline{Def:} The \emph{ordinal successor function}

    $S(x) = x \cup \{x\}$

    \underline{Ex:} $S(\emptyset) = \emptyset \cup \{\emptyset\} = \{\emptyset\}$ and $S(\{\emptyset\}) = \{\emptyset\} \cup \{\{\emptyset\}\} = \{\emptyset,\{\emptyset\}\}$

    $0 = \emptyset$

    $1 = S(0)$

    $2 = S(1) = S(S(0))$

    $n+1 = S(n)$

    \underline{Note} $n = \{ k : k < n\}$, n has exactly n elements. $n \leq m iff n \subseteq m$ and $n < m iff n \in m$

    \underline{Informally} a natural number is a set obtained by some finite number of iteration of the successor applied to $\emptyset$


    \subsection{Relations and Functions}

    \underline{Def} R is a (binary) relation if R is a set of ordered pairs, i.e. $\forall z \in R \exists x \exists y z = <x,y>$

    \underline{Note} If $\phi(z) is a formula the abbreviation \forall z \in R \phi(z) and \exists z \in R \phi(z)$ means

    $\forall Z \in R \phi(z) means \forall z (z \in R \implies \phi(z))$

    $\exists z \in R \phi(z) means \exists z (z \in R \and \phi(z))$

    \vspace{1em}
    \underline{Def} $dom(R) =^* \{x : \exists y <x,y> \in R\}$

    $ran(R) =^* \{y: \exists x <x,y> \in R\}$

    * need to see that there is a set $Z$ so that all $x$ and $y$ needed in these definitions are elements of $Z$

    If $<x,y> \in R$

    $\{\{x\},\{x,y\}\}$

    $\{x\} \in \bigcup R$ and $\{x,y\} \in \bigcup R$ also
    $x \in \bigcup \bigcup R$ and $x,y \in \bigcup \bigcup R$

    So $dom(R) = \{x \in \bigcup \bigcup R : \exists y <x,y> \in R\}$
    $ran(R) = \{y \in \bigcup \bigcup R : \exists x <x,y> \in R\}$

    \vspace{1em}
    \underline{Def} Properties of Relations

    Let $R$ be a relation and $A$ some set. We will say that $R$ is\ldots
    \begin{itemize}
        \item \emph{transitive on A} if $\forall x,y,z \in A$
        \[((x R y \and y R z) \implies x R z\]
        where $x R y$ means $<x,y> \in R$
        \item \emph{reflexive on A} if $\forall x \in A (x R x)$
        \item \emph{irreflexive on A} if $\forall x \in A \neg(x R x)$
        \item \emph{symmetric on A} if $\forall x,y \in A (xRy \leftrightarrow y R x)$
        \item \emph{partially orders A strictly} if R is irreflexive on A and transitive on A
        \item \emph{satisfies trichotomy on A} if  $\forall x,y \in A (x R y \lor y R x \lor x=y)$
        \item \emph{totally orders A strictly} if R partially orders A strictly and R satisfies trichotomy on A
        \item \emph{equivalence relation on A} if R is reflexive on A, symmetric on A, and transitive on A.
    \end{itemize}

    \underline{Ex} $A = \mathbb{N}$ and  $x R y$ if $x \equiv_3 y$. i.e. x-y is a multiple of 3. R is an equivalence relation on A.

    e.g. $x R y$ and $y R z$ then $x-y =3j$ for  some j and $y-z = 3k$ for some k
    $x-z=(x-y) + (y-z) = 3j + 3k = 3(j+k)$ so $x R z$

    \underline{Ex} $A = \mathbb{N}$  $x R y$ if x divides y is reflexive on A and transitive on A.

    \underline{Ex} $A = \mathbb{N}$ $x R y$ if $x < y$ R totally order A strictly

    \underline{Def} Let R be a relation and A a set. The restriction of R to A, $R \upharpoonright A = \{<x,y> \in R : x \in A \}$


    \underline{Def}
    \begin{itemize}
        \item R is a \emph{function} if R is a relation $\forall x \in dom(R) \exists! y <x,y> \in R$
        \item F, $x \in dome(R)$ we will write R(x) to mean the unique y with $<x,y> \in R$
    \end{itemize}

    \underline{Ex} $F = \{<0,0> ,<1,0>,<2,2>\}$ is a function $dom(F) = \{0,1,2\}$ and $ran(F) = \{0,2\}$ $F(0) = 0, F(1) = 0, F(2) = 2$

    \underline{Def} $F : A \rightarrow B$ means F is a function $dom(F) = A$ and $ran(F) \subseteq B$
    F above $F: \{0,1,2\} \rightarrow \{0,1,2\}$
    $F : A \rightarrow_{onto} B$ or $F: A \twoheadrightarrow B$
    if $F: A \rightarrow B$ and $ran(F) = B$ i.e. F is surjective or onto.
    $F: A \rightarrow^1-1 B$ or $F : A \hookrightarrow B$ if $F: A \rightarrow B$ and $\forall x_1, x_2 \in A$ then $(F(x_1) = F(x_2) \implies x_1=x_2)$ i.e. F is injective or one-to-one.
    $F: A \rightarrow^{1-1}_{onto} B$ or $F : A \leftrightarrow B$ as F is surjective and injective

    \underline{Ex}: $F(x) \sin X$
    $F : \mathbb{R} \rightarrow \mathbb{R}$
    $F : $

    \underline{Ex} $\sin([0,\pi / 2]) = [0,1]$ but we need to be more careful because everything is a set.

    \underline{Ex:} $F=\{<0,0>,<1,0>,<2,2>\}$ Recall $2 = \{0,1\}$
    $F(2)=2$
    the range of F applied to the set 2 is $\{F(0), F(1)\}=\{0,0\}=\{0\}$

    \underline{Def:} For a function F and a set A. $F" A$ (F applied to A) is $F" A = ran(F\upharpoonright A)=\{F(x) : x \in A \}$
    \underline{Ex} above F(2)=2, $F"2=\{0\}$


    \underline{\textbf{\emph{Jan 28}}}

    \emph{Note} axioms we have so far don't allow us to build many functions or relations.
    Given set A,B a function f: a $\rightarrow$ b will be a subset of the direct product A X B but axioms so far don't allow us to construct A X B
    even $\{\emptyset\}$ X B can't be produced

    \underline{Ex} Suppos ewe had the set of natural numbers $\mathbb{N}$
    And we want ot define $F(n) = S(n)$ (successor function).
    we need the set $\{<n,S(n)> : n \in \mathbb{N}\}\}$
    Let $\varphi(x,y) : y=S(x)$ i.e. $y = x \cup \{x\}$
    woule like to see the \"range\" of $\varphi$ exists applied to $\mathbb{N}$

    \subsection{Axiom 6 - Replacement Scheme}
    For every formula $\varphi(x,y)$ so that the variable $B$ does not occur in $\varphi$ we have the axiom.
    \[ \forall A ((\forall x \in A \exists! y \varphi(x,y)) \implies (\exists B \forall x \in A \exists y \in B \varphi(x,y)))\]

    \emph{Note} called replacement since we replace each x in A by the unique y with $\varphi(x,y)$
    Comprehension gives us the set range (f/A) where f is the function defined by $\varphi$

    \underline{Def: Direct Product} (Naive)
    \[S \times T = \{x : \exists s \in S \exists t \in T x=<s,t>\}\]
    need to see there is a set containing all such x's
    \underline{Lem:} Given set S and T there is a set Z so that for every $s \in S$ and $t \in T <s,t> \in Z$
    \underline{Cor:} THe direct prodcut exists: $S \times T = \{x \in Z : \exists s \in S \exists t \in T x=<s,t>\}$

    \underline{\emph{Proof of Lem:}}
    \begin{itemize}
        \item Fix $s \in S$.
        \item Let $\varphi(x,y)$ be $y=<s,x>$
        \item Then $\{s\} \times T = \{<s,t> : t \in T\} = \{<w \in B : \exists t \in T w = <s,t>\}$ where B is obtained from replacement applied to the set $A = T$ and $\varphi(x,y)$
        \item Next Let $\varphi(x,y)$ be $y = \{s\} \times T$ and $A=S$
        \item By replacement the set $n = \{\{s\} \times T : s \in S\}$
        \item Then $\bigcup n = \bigcup_{s \in S}\{s\} \times T = S \times T$
    \end{itemize}

    \textbf{Note:} We can now also define functions by formulas.

    \underline{Lemma:} Suppose $\forall x \in A \exists! y \varphi(x,y)$ then there is a function $f$ with $dom(f)=A$ so that for every $x \in A$ $f(x)$ is equal to the unique y so that $\varphi(x,y)$

    \emph{Pf:} Let $B$ be the set given by Replacement.
    Then $f = \{w \in A \times B : \exists x \exists y (w=<x,y>) \wedge \varphi(x,y))\}$

    \underline{Ex:} $A = \mathbb{N} \varphi(x,y) is y=S(x)$ then there is a function $f$ so with $dom(f) = \mathbb{N}$ and $f(n) =S(n)$ for all $n \in \mathbb{N}$

    \textbf{Note:} Multivariable function are relations can be defined as well

    \underline{Ex:} $f(x,y)$ can be viewed as $f: X \times Y \rightarrow Z$

    \textbf{Def:} Given a relation $R$ the \underline{inverse relation} $R^{-1}=\{<y,x> \in ran(R) \times dom(R) : <x,y> \in R\}$

    \underline{Note:} If f is a function, $f^{-1}$ won't necessarily be a function unless f is one-to-one

    \textbf{Def:} Given relations $F$ and $G$ the \underline{composition} $G \circ F$ is given by $G \circ F = \{<x,z> \in dom(F) \times ran(G): \exists y (<x,y> \in F \wedge <y,z> \in G)\}$
    \underline{Note:} If $F$ and $G$ are functions and $ran(F) \subseteq dom(G)$ then $G \circ F$ is the usual composition with $(G \circ F)(x)=G(F(x))$

    \underline{Def:} Suppose $(S,<) and (T,\prec)$ are relations, i.e. < is a relation with $dom(<)$ and $ran(<)$ subset of $S$ and similarly for $(T,\prec)$.
    Then the \underline{lexicographic product} on $S \times T$ is defined by relation $<_{lex}$ given by $<s,t>  <_{lex} <s',t'>$ when $s < s' \vee (s=s' \wedge t \prec t')$

    \underline{Ex:} $S=F=\{0,1,2\}$ < and $\prec$ be the usual ordering $0<1<2, 0 \prec 1 \prec 2$
    $<0,0> <_{lex} <0,1> <_{lex} <0,2> <_{lex} <1,0> \dots <_{lex} <2,2>$

    \emph{Note:} (homework) If < and $\prec$ are strict total orderings then so is $<_{lex}$.

    \underline{Def:} let $R$ be a relation on the $A$  and $S$ a relation on $B$.
    A function $F$ is an \underline{isomorphism} from $(A,R)$ onto $(B,S)$.
    if $F: A \rightarrow B$ so that $\forall x,y \in A$ $x R y \iff F(x) S F(y)$
    we say $(A,R)$ is isomorphic to $(B,S)$, $(A,R,) \cong (B,S)$ if there is a function $F$ which is an isomorphism of $(A,R)$ into $(B,S)$

    \underline{Def:} Let $R$ be an eqiuvalence relation on $A$ (domain and range of R are subsets of A).
    For $x \in A$ let the \underline{equivalence class} of $x$ be $[x] = \{y \in A : y R x\}$
    \emph{Note } $x R y \iff [x] = [y]$

    \underline{Def: } $A/R = \{[x] : x \in A\}$
    i.e. $\{w \in B : \exists x \in A w \ [x]\}$
    where $B$ is the set produced by applying Replacement scheme to the formula $\varphi(x,y) : y=[x]$

    \subsection{Interlude}
    Finite models of parts of Set Theory
    \emph{Note:} can describe a structure on a finite domain where we interpret $\in$ explicitly

    \textbf{Ex:} $X = \{a,b,c\}$ and $a \in b, b \in c$
    i.e. $\in$ is interpreted by relation $E$ where $E = \{ <a,b> , <b,c>\}$

    we can draw a directed graph where nodes are elements of x and arrow represent ordered pairs in $E$.

    Given such a structure, which axioms are satisfied when we interpret $\in$ by $E$

    \underline{Extensionality} two sets are the same exactly if they have the same elements:
    a has no elements, b has a single element a, and c has a single element b.
    satisfies extensionality as they are seperate.

    \underline{Pairing} given two sets then a set exists containing both:
    not satisfied - no set has both a and b as elements.

    \underline{Foundation} no infinite chain of membership:
    satisfied

    \emph{Ex:} $X = \{a,b,c\}$
    $E = \{<a,b>,<b,c>,<c,a>\}$
    (cyclic triangle)

    \underline{Extensionality}: a has element c, b has a, and c has b, satisfying
    \underline{Pairing} fails
    \underline{Foundation} fails

    \emph{Ex} $E =\{<a,c>,<b,c>\}$ (binary tree)
    Foundation and Pairing Fails, and so does Extensionality as a nd b are both empty sets.

    \emph{Ex} $X=\{a\}$ and $E=\{<a,a>\}$ (self-loop)
    Foundation fails, extensionality is trivially satisfied as there is only one set, and it satisfies pairing.

   \emph{\underline{\textbf{Jan 30}}}

    recall let r be a relation
    r is a strict toal order on A if - R is transitive on A, irreflixive on A and satisfies trichotomy
    \textbf{Def}: Let R be a relation. We say $a \in X$ is R-minimal in X if $\neg \exists z ( z \in X \wedge Z R a)$

    \emph{Note:}    a is not necessarily least, i.e. there may be more than one element of x which is R-minimal in X.

    \underline{EX:} $x=\mathbb{N}$ $n R m$ when $n<m$ 0 is the unique \underline{R-minimal} element in x.

    Let $A \subseteq \mathbb{N}$ $A \neq \emptyset$ then A has an R-minimal in A.

    \underline{Ex:} $x = \{ n \in \mathbb{N} : n \geq 2\} n R m$ when m is a proper multiple of n.
    p is R-minimal in X exactly when p is prime.
    infinite R-minimal

    \underline{Ex:} $x=\mathbb{Z}$ $n R m$ when $n < m$ No R-minimal element in X.
    no R-minimal

    \emph{Note:} we can also define R-maximal

    \textbf{Def} A relation R is \underline{well-founded} on X if for every non-empty $A \subseteq X$ there is an $a \in A$ which is R-minimal in A

    \underline{Ex:} $x=\mathbb{N}$ $n R m$ when $n < m$ R is well-founded on X.

    \underline{Ex:} $x = \{n \in \mathbb{N} : n \geq 2\}$ $n R m$ when m is a propers multiple of n.
    R is well-founded on X.
    Let A be a non-empty subset of X\@.
    pick $a_0 \in A$
    if $a_0$ is R-minimal in A we are done (no integer (element in A) is a proper divisor of it)
    else if not there is $a_1 \in A$ so that $a_1$ is a proper divisor of $a_0$
    Note $a_1 < a_0$
    if $a_1$ R-minimal in A we are done \dots etcetera
    This must stop at some point as we can't go below two since the sequence would be decreasing.
    The last term will be R-minimal in A.

    \underline{Ex:} $x \in \mathbb{Z}$ $n R m$ when $n < m$ R is not well-founded
    $A = \mathbb{Z}$ has no R-minimal element

    \underline{Ex:} $x=[0,1]$ $x R y$ when $x < y$.
    R is not well-founded on X.
    $A = (0,1)$ has no R-minimal element because if $a \in A$ $a>0$ so let $b=a/2$ and $b<a$ showing we can find something smaller.

    \textbf{Def:} A relation R is \underline{well-orders} a set A if R totally orders A strictly (transitive and satisfies trichotomy) and R is well-founded on A.

    \underline{Ex:} < well-orders $\mathbb{N}$.

    \underline{Ex:} < does not well-order $\mathbb{Z}$ or $[0,1]$

    \underline{Ex:} proper multiple relation does not well-order $\{n \in \mathbb{N} : n \geq 2\}$

    \textbf{Def}: A \underline{well-ordering} is a pair $(A ; R)$ where A is a set and R is a relation which well-orders R.

    \underline{Ex:} $(\mathbb{N};<)$ is a well-ordering

    \emph{Note:} For a well-ordering $(A;R)$ every non-empty $X \subseteq A $ has an \underline{R-least} element, i.e. there is a unique element of X which is R-minimal in X.
    suppose both x and y are R-minimal in X
    either x=y or $x R y$ meaining y is not R-minmal or vice versa.

    \emph{Note:} If $(S;<)$ and $(T;\prec)$ are both well-orderings then so is their lexicographic product.

    \emph{Note:} If R well-orders A and $B \subseteq A$ is a non-empty subset then R well-orders B.

    \underline{ex:} $(N;<)$

    B = primes

    < is a well-ordering of B
    
    \underline{Ex:} $0=\emptyset$ , $1 = \{\emptyset\}$, $2 = \{\{\emptyset\}\}\}$
    
    $x = \{0,1,2\}$ and $R = \in$
    
    Then $\in$ well-orders X.
    
    \emph{Note:} If X is a finite set then any strict total ordering of X is actually a well-ordering of X.
    
    
    \textbf{\underline{Ordinals}}
    Recall: $0=\emptyset$ and $n+1=S(n)$ where $S(x) = x \cup \{x\}$
    Informally: Natural numbers are 0, S(0), S(S(0))
    
    \textbf{Def}: A set $Z$ is \underline{transitive} if $\forall y \in Z (y \subseteq Z)$
    i.e. $\forall x \forall y ((x \in y \wedge y \in Z) \rightarrow x \in z)$
    
    \underline{Ex:} $z=2=\{\emptyset\{\emptyset\}\}$
    $y=\emptyset$
    $\emptyset \subseteq Z$
    $y=\{\emptyset\}$
    $\{\emptyset\} \subseteq Z$ since $\emptyset \in Z$
    so Z is transitive
    
    \underline{Ex:} $Z=\{\{\emptyset\}\}$ is not transitive
    $y=\{\emptyset\}$
    $y \in Z$
    $x=\emptyset$
    $x \in y$
    but $\emptyset \notin Z$
    
    \emph{Note:} The $\in $ relation behaves transitively with respect to the set Z.
    Z is transitive of $(x \in y \wedge y \in z) \implies x \in z$ for any x and y.

    (natural numbers are transitive)
    
    \textbf{Def} Z is a (von Neumann) \underline{ordinal} if Z is transitive, and Z is well-ordered by $\in$
    
    \underline{Ex:} $2 = \{\emptyset\{\emptyset\}\}$ is an ordinal
    elements: $\emptyset, \{\emptyset\}$
    
    $3=\{0,1,2\}$ is an ordinal
    
    every natural number is an ordinal (see this ?later on?)
    
    \emph{Note: } we usually use Greek letters to denote ordinal.
    
    \textbf{Def} For ordinal $\alpha$ and $\beta$:
    
    $\alpha < \beta$ means $\alpha \in \beta$
    
    $\alpha \leq \beta$ means $\alpha \in \beta$ or $\alpha = \beta$
    
    \textbf{Def} ON or Ord is the class of all ordinals.
    
    \emph{Note:} $ON$ is not a set
    
    $"x \in ON"$ abbreviates \("\)x is transitive and $\in$ well-order x(\")
    
    $" \forall x \in ON \varphi(x)"$ abbreviates $" \forall x (( x is transitive \wedge \in well-order x) \rightarrow \varphi(x))"$
    
    $"x \subseteq ON"$ , $"x \cap ON"$ are similarly abbreviated
    
    \textbf{\underline{Theorem}} $ON$ is well-ordered by $\in$, i.e. 
    \begin{enumerate}
        \item $\forall \alpha, \beta, \gamma \in ON ((\alpha < \beta \wedge \beta < \gamma) \rightarrow (\alpha < \gamma))$
        \item $\forall \alpha \in ON (\alpha \neg < \alpha)$
        \item $\forall \alpha, \beta \in ON ( \alpha < \beta \vee \beta < \alpha \vee \alpha = \beta)$
        \item $\forall z ((z \notin \emptyset \wedge z \subseteq ON) \rightarrow z has an \in - minimal element)$
        \end{enumerate}

    \underline{cor} There is no set which contains all ordinals.
    \underline{pf of cor} suppose X is a set which contains all ordinal
    By comprehension let $Y=\{Z \in X : Z is an ordinal\}$
    Then $Y=ON$ is a set.
    Theorem implies Y is transitive
    (1) If $alpha \in \beta$ and $\beta \in Y$ then $(\alpha < \beta)$ and $\alpha \in Y$ Y is well-ordered by $\in$
    Hence Y is an ordinal so $Y \in Y$
    but since Y is an ordinal by (2) of Theorem $Y \notin Y$ so contradiction

    \emph{Note} This is known as the Burali-Forti Paradox.

    we need some lemmas before proving the theorem

    \underline{Lem 1} ON is a transitive class, i.e. if $\alpha \in ON$ and $z \in \alpha$ then $z \in ON$

    \underline{pf} let $\alpha$ be an ordinal.
    Then $\alpha$ is transitive.
    Since $z \in \alpha$ then $z \subseteq \alpha$.
    $\in$ well-orders $\alpha$, so $\in$ well-orders $z$.
    To see $z$ is transitive: suppose $x \in y$ and $y \in z$
    so $y \in \alpha$ so $y \subseteq \alpha$ (since $\alpha$ is transitive and $z \subseteq \alpha$)
    so $x \in \alpha$
    hence $x,y,z \in \alpha$ and $x \in y$ and $y \in z$ so $x \in z$
    since $\in$ well-orders $\alpha$ and hence $\in$ is a transitive relation in $\alpha$
    which is what we wanted.
    Hence $z$ is transitive and well-ordered by $\in$ , so $z$ is an ordinal

    \emph{ordinals} \underline{feb 4}

    Recall: A set z is a transitive if whenever $y \in z$ then $y \subseteq$, i.e. if $x \in y$ and $y \in z$ then $x \in z$.
    A set $\alpha$ is an \underline{ordinal} if $\alpha$ is transitive and $\in$ well-orders $\alpha$.
    $\alpha < \beta \leftrightarrow \alpha \in \beta$ and $alpha \leq \beta \leftrightarrow \alpha < \beta \vee \alpha = \beta$

    \underline{Theorem} $ON$ is well-ordered by $\in$

    \underline{Lem 1} $ON$ is a transitive class, i.e. $\alpha \in ON$ and $z \in \alpha$ then $z \in ON$ where $ON$ is the class of ordinals. $\alpha$ is an ordinal (is what the in means)

    \underline{Lem 2} For $\alpha , \beta \in ON$ $\alpha \leq \beta $ iff $\alpha \subseteq \beta$.
    \begin{proof}

    $\alpha \leq \beta$ means $\alpha \in \beta \vee \alpha = \beta$
    if $\alpha = \beta$ immediately $\alpha \subseteq \beta$
    if $\alpha \in \beta$ then $\alpha \subseteq \beta$ since $\beta$ is transitive.

    suppose $\alpha \subseteq \beta$
    if $\alpha = \beta$ we are done, so suppose not.
    let $X = \beta \\ \alpha \neq \emptyset$ , $X \subseteq \beta$
    $\beta$ is wel-ordered by $\in$,
    X has an $\in$-minimal element $\xi$
    show $\xi = \alpha$
    For $\mu \in \xi$ we have $\mu \in \beta$ and $\mu \notin X$ so $\mu \in \alpha$
    Hence $\xi \leq \alpha$.
    Suppose $\xi \neq \alpha$.
    Then there is a$\mu \in \alpha \setminus \xi$
    $\mu$ and $\xi$ both in $\beta$.
    $\beta$ is totally ordered by $\in$ so one of $\mu \in \xi$, $\xi \in \mu$, or $\xi = \mu$.
    $\mu \neq \xi$ by choice of $\mu$
    since $\mu \in \xi$ and $\xi \notin \alpha$ $(\xi \in X = \beta \setminus \alpha)$ so $\mu = \xi$
    if $\xi \in \mu$ then $\xi \in \alpha$ (since $\mu \in \alpha$) but $\xi \notin \alpha$, hence $\xi \notin \mu$
    Hence $\xi = \alpha$, so $\alpha \in \beta$.
    \end{proof}

    \underline{Lem 3} If $\alpha, \beta \in ON$ then $\alpha \cap \beta \in ON$.
    (we will see $\alpha \cap \beta = min(\alpha, \beta)$)

    \begin{proof}
        $\alpha \cap \beta \subseteq \alpha$ so $\alpha \cap \beta$ is well-ordered by $\in$.
        $\alpha \cap \beta$ is transitive since if $x \in \alpha \cap \beta$.
        then $x \in \alpha$ so $x \subseteq \alpha$, and $x \in \beta$ so $x \subseteq \beta$.
        Hence $x \subseteq \alpha \cap \beta$.
        So $\alpha \cap \beta$ is an ordinal.
    \end{proof}

Proof of Theorem:
\begin{proof}
    \begin{enumerate}
        \item $\forall \alpha, \beta, \gamma \in ON$ and $\alpha < \beta \wedge \beta < \gamma$ then $\alpha < \gamma$
        \begin{itemize}
            \item $\alpha$ is a transitive set.
            \item $\beta < \gamma$ means $\beta \in \gamma$.
            \item $\alpha < \beta$ means $\alpha \in \beta$.
            \item So $\alpha \in \gamma$ i.e. $\alpha < \gamma$.
        \end{itemize}
        \item $\forall \alpha \in ON$ $\alpha \notin \alpha$
        \begin{itemize}
            \item $\alpha$ is well-ordered by $\in$.
            \item If $\alpha \in \alpha$ then $\forall x \in \alpha$ ($x \notin x$).
            \item Hence $\alpha \notin \alpha$.
        \end{itemize}
        \item $\forall \alpha, \beta \in ON$ ($\alpha \in \beta \vee \beta \in \alpha \vee \alpha = \beta$)
        \begin{itemize}
            \item Let $\delta = \alpha \cap \beta$.
            \item $\delta \in ON$ by Lemma 3.
            \item $\delta \subseteq \alpha$ so $\delta \leq \alpha$ by Lemma 2.
            \item So either $\delta = \alpha$ or $\delta \in \alpha$.
            \item If $\delta = \alpha$ then $\alpha \subseteq \beta$ so either $\alpha = \beta$ or $\alpha \in \beta$ and we are done.
            \item So assume $\delta \in \alpha$
            \item $\delta \subseteq \beta$ so either $\delta = \beta$ or $\delta \in \beta$.
            \item if $\delta = \beta$ then $\beta \subseteq \alpha$ so we are done.
            \item so assume $\delta \in \beta$
            \item so $\delta \in \alpha \cap \beta = \delta$ but $\delta \notin \delta$, contradiction
            \item so either $\delta = \alpha$ or $\delta = \beta$ and we are done
        \end{itemize}
        \item $\forall z$ ($z \neq \emptyset \wedge z \subseteq ON \rightarrow$ $z$ has an $\in$-least element.
        \begin{itemize}
            \item Let $\alpha \in z$ be arbitrary.
            \item If $\alpha$ is $\in$-minimal in $z$ we are done.
            \item Suppose not.
            \item Consider $\alpha \cap z = \{\beta \in z : \beta < \alpha\}$
            \item $\alpha \cap z \subseteq \alpha$
            \item if $\alpha \cap z = \emptyset$ we are done as $\alpha = \emptyset$ would be $\in$-least in $z$
            \item if $\alpha \cap z \neq \emptyset$ then there is an $\in$-minimal element of $\alpha \cap z$, call this $\gamma$
            \item $\gamma \in \alpha \cap z$ and for any $\beta \in \alpha (\beta \notin \alpha \cap z)$
            \item if $\beta \in \gamma$ then $\beta \in \alpha$ hence $\beta \notin z$
            \item Hence $\gamma$ is $\in$-minimal element of $z$ which is what we wanted
        \end{itemize}
    \end{enumerate}
\end{proof}

    \underline{Ex}
    \begin{itemize}
        \item $0=\emptyset$ is the least ordinal
        \item $1 = \{\emptyset\}$ is the next biggest ordinal (no ordinal $\alpha$ with $0 < \alpha$ and $\alpha < 1$)
    \end{itemize}

    \emph{Note} If $\alpha \in ON$ then $\alpha = \{\gamma \in ON : \gamma < \alpha\}$

    Recall $S(x) = x \cup \{x\}$

    \underline{Lem} If $\alpha \in ON$ then $S(x) \in ON$.
    \begin{proof}
        If $\beta \in S(\alpha)$ then either $\beta \in \alpha$ or $\beta = \alpha$.
        In both cases $\beta \subseteq \alpha$ hence $\beta \subseteq S(\alpha)$.
        So $S(x)$ is transitive.
        If $z \subseteq S(x)$ and $z \neq \emptyset$.
        If $z = \{\alpha\}$ then $\alpha$ is an $\in$-minimal element of $z$ (since $\alpha \notin \alpha$)
        Otherwise, $z \cap \alpha \neq \emptyset$ has an $\in$-minimal element since $\alpha \in ON$ which will be an $\in$-minimal element of $z$.
    \end{proof}

    \underline{Lemma} For $\alpha \in ON$, $S(\alpha)$ is the immediate succesor of $\alpha$, i.e. $\alpha < S(x)$ and there is no ordinal $\beta$ with $\alpha < \beta$ and $\beta < S(\alpha)$
    \begin{proof}
        $\alpha \in S(\alpha) = \alpha \cup \{\alpha\}$ so $\alpha < S(\alpha)$
        If $\beta < S(alpha)$ then $\beta \in \alpha \cup \{alpha\}$
        So either $\beta \in \alpha$ or $\beta = \alpha$
        i.e. either $\beta < \alpha$ or $\beta = \alpha$ so can't have $\alpha < \beta$
    \end{proof}

    \textbf{Def} An ordinal $\beta$ is a \underline{successor ordinal} if $\beta = S(\alpha)$ for some $\alpha$.
    Otherwise $\beta$ is a \underline{limit ordinal} if $\beta \neq 0$ and $\beta$ is not a successor ordinal.
    A \underline{finite ordinal} or \underline{natural number} is if every $\alpha \subseteq \beta$ is either 0 or a successor ordinal.

    \emph{Note:} Axioms so far only produce finite ordinal e.g. 0,1,2,....

    \emph{Note:} $\beta \neq 0$ is a limit ordinal $\iff$ $\forall \alpha < \beta$ $S(\alpha) < \beta$

    \subsection{Axiom 7: Axiom of Infinity}
    $\exists X (\emptyset \in X \wedge \forall y (y \in X \rightarrow S(y) \in X))$

    \emph{Note: } We want to see that this X is an infinite set containing every finite ordinal.

    \underline{Lemma: } If $n$ is a natural number then $S(n)$ is a natural number.
    \begin{proof}
        $S(n)$ is an ordinal from earlier.
        If $\alpha \leq S(n)$ then
        If $\alpha = S(n)$ then $\alpha$ is a successor (done)
        Otherwise if $\alpha < S(n)$ then $\alpha \in n \cup \{n\}$
        So $\alpha \leq n$ so either $\alpha = 0$ or $\alpha$ is a successor.
    \end{proof}

    \underline{Theorem: } \underline{Principle of Ordinary Induction}
    For any set $X$, if $\emptyset \in X$ and $\forall y \in X (S(y) \in X)$ then $X$ contains all natural numbers.

    \begin{proof}
        Suppose the above is true.
        Suppose $n$ is a natural number with $n \notin X$.
        Let $Y = S(n) \setminus X \neq \emptyset$ since $n \in Y$.
        Y is a set of natural numbers.
        Y has an $\in$-minimal element, call it $k$.
        $k \leq n$
        Either $k=0$ or $k = S(j)$ for some $j$.
        $k \neq 0$ since $k \notin X$ and $0 \in X$
        $k \neq S(j)$ since $j \notin Y$ ($j \in k$ and $k \in$-minimal in $Y$)
        if $k = S(j)$ then $j \in S(n)$ so $j \in X$ so $S(j) = k \in X$, contradiction
        Hence no natural number $n \notin X$
    \end{proof}

    \textbf{Def} Let $\omega = \{n \in X: n is a natural number\}$ where $X$ is the set from the Axiom of Infinity.
    So $\omega$ is the set of natural numbers sometimes denoted $\mathbb{N}$

    \emph{Note:} we will see that $\omega$ is an ordinal.
    $\omega$ is the least infinite ordinal.
    $\omega$ is the least limit ordinal.

    \underline{\emph{Feb 6.}}
    Recall $\omega = \{n \in X : n is a natural number\}$ where $X$ is the set from the Axiom of Infinity so that $\emptyset \notin X \wedge \forall Y (Y \in X \rightarrow S(X) \in X)$

    \underline{Lemma: } Suppose $X$ is a set of ordinals which is an initial segment of $ON$. i.e. $\forall \beta \in X \forall \alpha \in \beta (\alpha \in X)$ then $X \in  ON$

    \begin{proof}
        $X$ is well-ordered by $\in$ since any set of ordinals is well-ordered by $\in$

        X is transitive since if $\alpha \in \beta$ and $\beta \in X$ then $\alpha < \beta$ so $\alpha \in X$

        Hence $X$ is an ordinal.

    \end{proof}

    \underline{Lemma: } $\omega$ is the least-limit ordinal.
    \begin{proof}
        $\omega$ is an ordinal since if $n \in \omega$ then $n$ is a natural number and if $m < n$ then m is a natural number, hence $m \in \omega$

        so $\omega$ is an initial segment of $ON$

        $\omega \neq \emptyset$

        suppose $\omega = S(y)$ for some $y$ then $y \in S(y) = \omega$ so $y$ is a natural number so $S(y) = \omega$ would be a natural number, so $\omega \in \omega$ contradicts that $\alpha \notin \alpha$ for any ordinal $\alpha$

        hence $\omega$ is a limit ordinal.

        every $\alpha < \omega$ is a natural number so either 0 or $S(n)$ some n

        hence $\omega$ is the least limit ordinal

    \end{proof}

    \emph{Note: } Now we have $0,1,2,\dots,\omega,S(\omega), S(S(\omega)),\dots$.
    Like the think of $S(\omega)$ as $\omega + 1$, etcetera.
    Need to define ordinal arithmetic $\alpha + \beta$, $\alpha \dot \beta$, $\alpha^{\beta}$.
    we will define these in terms of certain well-orderings we define.
    See that every well-ordering corresponds to a unique ordinal.

    \underline{Ex} subsets of $\mathbb{R}$ with the usual ordering <.
    Let $A = \{1-\frac{1}{n} : n \geq 1\}$
    $(A ; <) \cong (\omega,\in)$
    Let $B = \{ 1-\frac{1}{n}: n \geq 1\}\cup \{1\}$
    $(B;<) \cong (S(\omega); \in)$ where $S(\omega) = \omega + 1$
    Let $C = \{1-\frac{1}{n}:n\geq1\}\cup\{2-1/n:n\geq1\}$
    $(C;<) \cong (\omega + \omega;\in)$


    \underline{Lemma} If $\alpha$ and $\beta$ are ordinals and $f : \alpha \leftrightarrow \beta$ is an isomorphism from $(\alpha,\in)$ onto $(\beta,\in)$ then $\alpha = \beta$ and $f$ is the identity function.

    \begin{proof}
        Fix $\xi \in \alpha$, sp $f(\xi) \in \beta$.
        If $\zeta < \xi$ then $f(\zeta) < f(\xi)$ and vice versa.
        Then $\{\nu \in \beta : \nu < f(\xi)\}=\{f(\mu) : \mu \in \alpha \wedge \mu < \xi\}$.
        Hence $f(\xi) = f``\xi = \{f(\gamma) : \gamma \in \xi\}$
        Claim that $f(\xi) = \xi$ for all $\xi \in \alpha$.
        Let $X = \{\xi \in \alpha : f(\xi) \neq \xi\}$
        If $X = \emptyset$ we are done so suppose not.
        Let $X \neq \emptyset$ then $X$ is a set of ordinals so there is a least $\xi \in X$.
        If $\mu < \xi$ then $f(\mu)=\mu$.
        So $f(\xi)=\{f(\mu) : \mu < \xi\} = \{\mu : \mu < \xi\} = \xi$ contradicting that $f(\xi) \neq \xi$.
        Hence $f(\xi) = \xi $ for every $\xi \in \alpha$ i.e. $f$ is the identity map, and $\beta = \alpha$.
    \end{proof}

    \emph{Note: } There can be bijections between unequal ordinals, but they won't be order preserving.

    \underline{Ex: } $\omega = 0,1,2,\dots$ and $\omega + 1 = 0,1,2,\dots,\omega$.
    Pairing $\omega$ with 0 and the rest with their successors, so we have a bijection

    \emph{Note: } In the above proof we used the \underline{principle of transfinite induction}.
    Every non-empty set of ordinals has a least element.
    So fi $\varphi(x)$ is a formula and for all $\alpha \in ON$
    $(\forall \beta < \alpha \varphi(\beta)) \rightarrow \varphi(\alpha)$
    Then $\varphi(\alpha)$ holds for all ordinals $\alpha$

    \emph{Note: } Often split into cases for $\alpha$.
    show $\varphi(0)$.
    show $\varphi(\beta) \rightarrow \varphi(S(\beta))$.
    for a limit ordinal $\gamma$ show $(\forall \beta < \gamma \varphi(\beta)) \rightarrow \varphi(\gamma)$
    (orders are rigid we cant have an order isomorphism? maybe?)

    \underline{\textbf{Theorem}}: If R is a well-ordering of A then there is a unique ordinal $\alpha$ so that $(A;R) \cong (\alpha;\in)$
    (ordinal gives a representation of all orderings up to isomorphism)

    \begin{proof}
        $\alpha$ is unique from the previous lemma since if $f : (A;R) \leftrightarrow (\alpha;\in)$ and $g : (A;R) \leftrightarrow (\beta;\in)$.
        Then $g \of f^{-1} : (\alpha;\in) \leftrightarrow (\beta;\in)$.
        So $g \of f^{-1} = identity$ i.e. $f=g$ and $\alpha = \beta$
        For $a \in A$ let $a \downarrow = \{x\in A : x R a\}$.
        Then $a \downarrow$ is also well-ordered by R.
        Call $a \in A$ \underline{good} if $(a \downarrow;R) \cong (\xi; \in)$ for some ordinal $\xi$.
        Let $G = \{a \in A : a is good\}$.
        Define $f : G \rightarrow ON$ by setting $f(a) = $ the unique $\xi \in ON$ so that $(a\downarrow;R)\cong(\xi;\in)$
        Also for $a \in G$ let $h_a$ be the unique isomorphism $h_a=(a\downarrow;R)\cong(f(a);\in)$ ($f(a)=\xi$)
        This is justified by replacement
        If $a\in G$ and $c R a$ (i.e. $c \in a\downarrow$) then $c \in G$ since $c \downarrow \subseteq a\downarrow$
        $h_c = h_a \uprightharpoon(c\downarrow): (c\downarrow;R)\cong(h_a(c);\in)$
        ($\upharpoonright$ is 'restricted')
        (Lecture drawing 54 mins into class)
        So $f(c) = h_a(c) < f(a)$
        Hence $f : G \rightarrow ON$ is order-preserving, i.e. $cRa \leftrightarrow f(c) < f(a)$.
        $ran(f)$ is an initial segment of $ON$ since if $\xi \in  ran(f)$ there is some $a \in G$ with $f(a) = \xi$
        So if $\mu < \xi`$ there is $c \in a\downarrow$ with $h_a(c) = \mu$
        But then $c \in G$ and $f(c) = h_a(c) = \mu$ so $\mu \in ran(f)$
        Hence $ran(f)$ is an ordinal $\alpha$ and f is an isomorphism from $(G;R)$ onto $(\alpha;\in)$.
        Want to show $G=A$.
        Suppose not, then let $e$ be the R-least elements of $A \setminus G$
        Then $e \downarrow = G$ since G is an R-initial segment of A.
        Hence $(e \downarrow;R) = (G;R) \cong (\alpha;\in)$.
        Hence $e \in G$, contradiction.
        Hence $G=A$ so $(A;R)\cong(\alpha;\in)$
    \end{proof}

    \textbf{Def: } For a well-ordering $(A;R)$ the \underline{order type} of $(A;R)$ , $type(A;R)$ is the unique ordinal $\alpha$ such that (A;R)$\cong$($\alpha;\in$)

    Recall: For (A;R) and (B;s) the \underline{lexicographic product} is the ordering of $A \times B$ given by $(a_1,b_1) <_{lex}(a_2,b_2) \leftrightarrow a_1 R a_2 \vee (a_1 = a_2 \wedge b_1 s b_2)$.

    \tetxbf{Def: } \underline{Ordinal arithmetic}
    For $\alpha,\beta \in ON$
    $\alpha \dot \beta = type(\beta \times \alpha, <_{lex})$
    $\alpha + \beta = type(\{0\}\times \alpha \cup \{1\} \times \beta, <_{lex})$

    \underline{Ex: } $\omega + 1$: $\{0\} \times \omega \cup \{1\}\times 1$ (see drawing at feb 6. 72 mins)

\end{document}

