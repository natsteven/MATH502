% Math 402/502 Homework



\documentclass[11pt]{amsart}


\pagestyle{empty}
\thispagestyle{empty}

\usepackage{graphicx}

\usepackage{amsmath}
\usepackage{amssymb}
\usepackage{latexsym}
\usepackage{amsopn}
\usepackage{amsthm}

\def\bbN{{\mathbb N}}
\def\bbR{{\mathbb R}}
\def\bbQ{{\mathbb Q}}
\def\bbZ{{\mathbb Z}}
\def\bbF{{\mathbb F}}
\def\bbE{{\mathbb E}}
\def\bbP{{\mathbb P}}

\DeclareMathOperator{\dom}{dom}

\newcommand{\set}[1]{\left\{\,#1\,\right\}}
\newcommand{\NN}{\mathbb N}
\newcommand{\ZZ}{\mathbb Z}
\newcommand{\QQ}{\mathbb Q}
\newcommand{\RR}{\mathbb R}

\DeclareMathOperator{\type}{type}
\DeclareMathOperator{\TC}{TC}
\newcommand{\leftsuperscript}[2]{{\vphantom{{#1}}}^{#2}{#1}}

\usepackage{bbold}


\newcommand{\hint}[1]{{\small \em \noindent [Hint: #1]}}


\begin{document}

\begin{center}
{\Large Math 402/502 Homework 6 -- due Friday, February 28}
\ \\
\end{center}

\ \\
 
 \begin{enumerate}

\item A set $X$ is  \emph{finite} if $X \preccurlyeq n$ for some $n \in \omega$; a set if \emph{infinite} if it is not finite. A set $X$ is \emph{Dedekind infinite} if there is an injection from $X$ into a proper subset of itself, i.e., there is $Y \subseteq X$ with $Y \neq X$ such that $X \preccurlyeq Y$. A set is {\it Dedekind finite} if it is not Dedekind infinite.

\ 
\begin{enumerate}
\item Show that if $X$ is Dedekind infinite, then $X$ is infinite.
\\
We have $X \preceq Y \text{and} Y \subset X$.
To show that $X$ is infinite we will show that $X \npreceq n$ for some $n \in \omega$.
Let us assume that $X$ is finite so $X \preceq n$ for some $n \in \omega$.
Since $Y \subset X$ and $X$ is finite than so is $Y$.
However if $Y$ is finite then $Y \preceq m$ for some $m \in \omega$.
and since $Y \subset X$ then $m < n$ and $Y \prec X$
Leading to a contradiction as $X$ is Dedekind infinite.
So $X$ is infinite.

\vfill
\item Show that $\omega$ is Dedekind infinite.
\\
The definition of Dedekind infinite is that there is an injection from $X$ into a proper subset of itself.
In this case we can take the function $f: \mathbb{N} \rightarrow \mathbb{N}\setminus \set{0}$ i.e. $f(x) = x+1 \forall x \in \omega$.
This function is injective and maps $\omega$ to a proper subset of itself.
$f(n) = f(m) \forall n,m \in \omega$ implies that $n = m$.
Therefore $\omega$ is Dedekind infinite.

\vfill
\item Show (without using the Axiom of Choice) that $X$ is Dedekind infinite if and only if $\omega \preccurlyeq X$.
\\
$\implies$: If $X$ is Dedekind infinite then there exists some $a \in X\setminus Y : Y \subset X \preceq Y$.
We can define $X$ recursively as $x_0 = a$ and $x_{n+1} = f(x_n) \forall n \in \omega$.
Here $a \notin Y$ and $f(x_n) \in Y \forall n \geq 1$
And if $0<i<j$ then if $x_i = x_j$ implying $f(x_{i-1}) = f(x_{j-1})$ implying $x_{i-1} = x_{j-1}$
But this gives us $x_1 = x_{j-i+1}$ implying $x_0 = x_{j-1}$ which is a contradiction.
Therefore we can define some function $g: \omega \rightarrow X$ implying $\omega \preceq X$.
$\impliedby$: If $\omega \preceq X$ then there exists some function $g: \omega \rightarrow X$.
Let $A = g(\omega)$
We can define $f: X \rightarrow X$ for $x = g(n) \in A$ as $f(x) = g(n+1)$ and for $x \in X \setminus A$ as $f(x) = x$.
$g(0) \notin ran(f)$ so $ran(f) \subset X$ and $f$ is injective because for $x,y \in X$ either:
$x,y \in X\setminus A$ then $f(x) = f(y)$ implies $x = y$
or $x \in A, y \in X \setminus A$ then $f(x)\in A$ and $f(y)=y\notin A$ so $f(x) \neq f(y)$
or $x,y \in A$ then $f(x) = g(n+1) = f(y) = g(m+1)$ implies $g(n) = g(m)$ implying $n = m$ and $x = y$
Therefore $f$ is an injection from $X$ to a proper subset of itself and $X$ is Dedekind infinite.
\vfill
\end{enumerate}


\newpage

\item Show that ordinal arithmetic does not increase cardinality. That is, suppose that $\alpha$ and $\beta$ are ordinals with $2 \leq \alpha$ and $\omega \leq \beta$. Show the following, where the operations are ordinal arithmetic:
\\
We know if $\alpha \geq \omega$ then $|\omega \times \omega| = |\omega|$.
Hence, if $\kappa \geq \omega$ is a cardinal, then $\kappa \times \kappa = \kappa$.
\ 
\begin{enumerate}
\item $|\alpha + \beta| = \max(|\alpha|, |\beta|)$
\\ If $|\alpha| \leq |\beta|$ then $|\alpha + \beta| \leq |\beta + \beta| = |\beta| = max(|\alpha|,|\beta|)$
and similarly if $|\beta| \leq |\alpha|$ then $\alpha \geq \omega$ so $|\alpha + \beta| \leq |\alpha| = max(|\alpha|,|\beta|)$
So $|\alpha + \beta| \leq max(|\alpha|,|\beta|)$
Finally $|\alpha + \beta| \geq max(|\alpha|,|\beta|)$ as $\alpha, \beta$ both $\subseteq \alpha + \beta$
So $|\alpha + \beta| = max(|\alpha|,|\beta|)$
\vfill
\item $|\alpha \cdot \beta| = \max(|\alpha|, |\beta|)$
\\ Using the definition of $\cdot$ we have $|\alpha \cdot \beta| = |\beta \times \alpha| = |\beta| \cdot |\alpha|$
For any cardinals $\kappa, \lambda$ we have $\kappa \cdot \lambda = max(\kappa,\lambda)$
So $|\alpha \cdot \beta| = max(|\alpha|,|\beta|)$
\vfill
\item $|\alpha^\beta| = \max(|\alpha|, |\beta|)$
\\ $|\alpha^\beta| = |\set{f : \beta \rightarrow \alpha}| \leq |\alpha|^{|\beta|}$
And we have |$\alpha|^{|\beta|} = max(|\alpha|,|\beta|)$
Then we know $|\alpha^\beta| \geq |alpha|$ and $|\alpha^\beta| \geq |\beta|$
Hence, $|\alpha^\beta| = max(|\alpha|,|\beta|)$
\vfill
\end{enumerate}

\newpage

\item Show that there is an $\aleph$-fixed point, i.e., there is an ordinal $\gamma$ (which is necessarily a cardinal), so that $\aleph_{\gamma} = \gamma$.
\\ Define a sequence such that $\aleph_0 = \alpha_0 = \omega$.
Then $\alpha_{n+1} = \aleph_{\alpha_n}$.
For each limit ordinal $\gamma$ let $\alpha_\gamma = \sup \set{\alpha_n : n < \gamma}$
Let $\gamma = \sup \set{\alpha_n : n < \omega_1}$
$\alpha_0 = 0, \alpha_1 = \aleph_0 > 0 = \alpha_0, \alpha_2 = \aleph_{\aleph_0} > \aleph_0 = 1 \dots$
So we have a strictly increasing sequence and $\aleph{\alpha_n} < \aleph_\gamma$ for $n < \omega_1$
And $\gamma \leq \aleph_\gamma$.
If $\gamma < \aleph_\gamma$ then $\exists \beta < \gamma$ s.t. $\gamma = \aleph_\beta$
and there would be some $n < \omega_1$ s.t. $\beta < \alpha_n$.
Implying $\aleph_\beta < \aleph_{\alpha_n} = \alpha_{n+1} < \gamma$ contradicting $\gamma = \aleph_\beta$
So $\gamma = \aleph_\gamma$

\newpage

\item  Suppose $A$ is well-orderable, and there is a surjection $f$ from $A$ onto $B$. Show (without using the Axiom of Choice) that there is an injection $g$ from $B$ into $A$.
\\ Since $A$ is well-orderable, let $<$ be a well-ordering of $A$.
Consider the set $F(b) = {a \in A | f(a) = b}$
We know $F(b)$ is non-empty for each $b \in B$.
Define $g: B \rightarrow A$ as the $<$-minimal element of $F(b)$ for each $b \in B$.
Suppose $b_1, b_2 \in B$ where $b_1 \neq b_2$ then $F(b_1) \cap F(b_2) = \emptyset$.
Since $g(b_1) \in F(b_1)$ and $g(b_2) \in F(b_2)$ we have $g(b_1) \neq g(b_2)$.
Thus, g is an injection from B into A.



\newpage

\item Recall that $\omega_1$ is the least uncountable ordinal. Without using the Axiom of Choice, show that there is a surjection from $\mathcal{P}(\omega)$ onto $\omega_1$.

\ 
\newcommand{\PP}{\mathcal{P}}
\noindent
{\small [Hint: First, show that $\mathcal{P}(\omega) \times \mathcal{P}(\omega \times \omega) \approx \mathcal{P}(\omega)$, so it suffices to find a surjection from $\mathcal{P}(\omega) \times \mathcal{P}(\omega \times \omega)$ onto $\omega_1$. To show this, consider the set
\[ W= \{ (X,R) \in \mathcal{P}(\omega) \times \mathcal{P}(\omega \times \omega) : \text{$R$ is a well-ordering of $X$} \} \]
from the proof of Hartogs' Theorem.] }
\\
$\omega \times \omega = \omega$ so $\PP(\omega \times \omega ) \approx \PP(\omega)$
and $\PP(\omega) \times \PP(\omega \times \omega ) \approx \PP(\omega) \times \PP(\omega)$
Then define a bijection $f: \PP(\omega) \times \PP(\omega) \rightarrow \PP(\omega)$ as $f(A,B) = \set{2n : n\in A} \cup \set{2n+1 : n \in B})$
So $\PP(\omega) \times \PP(\omega) \approx \PP(\omega)$
Now using $W$ from the hint.
Define a function $g: W \rightarrow ON$ that maps each $(X,R)$ to its order type.
For any countable ordinal $\alpha$, there exists a well-ordering of a subset of $\omega$ with order type $\alpha$.
This means the range of $g$ includes all countable ordinals.
Now define $h: \PP(\omega) \times \PP(\omega \times \omega) \hookrightarrow \omega_1$:
If $(X,R) \in W$ and $g(X,R) < \omega_1$  then $h(X,R) = g(X,R)$
Otherwise, $h(X,R) = 0$
The range of $h$ is contained in $\omega_1$ and for every  $\alpha < \omega_1$, there exists $(X,R) \in W$ s.t. $g(X,R) = \alpha$, so $h(X,R) = \alpha$
We can compose this surjection with the bijection to get a surjection from $P(\omega)$ onto $\omega_1$.

\end{enumerate}



\end{document}
