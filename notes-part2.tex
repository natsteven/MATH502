%! Author = nathanaelsteven
%! Date = 2/18/25

%% Preamble
%\documentclass[11pt]{article}
%
%% Packages
% file already inlcudes from parent:
%
%\usepackage{amsmath}
%\usepackage{graphicx}
%\usepackage[colorlinks=true, allcolors=blue]{hyperref}
%\usepackage{amsfonts}
%\usepackage{amssymb}
%\usepackage{upgreek}
%\usepackage{amsthm}
%% Document
%\begin{document}

\newcommand{\set}[1]{\{#1\}}

\emph{Note: } If there is an injection $f : A \leftrightarrow B$ then there is a surjection $g : B \leftrightarrow A$ (A,B non-empty) (these arrows may be incorrect :)
pick $a_0 \in A$
define $g(b) = $ the unqiue $a \in A$ with $f(a) = b)$ if $b \in ran(f)$ $a_0$ if $b \notin ran(f)$

\empth{Note: } (Without Axiom of Choice) Having a suhrection $g : B \twoheadrightarrow A$ does not necessarily imply that there is an injection $f : A \hookrightarrow B$

\emph{Note: } we can have $A \subseteq B$ $A \neq B$ and $A \approx B$
$B = \omega A = \omega \setminus \{0\}$
$f : B \rightarrow A f(n) = n+1$
$f$ is a bijection.

\underline{Lemma: } If $B \subseteq A$ and $A \preceq B$ then $A \approx B$

\begin{proof}
    Let $f : A \hookrightarrow B$.
    Note $A \subseteq B \subseteq f(A) = ran(f) = f``A$
    $f^n = f \dot ... \dot f$ n times and $f^0 = identity$
    Let $H_n = f^n(A) \setminus f^n(B)$
    $K_n = f^n(B) \setminus f^{n+1}(A)$
    (diagram in slides)
    $B = f^0(B)$ and also for $A$
    we get slices of A
    Note that thr $H_n$'s and $K_n$'s are pairwise disjoint.
    Let $P = \bigcap (n \in \omega) f^n(A) = \bigcup (n \in \omega) f^n(B)$
    P disjoint from Hns and Kns
    $P_1$ $H_n$ for $n \in \omega$ and Kn for $\n \in \omega$
    partition $A$ into disjoint pieces.
    $A = P \cup H_0 \cup H_1 \cup \dots \cup K_0 \cup K_1 \dots$
    $B = P \cup H_1 \cup \dots K_0 \dots$
    Define $g : A \leftrightarrow B$ by
    $g(x) = \{f(x) if x \in \cup (n) H_n , x if x \in P \cup cup (n) K_n\}$
    (diagram in slides)
    $f(H_n) = H_{n+1}$
    $f : H_n \leftrightarrow H_{n+1}$
    So $g$ is a bijection from $A$ to $B$.
\end{proof}

\underline{Ex: } $A = [0,1] B = (0,1]$
$f : A \hookrightarrow B$ by $f(x) = \frac{x+1}{2}$.
$f$ is an injection.
(diagram in slides)
$H_0 = A \setminus B$
$K_0 = B \setminus f(A)$
$K_0 = (0,1/2)$
$H_0 = f(A) \setminus f(B)$
$H_0 = 0$
$K_1 = (1/2,3/4)$
$H_1 = 1/2$
$H_2 = 3/4$
$g(x) = \frac{x+1}{2}) if x = 1-\frac{1}{n} for n \geq 1, x if not$.


\underline{\textbf{Schroder-Bernstein Theorem:}} $A \approx B$ if and only if $A \preceq B$ and $B \preceq A$
\begin{proof}
    $\implies$ is immediate.
    $\impliedby$ is more difficult.
    Let $f : A \hookrightarrow B$ and $g : B \hookrightarrow A$
    Let $B^~ = h(B) \subseteq A$
    $B \approx B^~$ via $h$.
    $B^~ \subseteq A$ and $g \circ f : A \hookrightarrow B^~$
    By Lemma there is a bijection $h : A \leftrightarrow B^~$
    \end{proof}
    (went ot bathroom)
    an example and two notes missed
\textbf{\underline{Cantor's Theorem:}} For any set $A$ $A \prec \mathbb{P}(A)$
\begin{proof}
    Let $f(x) = \set{x}$ $f : A \hookrightarrow \mathbb{P}(A)$ so $A \preceq \mathbb{P}(A)$
    Show there is no bijection from $A$ to $\mathbb{P}(A)$
    Let $f : A \rightarrow \mathbb{P}(x)$
    show $f$ is not a surjection.
    Let $D = \set{x \in A : x \notin f(x)}$
    $D \in \mathbb{P}(A)$
    claim $D \notin ran(f)$
    Suppose $D = f(x)$ for some $x$.
    $x \in D \iff x \notin f(x)$.
    $\iff x \notin D$.
    contradiction.
    so $D \notin ran(f)$.
    \end{proof}

\underline{Ex: } $\mathbb{N} \prec \mathbb{P}(\mathbb{N})$ so $\mathbb{N} \prec \mathbb{R}$.

\underline{Ex: } $ A = \mathbb{N}$.
$f : \mathbb{N} \rightarrow \mathbb{P}(\mathbb{N})$
(illustration)
$D$ can'r appear as a row in table.

\underline{Cantor's Paradox: } There is no universal set.
Suppose $V$ were a universal set.
Then $\mathbb{P}(V) \subseteq V$
$\mathbb{P}(V) \preceq V$
conradiction.

\emph{Note: } $\mathbb{P}(A) \approx 2 (left super A) = \set{f : f is a function f: A \rightarrow \set{0,1}}$
$X \subseteq A \hookrightarrow f_x(a) = \set{1 if a \in X, 0 if a \notin X}$

\emph{Note: } $2^{\mathbb{N}}$

\emph{Note: } $A (left super B) (all left super C) \approx A (left super C \times B)$ similar to $(x^y)^z = x^y^z$
$A (left super B \cup C) \approx A (left super B) \times A (left super C)$ for (something)

\defn A set $A$ is \underline{countable} if $A \preceq \omega$
$A$ is \underline{finite} if $A \preceq n$ for some $n \in \omega$
$A$ is \underline{infinite} if $A$ is not finite.
$A$ is \underline{uncountable} if $A$ is not countable.

\underline{Ex: } $\mathbb{Q}$ is countable.

\underline{EX: } $\mathbb{P}(\omega)$ is uncountable

\defn A (von-Neumann) \underline{cardinal} is an ordinal $\alpha$ so that $\xi \npreceq \preceq \alpha$ for all $\xi < \alpha$

\emph{Note: } Equivalent to saying $\nexists \xi < \alpha (\xi \approx \alpha)$
call these \underline{initial ordinals}

\underline{Ex: } $n \in \omega$ 0,1,2 \dots are cardinals.
$\omega$ is a cardinal.

\underline{Ex: } $\omega + 1, \omega + \omega$ are not cardinals.


\section{Feb 20}

\underline{Def: } A \underline{Cardinal} is an ordinal $\alpha$ so that $\forall \xi < \alpha \xi \prec \alpha$

\textbf{Thm}
(1) Every cardinal $\geq \omega$ is a limit ordinal
(2) Every natural number is a cardinal
(3) If $A$ is a set of cardinal then $supA = \bigcup A$ is a cardinal
(4) $\omega$ is a cardinal

\begin{proof}
(1) Suppose $\alpha \geq \omega$ is a cardinal
    Suppose $\alpha$ is a successor $\alpha = \delta + 1$
    So $\delta < \alpha$
    Define $f: \alpha \rightarrow \delta$ ($\alpha = \delta \cup \set{\delta}$)
    $f(\delta) = 0$
    $f(n) = n+1$ for $n \in \omega$
    $f(\xi) = \xi$ for $\omega \leq \xi < \delta$
    $f$ is a bijection, so $\alpha \appox \delta$ contradicting that $\alpha$ is a cardinal.
(2) use ordinary inducitno
    $0 = \emptyset$ is a cardinal vacuously.
    Suppose $n$ is a cardinal ($n \in \omega$).
    Suppose $S(n)=n+1$ is not a cardinal.
    Then there is some $\xi < n+1$ with $\xi \approx S(n)$
    Let $f : \xi \leftrightarrow S(n) = n \cup \set{n}$
    Can't have $\xi = 0$ so $\xi=S(m)$ for some $m$.
    $f : S(m) =m \set{m} \leftrightarrow S(n) = n \set{n}$
    Note $m < n$
    Let $a$ be such that $f(a) = n$
    If $a=m$ then $f \upharpoonright m : m \leftrightarrow n$ contradicting that $n$ is a cardinal.
    If $a \neq m$ define $g : m \leftrightarrow n$ by $g(a) = f(m)$ and $g(b) = f(b)$ for $b \neq a$
(diagram in slides)
    so $g$ is a bijection fro m to n contradicting that $n$ is a cardinal.
(3) Let $A$ be a set of  ardinal if $sup A$ is not a cardinal.
    Let $\xi < sup A$ so that $\xi \approx sup A$ (Supremum of A is least upper bound)
    There is $\alpha \in A$ with $\xi < \alpha \leq sup A$
    $\xi \preceq \alpha$ and $\alpha \preceq sup A \approx \xi$ so $\alpha \preceq \xi$
    By Schroder-Bernstein $\alpha \approx \xi$ so $\alpha$ is not a cardinal.
(4) $\omega$ is a cardinal by (2) and (3)
    Let A be $\omega$.
    $sup A = \omega$ (omega least infinite ordinal)
    so $\omega$ is a cardinal.
\end{proof}

\underline{Def:} A set is \underline{well-orderable} if there is a relation R so that R well-orders A.

\emph{Note:} A is well-orderable if and only if there is an ordinal $\alpha$ so that $A \approx \alpha$
For $\leftarrow$ If $A \approx \alpha$ let $f : A \leftrightarrow \alpha$
Define R on A by $a_1 R a_2 \leftrightarrow f(a_1) < f(a_2)$
Then $(A;R) \cong (\alpha, \in)$
So A is well-ordered by R.

\emphe{Note:} Every ordinal is in bijection with some unique cardinal.
Given $\alpha \in Ord$ let $\beta$ be the $\in$-least element of
\set{\xi \leq \alpha : \alpha \approx \xi}
So $\alpha \approx \beta$ and $\beta$ is a cardinal.
If $\xi < \beta$ then $\xi \notint$ (above set) so $\beta \neg \approx \xi$

\underline{Def:} If A is well-orderable, let $|A|$ be the least ordinal $\alpha$ so that $A \approx \alpha$ call $|A|$ the \underline{cardinality} of A.

\underline{Ex: } If $\alpha$ is a cardinal $|\alpha| = \alpha$
$|omega + omega| = \omega$

\underline{Ex: } Any set of ordinals is well-orderable.

\textbf{Lemma: } Let $A,B$ be well-orderable sets then:
(1) $|A|$ is a cardinal
(2) $A \preceq B \iff |A| \leq |B|$
(3) $A \approx B \iff |A| = |B|$
(4) $A \prec B \iff |A| < |B|$

\begin{proof}
(1) above
(2) $\implies$ suppose $f : A \hookrightarrow B$
    Let $g : B \approx |B|$
    then $g \circ f : A \hookrightarrow |B|$
    $(g \circ f)`` A \subseteq |B|$
    so well-orderable
    Type $((g \circ f)``A;\in) \leq type(|B|,\in) = |B|$
    $\impliedby$ suppose $|A| \leq |B|$
    $A \approx |A| \subseteq |B| \approx B$
    So $A \preceq B$ (an injection exists).
(3) $\implies$ Suppose $A \approx B$
    $|A| \approx A \approx B \approx |B|$
    can't have $|A| < |B|$ or $|B| < |A|$
    so $|A| = |B|$
    $\impliedby$ if $|A| = |B|$ then $A \approx |A| = |B| \approx B$
(4) Follows from (2) and (3).
\end{proof}

\emph{Lemma:} If $A$ is well-orderable and there is a surjection $f: A \twoheadrightarrow B$ then $B$ is well-orderable.
\emphe{proof} Let R be a well-ordering of A.
Define S on B by $b_1 S b_2 \iff min(f^-1(\set{b_1})) R min(f^-1(\set{b_2}))$
Then S is a well-ordering of B.

\emph{Note:} We haven't yet seen that there are uncountable cardinals.
We saw $\omega \prec \mathbb{P}(\omega)$ so $\mathbb{P}(\omega)$ is uncountable, but we don't know $\mathbb{P}(\omega)$ is well-orderable.
Need axiom of choice to see that $\mathhbb{P}(\omega)$ is well-orderable.

\textbf{Thm \underline{Hartog's Theorom} : } For every set $A$ there is a cardinal $k$ so that $k \npreceq A.$
Hence if A is a well-orderable set then there is a cardinal $k$ so that $|A| < k$.

\begin{proof}
    Let $W$ be the set of pairs of the form $(x,R)$ in $\mathbb{P}(A) \times \mathbb{P(A \times A)}$ so that $x \subseteq A, R \subseteq \A \times \A$
    and $R$ well-orders $x$.
    Thus $W$ is the set of all well-orderings of subsets of $A$.
    Note for $\alpha \in ON$
    $\alpha \preceq A \iff \alpha = type(x;R)$ for some $(x,R) \in W$
    Let $\beta = sup \set{type(x,R) + 1 : (x,R) \in W}$
    If $\alpha \preceq A$ then $\alpha < \beta$
    Let $\kappa = |\beta|$ then $\kappa \approx \beta$, $\kapp$ is a cardinal
    $\kappa \npreceq A$
\end{proof}

\underline{Def: } Let $\aleph(A)$ be the least cardinal $k$ so that $k \npreceq A$
Called the \underline{Hartog's number or aleph} of A.

\underline{Def: } For an ordinal $\alpha$ let $\alpha^+ = \aleph(\alpha)$

\emph{Note: } $|\alpha| < \alpha^+$
For a cardinal $\kappa$, $\kappa^+$ is the next biggest cardinal.

\textbf{Def: } \underline{The Aleph Sequence}
Define by recursion of $\xi \in ON$
$\aleph_0 = \omega_0 = \oemga$
$\aleph_{\xi+1} = \omega_{\xi + 1} = (\aleph_\xi)^+$
$\aleph_\lambda = \omega_\lambda = sup \set{\aleph_\xi : \xi < \lambda}$ for $\lambda$ a limit ordinal.

\emph{Note:} Generally use $\aleph_\xi$ when talking about cardinality, use $\omega_\xi$ when talking about order type (they are the same).

We shall see that all infinite cardinal appear in this sequence somewhere

\emph{Note:} $\aleph_1$ is the least uncountable ordinal/cardinal.
$\omega_1 = \set{\alpha : \alpha is a countable ordinal}$

\emph{Note:} ordinal arithmetic does not increase cardinality for infinite ordinals.
For $\alpha, \beta \geq \omega$
$|\alpha + \beta| = |\alpha \dot \beta| = |\alpha^\beta| = max(|\alpha|,|\beta|)$
cardinal exponentiation will be different than ($2^omega = 2 \dot 2 \dots = \omega$)

\textbf{Thm:} If $\alpha \geq \omega$ then $|\alpha \times \alpha| = |\alpha|$

\emph{Note:} $\alpha \times \alpha$ is well-orderable by $<_{lex}$

\begin{proof}
    It suffices to prove that when $\kappa$ is a cardinal.
    Since if $|\alpha| = \kappa$ then $\alpha \approx \kappa$
    and $\alpha \times \alpha \approx \kappa \times \kappa$
    $\kappa \preceq \kappa \times \kappa$ by $f(\xi) = (\xi, \xi)$ for $\xi < \kappa$
    Need to show $|\kappa \times \kappa | \leq \kappa$ for $\kappa$ a cardinal.
    Produce a well-ordering of $\kappa \times \kappa$ of order type $\kappa$.
    Define a relation on $ON \times ON, \triangleleft$ that is a well-ordering of $ON \times ON$
    in the sense $\in$ is a well-ordering of $ON$.
    Such that $type(\kappa \times \kappa, \triangleleft) = \kappa$ when $\kappa \subseteq \omega$ is a cardinal
    Define $(\xi_1,\xi_2) \triangleleft (\eta_1,\eta_2)$
    if $max(\xi_1,\xi_2) < max(\eta_1,\eta_2)$ \vee
    $[max(\xi_1,\xi_2) = max(\eta_1,\eta_2) \wedege max(\xi_1,\xi_2) <_{lex} max(\eta_1,\eta_2)]$
    (diagram in slides).
    $\triangleleft$ is different thean $<_{lex}$



%\end{document}