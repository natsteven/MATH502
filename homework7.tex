% Math 402/502 Homework



\documentclass[11pt]{amsart}


\pagestyle{empty}
\thispagestyle{empty}

\usepackage{graphicx}

\usepackage{amsmath}
\usepackage{amssymb}
\usepackage{latexsym}
\usepackage{amsopn}
\usepackage{amsthm}

\def\bbN{{\mathbb N}}
\def\bbR{{\mathbb R}}
\def\bbQ{{\mathbb Q}}
\def\bbZ{{\mathbb Z}}
\def\bbF{{\mathbb F}}
\def\bbE{{\mathbb E}}
\def\bbP{{\mathbb P}}

\DeclareMathOperator{\dom}{dom}

\newcommand{\set}[1]{\left\{\,#1\,\right\}}
\newcommand{\NN}{\mathbb N}
\newcommand{\ZZ}{\mathbb Z}
\newcommand{\QQ}{\mathbb Q}
\newcommand{\RR}{\mathbb R}

\DeclareMathOperator{\type}{type}
\DeclareMathOperator{\TC}{TC}
\newcommand{\leftsuperscript}[2]{{\vphantom{{#1}}}^{#2}{#1}}

\usepackage{bbold}


\newcommand{\hint}[1]{{\small \em \noindent [Hint: #1]}}
\newcommand{\note}[1]{{\small \em \noindent [Note: #1]}}


\begin{document}

\begin{center}
{\Large Math 402/502 Homework 7 -- due Friday, March 7}
\ \\
\end{center}

\ 
 
 \begin{enumerate}

\item Show that the Axiom of Choice implies that for every relation $R$ there is a partial function $F$ so that $F \subseteq R$ and $\text{dom}(F)=\text{dom}(R)$. 

\note{Such a function is called a {\it selector} for $R$.}

\newpage

\item
\begin{enumerate}
\item Show that the Axiom of Choice implies that every infinite set is Dedekind infinite.
\vfill
\item It is consistent with ZF that there are infinite sets which are Dedekind finite. Suppose $X$ is such a set, and let 
\[ Y = \{ A \subseteq X : \text{$A$ is finite}\}.\]
Show that $Y$ is also infinite and Dedekind finite.
\vfill
\end{enumerate}

\newpage


\item Show that the Axiom of Choice is equivalent to the statement:
\begin{quotation}
For any two sets $X$ and $Y$, there is either a surjection from $X$ onto $Y$ or a surjection from $Y$ onto $X$.
\end{quotation}

\hint{For one direction, use that AC is equivalent to the analogous statement for injections, and the existence of an injection from $X$ into $Y$ implies a surjection from $Y$ onto $X$. For the other direction, let $A$ be any set; Hartogs Theorem says that there is an ordinal $\kappa$ so that $\kappa \not\preccurlyeq A$. Use the given statement to show that $A$ can be well-ordered. }



\end{enumerate}



\end{document}
