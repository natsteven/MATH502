% Math 402/502 Homework



\documentclass[11pt]{amsart}


\pagestyle{empty}
\thispagestyle{empty}

\usepackage{graphicx}

\usepackage{amsmath}
\usepackage{amssymb}
\usepackage{latexsym}
\usepackage{amsopn}
\usepackage{amsthm}

\def\bbN{{\mathbb N}}
\def\bbR{{\mathbb R}}
\def\bbQ{{\mathbb Q}}
\def\bbZ{{\mathbb Z}}
\def\bbF{{\mathbb F}}
\def\bbE{{\mathbb E}}
\def\bbP{{\mathbb P}}

\DeclareMathOperator{\dom}{dom}

\newcommand{\set}[1]{\left\{\,#1\,\right\}}
\newcommand{\NN}{\mathbb N}
\newcommand{\ZZ}{\mathbb Z}
\newcommand{\QQ}{\mathbb Q}
\newcommand{\RR}{\mathbb R}
\newcommand{\PP}{\mathbb P}

\DeclareMathOperator{\type}{type}
\DeclareMathOperator{\TC}{TC}
\newcommand{\leftsuperscript}[2]{{\vphantom{{#1}}}^{#2}{#1}}

\usepackage{bbold}


\newcommand{\hint}[1]{{\small \em \noindent [Hint: #1]}}
\newcommand{\note}[1]{{\small \em \noindent [Note: #1]}}


\begin{document}

\begin{center}
{\Large Math 402/502 Homework 7 -- due Friday, March 7}
\ \\
\end{center}

\ 
 
 \begin{enumerate}

\item Show that the Axiom of Choice implies that for every relation $R$ there is a partial function $F$ so that $F \subseteq R$ and $\text{dom}(F)=\text{dom}(R)$. 

\note{Such a function is called a {\it selector} for $R$.}
\\

Any relation $R$ is a set of ordered pairs, so we can consider the set of all $x$ such that there is a $y$ so that $xRy$.
And specifically $dom(R) = \set{x | \exists y : xRy}$
Then for each $x \in dom(R)$ we define $Y = \{ y | \exists x: xRy\}$.
AC gives us a choice function so that $y = g(Y) \in Y$ for each $x \in dom(R)$.
Then we can define $F = \set{x,g(Y)| x \in dom(R)}$
Then $f$ is a selector for $R$ where $F \subseteq R$and $dom(R) = X$.

\newpage

\item
\begin{enumerate}
\item Show that the Axiom of Choice implies that every infinite set is Dedekind infinite.
\\
Let A be an infinite set.
Then define $f: \NN \rightarrow \PP(\PP(A))$ as $f(n) = \set{B \subseteq A | |B| = n}$.
Then by AC we have a choice function $g$ so that we select a $g(n) \in f(n)$ for each $n \in \NN$.
And we have $G = \set{g(n) | n \in \NN})$
Then define $H = \bigcup G$
and can define a bijection $h: \NN \rightarrow H$.
Then we define a bijection $B: A \rightarrow \setminus h(0))$ as $B(a) = \begin{cases} a & \text{if } a \notin G \\ h(n+1) & \text{if } a = h(n) \end{cases}$
Then $B$ is an injection from $A$ into $A \setminus h(0)$ which is a proper subset of $A$, so $A$ is Dedekind infinite.

\vfill
\item It is consistent with ZF that there are infinite sets which are Dedekind finite. Suppose $X$ is such a set, and let 
\[ Y = \{ A \subseteq X : \text{$A$ is finite}\}.\]
Show that $Y$ is also infinite and Dedekind finite.
\\

X is infinite and contains every finite subset of itself.
So for each $n \in \NN$ there is a subset $A_n$ of size $n$.
Each finite subset of $X$ is an element of $Y$ and there are infinitely many so $Y$ is infinite.
Now suppose Y is Dedekind infinite.
Then there is an injection $f: Y \rightarrow Y$ so that $f(Y) \subset Y$.
and there exists $Z = \set{x : x \in Y \wedge x \notin f(Y)}$
But then there exists some Dedekind infinite subset of $Y$ within $Y$.
And since $Y$ is composed of all subsets of $X$ then there would be a Dedekind infinite subset of $X$.
Contradicting the original assumption about $X$ so $Y$ is Dedekind finite.


\vfill
\end{enumerate}

\newpage


\item Show that the Axiom of Choice is equivalent to the statement:
\begin{quotation}
For any two sets $X$ and $Y$, there is either a surjection from $X$ onto $Y$ or a surjection from $Y$ onto $X$.
\end{quotation}

\hint{For one direction, use that AC is equivalent to the analogous statement for injections, and the existence of an injection from $X$ into $Y$ implies a surjection from $Y$ onto $X$. For the other direction, let $A$ be any set; Hartogs Theorem says that there is an ordinal $\kappa$ so that $\kappa \not\preccurlyeq A$. Use the given statement to show that $A$ can be well-ordered. }
\\
  AC is equivalent to the statement that for any two sets $X$ and $Y$ there is either an injection from $X$ into $Y$ or an injection from $Y$ into $X$.
  If there is an injection from $X$ into $Y$ then there is a surjection from $Y$ onto $X$.
  When $f$ is our injection, define $g(y) = \begin{cases} x & \text{if } y \in f(x) \\ \text{some fixed } x_0 \in X & \text{ when } y \notin f(x) \end{cases}$
  Similarly for the case where there is an injection from $Y$ into $X$.
  Now we will show that the surjection statemtn is equivalent to AC.
  Let A be any set and let $\kappa$ be an ordinal so that $\kappa \not\preccurlyeq A$.
  Applying the surjection statement to $A$ and $\kappa$.
  If there is a surjection $f$ from $A$ onto $\kappa$.
  For each $\beta < \kappa$, let $A_\beta = \set{a \in A | f(a) = \beta}$
  $\kappa$'s well-ordering gives us a well-ordering of $A$.
  For $a,b \in A$ we have $a < b$ if $f(a) < f(b)$ or $f(a) = f(b)$ and a precedes $b$ in the well-ordering of $A_\beta$ and so on.
  If there is a surjection $g$ from $\kappa$ onto $A$.
  Consider $B = \set{\beta < \kappa | g(\beta) = a}$
  We can construct an injection from $ran(g)$ into $\kappa$ as the $\min(B)$
  But $ran(g) = A$ giving us a contradiction to the assumption that $\kappa \not\preccurlyeq A$.
  So there is a surjection from $A$ onto $\kappa$ and $A$ can be well-ordered so AC holds.

\end{enumerate}



\end{document}
