% Math 402/502 Homework



\documentclass[11pt]{amsart}


\pagestyle{empty}
\thispagestyle{empty}

\usepackage{graphicx}

\usepackage{amsmath}
\usepackage{amssymb}
\usepackage{latexsym}
\usepackage{amsopn}
\usepackage{amsthm}


\DeclareMathOperator{\dom}{dom}
\DeclareMathOperator{\ran}{ran}

\newcommand{\set}[1]{\left\{\,#1\,\right\}}
\newcommand{\NN}{\mathbb N}
\newcommand{\ZZ}{\mathbb Z}
\newcommand{\QQ}{\mathbb Q}
\newcommand{\RR}{\mathbb R}


\usepackage{bbold}


\newcommand{\hint}[1]{{\small \em \noindent [Hint: #1]}}


\begin{document}

\begin{center}
{\Large Math 402/502 Homework 2 -- due Friday, January 31}

 \textbf{Nat Steven}
\ \\
\end{center}

 \begin{enumerate}


%\item For each structure, draw a directed graph representing the membership relation as was done in class. 
%Then determine which of the following axioms is satisfied by the structure: Extensionality, Foundation, Pairing.
%
%\ 
%\begin{enumerate}
%\item $U=\{a,b\}$, $a \in b$, $b \in a$
%\vfill
%\item $U=\{a,b,c\}$, $a \in c$, $b \in c$, $c \in c$
%\vfill
%\item $U= \{a,b,c\}$, $a \in b$, $a \in c$, $b \in c$
%\vfill
%\end{enumerate}

%\item Recall that the successor of a set $x$ is defined as $S(x) = x \cup \{ x \}$. Prove that if $S(x)=S(y)$ then $x=y$. 
%
%\hint{You will need to use the Foundation Axiom.}
%
%\newpage

\item Recall that we defined the ordered pair as $\langle x,y\rangle = \{ \{x\}, \{x,y\}\}$.


\begin{enumerate}
\item Show that if $\langle x_1,y_1 \rangle = \langle x_2, y_2 \rangle$ then $x_1=x_2$ and $y_1=y_2$.

\noindent\begin{proof}

 If $\langle x_1,y_1 \rangle = \langle x_2, y_2 \rangle$, then by definition,
 \[
  \{ \{x_1\}, \{x_1,y_1\}\} = \{ \{x_2\}, \{x_2,y_2\}\}.
 \]
 By the Axiom of Extensionality, two sets are equal if and only if they have the same elements.
 Therefore, the two sets above must contain the same elements.
 Since both sets contain two elements, there are two possibilities:

 $\{x_1\} = \{x_2\}$ and $\{x_1, y_1\} = \{x_2, y_2\}$,

 or $\{x_1\} = \{x_2, y_2\}$ and $\{x_1, y_1\} = \{x_2\}$.

\vspace{1em}
 Case 1: $\{x_1\} = \{x_2\}$ and $\{x_1, y_1\} = \{x_2, y_2\}$.
 From $\{x_1\} = \{x_2\}$, by Extensionality, we have $x_1 = x_2$.
 Substituting $x_1$ for $x_2$ in $\{x_1, y_1\} = \{x_2, y_2\}$ gives $\{x_1, y_1\} = \{x_1, y_2\}$.
 Again, by Extensionality, since $x_1$ is in both sets, we must have $y_1 = y_2$.

 \vspace{1em}
 Case 2: $\{x_1\} = \{x_2, y_2\}$ and $\{x_1, y_1\} = \{x_2\}$.
 From $\{x_1\} = \{x_2, y_2\}$, we have $x_1 = x_2$ and $x_1 = y_2$.
 From $\{x_1, y_1\} = \{x_2\}$, we have $x_1 = x_2$ and $y_1 = x_2$.
 Therefore, $x_1 = x_2$ and $y_1 = x_2 = x_1 = y_2$, so $y_1 = y_2$.

 In either case, we have shown that $x_1 = x_2$ and $y_1 = y_2$.

\end{proof}

\item Show that it would not work to define the ordered pair as $(x,y) = \{ x, \{y\}\}$, i.e., find $x_1,y_1,x_2,y_2$ for which the analogue of the previous part is false.
\vspace{1em}

Let $A = (x_1, y_1)$ and $B = (x_2, y_2)$.
From our new definition $A = \{x_1, \{y_1\}\}$ and $B = \{x_2, \{y_2\}\}$.
Now, let $x_1 = \{a_2\}$, $y_1 = a_1$, $x_2 = a_1$, and $y_2 = \{a_2\}$.
Then $A = \{\{a_2\}, a_1\}$ and $B = \{a_1, \{a_2\}\}$.
Assuming $a_1 \neq \{a_2\}$ then $x_1 \neq x_2$ and $y_1 \neq y_2$
But $A = B$ i.e. $(x_1,y_1) = (x_2,y_2)$ when $x_1 = y_2$ and $y_1 = x_2$.

\vfill
\end{enumerate}

\newpage

\item Let $f$ be a function and $A, B \subseteq \text{dom}(f)$.
\begin{enumerate}
\item Show that $f``(A \cup B) = f``A \cup f``B$.

\begin{proof}
% \[
% f``(A \cup B) = \ran(f \upharpoonright A \cup B) = \{f(x) : x \in A \cup B\}
%\]
% and
%\[
% f``A \cup f``B = \ran(f \upharpoonright A) \cup \ran(f \upharpoonright B) = \{f(a): a \in A\} \cup \{f(b) : b \in B\}
% \]
%Now $\exists y \in Y : y \in A \vee y \in B$ and $ Y = A \cup B$
% We can see that $\{f(y)\} = \{f(a)\} \cup \{f(b)\}$ as the union of the functions outputs on $A$ and $B$ would be the same as the output of the function on the union of $A$ and $B$.

Let $y \in f``(A \cup B)$.
By definition, this means $x \in A \cup B$ such that $y = f(x)$.
So we have $x \in A \vee x \in B$.
If $x \in A$ then $y \in f``A$.
If $x \in B$ then $y \in f``B$.
Therefore $y \in f``A \cup f``B \rightarrow f``(A \cup B) \subseteq f``A \cup f``B$.

Now let $y \in f``A \cup f``B$.
This means $y \in f``A \vee y \in f``B$.
If $y \in f``A$ then $\exists x \in A : y=f(x)$. Then, also $x \in A \cup B$ so $y \in f``(A \cup B)$.
If $y \in f``B$ then $\exists x \in B : y=f(x)$. Then, also $x \in A \cup B$ so $y \in f``(A \cup B)$.
Therefore $f``A \cup f``B \subseteq f``(A \cup B)$.

So we have shown $f``(A \cup B) = f``A \cup f`` B$ as the two sets are subsets of each other.

\end{proof}

\vfill
\item Show that in general $f``(A \cap B) \neq f``A \cap f``B$.

\begin{proof}
If we have a function $f$ that is non-injective, i.e. we have $x_1 \neq x_2$ and $f(x_1) = f(x_2) = y$.
Then when $x_1 \in A \wedge x_1 \notin B$ and $x_2 \in B \wedge \notin A$.
$f``a = \{y =f(x_1) : x_1 \in A\}$ and $f``b=\{y=f(x_2) : x_2 \in B\}$.
So $f``A \cap f``B = \{y\}$.
But $x_1 \neq x_2 \rightarrow x_1, x_2 \notin A \cap B \rightarrow y \notin f``(A \cap B)$.
Therefore, generally $f``(A \cap B) \neq f``A \cap f``B$.

\end{proof}

\vfill
\item Give a necessary and sufficient condition on a function $f$ so that $f``(A \cap B) = f``A \cap f``B$ for all $A, B \subseteq \text{dom}(f)$.

\vspace{1em}
If f is injective or one-to-one, then $f``(A \cap B) = f``A \cap f``B$ for all $A, B \subseteq \text{dom}(f)$.

\vfill
\end{enumerate}

\newpage

\item Let $(X, <)$ and $(Y, \prec)$ be strict total orders, i.e., $<$ is a binary relation which totally orders $X$ strictly, and similarly for $\prec$. We define the \emph{lexicographic order}, $<_{\text{lex}}$ on the set $X \times Y$ by setting
\[ (x_1,y_1) <_{\text{lex}} (x_2,y_2) \ \Leftrightarrow\ (x_1<x_2) \vee (x_1=x_2 \wedge y_1 \prec y_2) \]
and call the resulting relation the \emph{lexicopgraphic product}.
 Show that the lexicographic product of $X$ and $Y$ is a strict total order of $X \times Y$.
{\em Note that we have not yet showed that the direct produce $S \times T=\{\langle s,t \rangle : s \in S \wedge t \in T\}$ exists, but you can assume this.}

\begin{proof}
 $(X,<)$ is a strict total order so $\forall x \in X$, $<$ is irreflexive, transitive, and satisfies trichotomy for $A$.
 Similarly for $Y$.
 Now we will show that $<_{\text{lex}}$ is irreflexive, transitive, and satisfies trichotomy for $X \times Y$.
 We have $X \times Y = \{\langle x, y \rangle : x \in X \wedge y \in Y\}$
 From trichotomy we know that $\forall x_1,x_2 \in X$ either: $x_1 < x_2$, $x_2 < x_1$, or $x_1 = x_2$.
 Similarly for $(Y,\prec)$.
 Therefore for $(x_1,y_1), (x_2,y_2)$:
 Either $x_1 < x_2$, $x_2 < x_1$, or $x_1 = x_2$.
 And when $x_1 = x_2$, either $y_1 \prec y_2$, $y_2 \prec y_1$, or $y_1 = y_2$.
 Therefore $\forall (x,y) \in X \times Y$ $<_{lex}$ satisfies trichotomy.
 $\forall (x,y) \in X \times Y$
 \end{proof}
%\item Let $\omega$ be the set of all finite von Neumann ordinals, i.e., $\bbN$ (we have not yet seen that this set exists, but will shortly). Suppose $f$ is a function with $\text{dom}(f) = \omega$ with the property that $f(x)=f``(x)$ for all $x \in \omega$. Show that $f(x) = x$ for all $x \in \omega$.
%

\end{enumerate}



\end{document}
